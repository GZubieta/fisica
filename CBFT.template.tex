	\documentclass[10pt,oneside]{CBFT_article}
	% Algunos paquetes
	\usepackage{amssymb}
	\usepackage{amsmath}
	\usepackage{graphicx}
	\usepackage{libertine}
	\usepackage[bold-style=TeX]{unicode-math}
	\usepackage{lipsum}

	\usepackage{natbib}
	\setcitestyle{square}

	\usepackage{polyglossia}
	\setdefaultlanguage{spanish}


	\usepackage{CBFT.estilo} % Cargo la hoja de estilo

	% Tipografías
	% \setromanfont[Mapping=tex-text]{Linux Libertine O}
	% \setsansfont[Mapping=tex-text]{DejaVu Sans}
	% \setmonofont[Mapping=tex-text]{DejaVu Sans Mono}

	%===================================================================
	%	DOCUMENTO PROPIAMENTE DICHO
	%===================================================================

\title{CBFT}
\author{Electrodinámica}
\date{\today}

\begin{document}
\maketitle
\tableofcontents



\section{Gravedades alternativas recientes}
 
La gravedad usual, GR o {\it Einstein relativity} es la dada por la accción
\be
S = \kappa \:\int d^Dx \:\sqrt{-g}\:( R - 2\Lambda)
\ee
donde $\Lambda<0$ y $R$ es el escalar de Ricci. Es la acción de Einstein-Hilbert.
Esta teoría así como está no es renormalizable y no es posible por ende empezar a pensar en ella 
como una primera aproximación plausible a una QFT para la gravedad.
\notamargen{Acá podemos poner pequeñas observaciones de cosas a corregir, aunque creo que este margen
es demasiado pequeño para ser útil}.

De cualquier manera, es renormalizable en $D=4$ por la inclusión de términos cuadráticos en la 
curvatura. 
Eligiendo parámetros en forma apropiada se elimina el modo escalar (asociado a $\Box R$) y se hace 
massless al modo de spin-2. Esta es {Critical Gravity}
% \cite{critical_grav}
y es un análogo para $4D$ de Chiral TMG o de critical NMG con constante cosmológica que son teorías de dimensión $D=3$.

\section{Nueva sección}

The currents {\bf j} may, however, produce a non-zero macroscopic magnetic
moment, i.e. the integral $\int \bf{r\times j}\;dV$, again taken over an elementary cell,
together with $kr\gg1$, ensures the validity of the usual approximate theory of {\it Fresnel
diffraction}. Accordingly we have near the boundary $Ob$ of the complete shadow the
asymptotic expression
\be
u(r,\phi)=e^{-ikr\;\cos(\phi-\phi_0)}\frac{1-i}{\sqrt{(2\pi)}}\int_{-\infty}^{\omega}e^{i\eta 
2}d\eta,
\ee
\be
\omega=-(\phi - \phi_0 -\pi)\sqrt{\tfrac{1}{2}(kr)}.
\label{omega}
\ee

Similarly, near the boundary $Oa$ of the ``shadow'' of the reflected wave
\be
u(r,\phi)=e^{-ikr\cos(\phi-\phi_0)}+e^{-ikr\cos(\phi+\phi_0)}\frac{1-i}{\sqrt{(2\pi)}}\int_{-\infty}
^ { \omega}e^{i\eta 2}d\eta,
\ee
\eqnn{
	\omega=-(\phi + \phi_0 -\pi)\sqrt{\tfrac{1}{2}(kr)}.
	}
Tiraremos unas magias con los nuevos comandos para derivadas
\[ \dpar[2]{f(x,y,z)}{y} = k \qquad \dtot[2]{f(x,y,z)}{y} = k\]
y luego unos tensores de relatividad con índices griegos,
\[ R\hx\mu\nu.\gamma\delta. \quad R\hx.\gamma\delta\alpha\beta\rho\mu\nu. 
\quad R\hx\phantom{a}cb.d.,\]
pero la solución definitiva está lejos de ser encontrada. 

In this approximation the diffraction pattern is independent of the angle of the wedge
and of the direction of polarisation of the wave.
\[ \partial \Psi = -iE\Psi \]

\lipsum[3]

\begin{figure}[ht]
\centering
\includegraphics[width=0.5\textwidth]{./images/relgen7.pdf} 
\caption{Dando una vueltita por la variedad.} \label{fig_coord_norm}
\end{figure}

\lipsum[2]

\begin{ejemplillo}{\bf Tiene que ser así}
Un ejemplo con algo de texto como ensayo. Element of $G$ belongs to exactly one left coset of 
$H$. Si ahora me tiro una ecuación, tipo
\[ \log(x)\equiv \int_1^x t^{-1}dt\]
Moreover each left coset of $H$ contains $|H|$ elements. Each element~$x$ of $G$ belongs to at 
least one left coset of $H$ in $G$ (namely the coset $xH$), and no element can belong to two 
distinct left cosets of $H$ in $G$ (see Lemma).
\label{etiquetaejemplo2}
\end{ejemplillo}

\lipsum[1]


\begin{ejemplillo}{\bf Tiene que ser así}
Un ejemplo con algo de texto como ensayo. Element of $G$ belongs to exactly one left coset of 
$H$. Si ahora me tiro una ecuación, tipo
\[ \log(x)\equiv \int_1^x t^{-1}dt\]
Moreover each left coset of $H$ contains $|H|$ elements. Each element~$x$ of $G$ belongs to at 
least one left coset of $H$ in $G$ (namely the coset $xH$), and no element can belong to two 
distinct left cosets of $H$ in $G$ (see Lemma).
\label{eje3}
\end{ejemplillo}

Como vimos en los ejemplos \eqref{eje3} y \eqref{etiquetaejemplo2} para la ecuación 
\eqref{omega} todo es trivialmente cierto y se ve de manera {\it straightforward}.

\begin{problemas}
	\item Calcular una cosa bajo cierto estado de ánimo.
Ahora otra vez la ecuación,
	\[ \log(x)\equiv \int_1^x t^{-1}dt\]
	Moreover each left coset of $H$ contains $|H|$ elements. Each 
	element~$x$ of $G$ belongs to at least one left coset
	\item Decir, con argumentos, si $\langle T_{\mu\nu}\rangle$ es o no posible. 
\end{problemas}


\begin{notasfinales}

\item {\bf Titulete de la nota}
Acá irán las observaciones de final de capítulo y aclaraciones 
pertinentes. 
Es imposible saber cómo va a continuar el tipo cuando llegue a su costado
de renglón.
\item {\bf Una fórmula}
Acá otras observaciones, como que
\[
\frac{ds}{dt}=-c_1+c_2s
\]
y luego continuamos con cualquier otra cosa citando al único libro existente \citep{Brekho}.
\lipsum[19]
\end{notasfinales}



\bibliographystyle{CBFT-apa-good}	% (uses file "apa-good.bst")
\bibliography{CBFT.Referencias} % La base de datos bibliográfica

\end{document}
