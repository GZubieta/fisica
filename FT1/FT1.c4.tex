	\documentclass[10pt,oneside]{CBFT_book}
	% Algunos paquetes
	\usepackage{amssymb}
	\usepackage{amsmath}
	\usepackage{graphicx}
	\usepackage{libertine}
	\usepackage[bold-style=TeX]{unicode-math}
	\usepackage{lipsum}

	\usepackage{natbib}
	\setcitestyle{square}

	\usepackage{polyglossia}
	\setdefaultlanguage{spanish}


	\usepackage{CBFT.estilo} % Cargo la hoja de estilo

	% Tipografías
	% \setromanfont[Mapping=tex-text]{Linux Libertine O}
	% \setsansfont[Mapping=tex-text]{DejaVu Sans}
	% \setmonofont[Mapping=tex-text]{DejaVu Sans Mono}

	%===================================================================
	%	DOCUMENTO PROPIAMENTE DICHO
	%===================================================================

\begin{document}

% =================================================================================================
\chapter{Expansión en un campo multipolar}
% =================================================================================================

% =================================================================================================
\section{Desarrollo dipolar del campo magnético}
% =================================================================================================

El potencial vector de un dipolo es
\[
	\vb{A}(\vb{x}) = \frac{\vb{v}\times(\vb{x}-\vb{x}')}{|\vb{x}-\vb{x}'|^3} = \vb{m} \times \Nabla 
		\frac{1}{|\vb{x}-\vb{x}'|}
\]
\[
	\vb{A}(\vb{x}) = \int_{V'} \vb{\mathcal{M}}(\vb{x}') \times \Nabla 
			\left(\frac{1}{|\vb{x}-\vb{x}'|}\right) dV'
\]
Es el potencial vector de una distribución de momento dipolar magnético con densidad $ \vb{M}(\vb{x}')$

\[
	\vb{A}(\vb{x}) = \int_{V'} \frac{\rotorm{\vb{M}}}{|\vb{x}-\vb{x}'|} dV' +
			\int_{S'} \frac{\vb{M}\times\hat{n}}{|\vb{x}-\vb{x}'|} dS'
\]
y se pueden pensar como corrientes $\vb{J}_M$ y $\vb{g}_M$,
\[
	\vb{A}(\vb{x}) = \frac{1}{c} \int_{V'} \frac{\vb{J}_M}{|\vb{x}-\vb{x}'|} dV' +
			\frac{1}{c} \int_{S'} \frac{\vb{g}_M}{|\vb{x}-\vb{x}'|} dS'
\]


% =================================================================================================
\section{Desarrollo multipolar}
% =================================================================================================






% =================================================================================================
\section{Dipolo}
% =================================================================================================









% =================================================================================================
\section{Campo dipolar}
% =================================================================================================













% \bibliographystyle{CBFT-apa-good}	% (uses file "apa-good.bst")
% \bibliography{CBFT.Referencias} % La base de datos bibliográfica

\end{document}
