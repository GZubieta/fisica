	\documentclass[10pt,oneside]{CBFT_book}
	% Algunos paquetes
	\usepackage{amssymb}
	\usepackage{amsmath}
	\usepackage{graphicx}
	\usepackage{libertine}
	\usepackage[bold-style=TeX]{unicode-math}
	\usepackage{lipsum}

	\usepackage{natbib}
	\setcitestyle{square}

	\usepackage{polyglossia}
	\setdefaultlanguage{spanish}


	\usepackage{CBFT.estilo} % Cargo la hoja de estilo

	% Tipografías
	% \setromanfont[Mapping=tex-text]{Linux Libertine O}
	% \setsansfont[Mapping=tex-text]{DejaVu Sans}
	% \setmonofont[Mapping=tex-text]{DejaVu Sans Mono}

	%===================================================================
	%	DOCUMENTO PROPIAMENTE DICHO
	%===================================================================

\begin{document}

\chapter{Ecuaciones de Hamilton-Jacobi}

\[
	q_i \longrightarrow Q_i \equiv \beta_i \qquad p_i \longrightarrow P_i \equiv \alpha_i
\]
Pasamos a unas nuevas coordenadas y momentos $(\beta_i,\alpha_i)$ que son constantes. Entonces
la acción es del tipo $F_2$, i.e.
\[
	S = S(q_i, \alpha_i, t).
\]
Entonces
\be
	\dpar{S}{q_i} = p_i \qquad \dpar{S}{\alpha_i} = \beta_i \qquad \dpar{S}{t} = H - K  
\label{ecshamjac}	
\ee
donde 
\[
	H(q_i,p_i,t) - \dpar{S}{t} = K = 0
\]
y esto lleva a la ecuación de Hamilton-Jacobi,
\[
	H(q_i,p_i,t) - \dpar{S}{t} = 0
\]
que no es otra cosa que una ecuación en derivadas parciales (PDE). Notemos que 
\[
	\dpar{S}{q_i} = p_i(q_i,\alpha_i,t) \qquad \dpar{S}{\alpha_i} = \beta_i(q_i,\alpha_i,t)
\]
y además que Hamilton-Jacobi tiene solución si el problema es totalmente separable.
Si $H=H(q_i,\alpha_i)$ entonces $dH/dt = \partial H/\partial t=0$ y en ese caso es $H=cte.$ y
podemos poner $H=\alpha_1$.
Entonces
\[
	\dpar{S}{t} = -\alpha_1 \quad \longrightarrow \quad S=W(q_i,\dpar{S}{q_i}) -\alpha_1 t .
\]

Se procede en la misma forma con cada coordenada hasta obtener $S$.

Podemos ver que si $\alpha_1 = \alpha_1(\alpha_i)$, y me quedo con $H=\alpha_1 \equiv K$ entonces
\[
	\dpar{K}{\alpha_i} = a = \dot{Q}_i \longrightarrow Q_i = \beta = a t + \beta_0 
\]
\[
	\dpar{K}{\beta_i} = 0 = -\dot{P}_i \longrightarrow P_i = \alpha_i (ctes.).
\]

La $\alpha_1$ no puede depender de $q_i$ pues si se tuviera $\partial \alpha_1 /\partial q_i \neq 0$ 
no sería constante $\alpha_1$ pues $\dot{q}\neq 0$.

Luego, invirtiendo las ecuaciones \eqref{ecshamjac} determinamos las trayectorias
\[
	q_i = q_i(\alpha_i, \beta_i, t).
\]

Además, si el problema es totalmente separable, entonces
\[
	S = \sum_i^N \; W(q_i, \alpha_1,...,\alpha_n) - \alpha_1 t
\]
y tendré tantas constantes de movimiento como grados de libertad. La solución se compone de problemas
independientes en una variable.

\subsection{Preservación del volumen en una transformación canónica}






% \bibliographystyle{CBFT-apa-good}	% (uses file "apa-good.bst")
% \bibliography{CBFT.Referencias} % La base de datos bibliográfica

\end{document}
