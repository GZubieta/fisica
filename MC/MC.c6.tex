	\documentclass[10pt,oneside]{CBFT_article}
	% Algunos paquetes
	\usepackage{amssymb}
	\usepackage{amsmath}
	\usepackage{graphicx}
	\usepackage{libertine}
	\usepackage[bold-style=TeX]{unicode-math}
	\usepackage{lipsum}

	\usepackage{natbib}
	\setcitestyle{square}

	\usepackage{polyglossia}
	\setdefaultlanguage{spanish}


	\usepackage{CBFT.estilo} % Cargo la hoja de estilo

	% Tipografías
	% \setromanfont[Mapping=tex-text]{Linux Libertine O}
	% \setsansfont[Mapping=tex-text]{DejaVu Sans}
	% \setmonofont[Mapping=tex-text]{DejaVu Sans Mono}

	%===================================================================
	%	DOCUMENTO PROPIAMENTE DICHO
	%===================================================================

\title{CBFT Mecánica clásica}
\author{Cuerpos rígidos}
\date{\today}

\begin{document}
\maketitle
\tableofcontents


% =================================================================================================
\section{Cuerpos rígidos}
% =================================================================================================

Los vínculos constituyen la condición de rigidez,
\[
	|\vb{r}_i \vb{r}_j | = d_{ij}	\qquad i \neq j
\]

Del discreto al continuo
\[
	\vb{R} = \frac{\sum_i m_i\vb{r}_i}{\sum_i m_i} \longrightarrow 
	\vb{R} = \frac{\int \rho \vb{r}_i dv }{\int \rho dv} 
\]




\bibliographystyle{CBFT-apa-good}	% (uses file "apa-good.bst")
\bibliography{CBFT.Referencias} % La base de datos bibliográfica

\end{document}
