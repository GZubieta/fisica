	\documentclass[10pt,oneside]{CBFT_article}
	% Algunos paquetes
	\usepackage{amssymb}
	\usepackage{amsmath}
	\usepackage{graphicx}
	\usepackage{libertine}
	\usepackage[bold-style=TeX]{unicode-math}
	\usepackage{lipsum}

	\usepackage{natbib}
	\setcitestyle{square}

	\usepackage{polyglossia}
	\setdefaultlanguage{spanish}


	\usepackage{CBFT.estilo} % Cargo la hoja de estilo

	% Tipografías
	% \setromanfont[Mapping=tex-text]{Linux Libertine O}
	% \setsansfont[Mapping=tex-text]{DejaVu Sans}
	% \setmonofont[Mapping=tex-text]{DejaVu Sans Mono}

	%===================================================================
	%	DOCUMENTO PROPIAMENTE DICHO
	%===================================================================

\title{CBFT Mecánica clásica}
\author{Mecánica lagrangiana}
\date{\today}

\begin{document}
\maketitle
\tableofcontents

% =================================================================================================
\section{Principio de los trabajos virtuales}
% =================================================================================================
Escribimos las ecuaciones de Newton para un sistema de partículas,
\notamargen{Esto es sumamente sketchi, debemos leer la carpeta de la cursada y luego la
teoría.}
\[
m_i \vb{a}_i = \vb{F}_i = \vb{F}_i^a + \vb{F}_i^v
\]
pero sabiendo que el momento viene de las fuerzas aplicadas,
\[
m_i \vb{a}_i = \dot{\vb{P}}_i
\]
de manera que 
\[
\dot{\vb{F}}_i - \vb{F}_i^a - \vb{F}_i^v = 0,
\]
y entonces, sumando en las $N$ partículas del sistema
\[
\sum_i^N \left( \dot{\vb{P}}_i - \vb{F}_i^a - \vb{F}_i^v \right) \cdot \delta\vb{r}_i  = 0
\]
donde $\delta\vb{r}_i$ son desplazamientos virtuales. Si hacemos estos desplazamientos
compatibles con los vínculos
\[
\sum_i^N \left( \dot{\vb{P}}_i - \vb{F}_i^a \right) \cdot \delta\vb{r}_i - \sum_i^N  \vb{F}_i^v  \cdot \delta\vb{r}_i  = 0
\]
donde el último término es nulo debido a que la fuerza de vínculos son perpendiculares
a los desplazamientos virtuales, es decir 
\[
\vb{F}_i^v \perp \delta\vb{r}_i
\]
si es que, por supuesto, los $\delta\vb{r}_i$ son compatibles con los vínculos.

Esto nos deja entonces, el Principio de los Trabajos Virtuales,
\[
\sum_i^N \left( \dot{\vb{P}}_i - \vb{F}_i^a \right) \cdot \delta\vb{r}_i = 0 
\]
donde como son independientes entonces se sigue que
\[
\dot{\vb{P}}_i - \vb{F}_i^a = 0 \quad \forall i
\]

\begin{notas}{Relación vínculos y desplazamientos:}
El hecho de que la fuerza de vínculo sea perpendicular a los desplazamientos puede
verse a partir de que la ecuación de vínculo en un sistema toma la forma
\notamargen{¿Y esta magia? Hay que aclarar realmente que sea así como se dice que es.}
\[
f(\vb{r}_i) - K = 0 
\]
luego, derivando implícitamante cada ecuación y sumando (si se nos permite un pequeño
abuso de notación)
\[
\sum_i^N \dpar{f}{\vb{r}_i} d\vb{r}_i = 0 
\]
pero esto no es otra cosa que
\[
\nabla f \cdot \vb{\delta r} = 0
\]
donde debemos entender al gradiente y al vector $\vb{\delta r}$ como $N$ dimensionales.
\end{notas}

% =================================================================================================
\section{Construcción del lagrangiano}
% =================================================================================================

Consideremos un sistema de $N$ partículas, $k$ ecuaciones de vínculo y por ende $3N - k$ grados de libertad
(estamos en 3 dimensiones).

Tenemos $N$ relaciones
\[
\vb{r}_i = \vb{r}_i(q_1,q_2,...,q_{3N-k},t)
\]
entonces una variación serán
\[
\delta \vb{r}_i =  \sum_{j=1}^{3N-k} \left( \dpar{\vb{r}_i}{q_j} \right) \delta q_j + \dpar{\vb{r}_i}{t}\delta t
\]
donde el último $\delta t$ es nulo por ser un desplazamiento virtual de manera que
\[
\delta \vb{r}_i =  \sum_{j=1}^{3N-k} \left( \dpar{\vb{r}_i}{q_j} \right) \delta q_j.
\]

Por otro lado
\[
\sum_i^N \dot{\vb{P}}_i \cdot \delta \vb{r}_i - \sum_i^N  \vb{F}_i^a \cdot \delta \vb{r}_i = 0
\]
y se puede reescribir el primer término como
\[
\dot{\vb{P}}_i \cdot \delta \vb{r}_i = m_i \dtot{\vb{v}_i}{t}\sum_{j=1}^{3N-k} \left( \dpar{\vb{r}_i}{q_j} \right) \delta q_j
\]
resultando
\[
\sum_i^N m_i \dtot{\vb{v}_i}{t} \cdot \sum_{j=1}^{3N-k} \left( \dpar{\vb{r}_i}{q_j} \right) \delta q_j
- \sum_i^N  \vb{F}_i^a \cdot \delta \vb{r}_i = 0
\]

La idea ahora es reescribir todo en términos más convenientes, para que aparezca un término multiplicado
a una variación arbitraria. De esta manera quedará una sumatoria de un sumando multiplicado por una
variación igualada a cero. No cabe otra posibilidad que el sumando sea nulo para cada índice de la suma.
\notamargen{Escrito muy mal este texto. La idea es clara, no obstante: hay que purificarla}

Consideremos la derivada total de 
\[
\frac{d}{dt}\left( m_i\vb{v}_i\dpar{\vb{r}_i}{q_j} \right) =
m_i \dtot{\vb{v}_i}{t}\dpar{\vb{r}_i}{q_j} + m_i \vb{v}_i \frac{d}{dt}\left(\dpar{\vb{r}_i}{q_j}\right).
\]
Pero la diferencial del vector $\vb{r}_i$ es (notemos que no es una variación)
\[
d\vb{r}_i = \sum_{j=1}^{3N-k} \left( \dpar{\vb{r}_i}{q_j} \right) dq_j + \dpar{\vb{r}_i}{t} dt
\]
y entonces
\[
\dot{\vb{r}}_i = \vb{v}_i = \sum_{j=1}^{3N-k} \left( \dpar{\vb{r}_i}{q_j} \right) \dot{q}_j + \dpar{\vb{r}_i}{t}.
\]
La derivada de la velocidad de la partícula $i$-ésima respecto a la coordenada $l$-ésima es
\[
\dpar{\vb{v}_i}{\dot{q}_l} = \dpar{\vb{r}_i}{q_l} = \frac{\partial \vb{r}_i/\partial t}{\partial q_l/\partial t}.
\]
Si derivamos nuevamente
\[
\frac{\partial}{\partial q_l} \left( \dtot{\vb{r}_i}{t} \right) =
\dpar{\vb{v}_i}{q_l} = \sum_{j=1}^{3N-k} \dparcru{\vb{r}_i}{q_j}{q_l} \dot{q}_j + \dparcru{\vb{r}_i}{t}{q_l}.
\]
\[
\frac{d}{dt} \left( \dpar{\vb{r}_i}{q_l} \right) = 
\frac{d}{dt} \left( \sum_{j=1}^{3N-k} \dparcru{\vb{r}_i}{q_j}{q_l} dq_j + \dparcru{\vb{r}_i}{t}{q_l} dt \right) 
\]
de tal manera que 
\[
\frac{d}{dt} \left( \dpar{\vb{r}_i}{q_l} \right) = \dpar{\vb{v}_i}{q_l}
\]

Volvemos ahora a la eq III y 
\[
\sum_i^N \sum_{j=1}^{3N-k} \left[ 
\frac{d}{dt} \left( m_i \vb{v}_i \dpar{\vb{r}_i}{q_j} \right) -  m_i \vb{v}_i \frac{d}{dt}\left( \dpar{\vb{v}_i}{q_j} \right)
\right] \delta q_j
\]
y este corchete lo reescribimos como 
\[
\sum_i^N \sum_{j=1}^{3N-k} \left[ 
\frac{d}{dt} \left( m_i \vb{v}_i \dpar{\vb{v}_i}{\dot{q}_j} \right) -  m_i \vb{v}_i \dpar{\vb{v}_i}{q_j} 
\right] \delta q_j
\]

\[
\sum_i^N \sum_{j=1}^{3N-k} \left\{ 
\frac{d}{dt} \left[ \frac{\partial}{\partial \dot{q}_j} \left( \frac{m_i}{2} \vb{v}_i^2 \right) \right] - 
 \frac{\partial}{\partial q_j} \left( \frac{m_i}{2} \vb{v}_i^2 \right)
\right\} \delta q_j
\]

Ahora introducimos la sumatoria en $i$ hacia adentro de ambos términos,
\[
\sum_{j=1}^{3N-k} \left\{ 
\frac{d}{dt} \left[ \frac{\partial}{\partial \dot{q}_j} \left( \sum_i^N \frac{m_i}{2} \vb{v}_i^2 \right) \right] - 
 \frac{\partial}{\partial q_j} \left( \sum_i^N \frac{m_i}{2} \vb{v}_i^2 \right)
\right\} \delta q_j
\]
de modo que dentro de los paréntesis resulta $T$, luego 
\[
\sum_i^N \dot{\vb{P}}_i \cdot \delta \vb{r}_i = 
\sum_{j=1}^{3N-k} \left\{ 
\frac{d}{dt} \left[ \frac{\partial}{\partial \dot{q}_j} \left( T \right) \right] - 
 \frac{\partial}{\partial q_j} \left( T \right) \right\} \delta q_j
\]

\[
\sum_i^N \dot{\vb{P}}_i \cdot \delta \vb{r}_i = 
\sum_{j=1}^{3N-k} \sum_i^N \vb{F}_i^a \cdot \dpar{\vb{r}_i}{q_j} \delta q_j =  
\sum_{j=1}^{3N-k} \sum_i^N Q_j \delta q_j
\]
siendo $Q_j$ la fuerza generalizada. Entonces
\[
\sum_{j=1}^{3N-k} \left\{ \frac{d}{dt}
\left[ \frac{\partial}{\partial \dot{q}_j} \left( T \right) \right] - \frac{\partial}{\partial q_j} \left( T \right) - Q_j 
\right\} \delta q_j =  0.
\]

Si suponemos que las fuerzas son conservativas entonces 
\[
Q_j \delta q_j = -\dpar{V}{q_j}\delta q_j
\]
y como $V=V(\vb{r}_1,...,\vb{r}_n)$ se tiene 
\[
V = \sum_i^N  \dpar{V}{r_i} \delta \vb{r}_i = \dpar{V}{\vb{r}_i} \cdot \dpar{\vb{r}_i}{q_j} \delta q_j =
\]
pero 
\[
Q_j = - \dpar{V}{q_j}
\]
y entonces 
\[
\sum_{j=1}^{3N-k} \left\{ 
\frac{d}{dt} \left( \dpar{T}{\dot{q}_j} \right) - \frac{\partial}{\partial q_j} \left( T - V \right) \right\} \delta q_j =  0.
\]

Definimos como 
\[
\Lag \equiv T - V
\]
y entonces podemos escribir
\[
\sum_{j=1}^{3N-k} \left[
\frac{d}{dt} \left( \dpar{\Lag}{\dot{q}_j} \right) -  \dpar{\Lag}{q_j} \right] \delta q_j =  0.
\]

Si existieran fuerzas que no provienen de un potencial entonces
\[
Q_j + Q_j^{NC} = -\dpar{V}{q_j} + Q_j^{NC}
\]
y finalmente 
\[
\sum_{j=1}^{3N-k} \left[
\frac{d}{dt} \left( \dpar{\Lag}{\dot{q}_j} \right) -  \dpar{\Lag}{q_j} \right] \delta q_j = 
\sum_{j=1}^{3N-k} Q_j^{NC} \delta q_j
\]

Como esto vale para todo grado de libertad $l$ llegamos a
\[
\frac{d}{dt} \left( \dpar{\Lag}{\dot{q}_j} \right) -  \dpar{\Lag}{q_j} = Q_j^{NC}
\]
que son las ecuaciones de Euler-Lagrange. Este es el resultado más importante del capítulo.

% =================================================================================================
\section{Invariancia del lagrangiano ante adición de una derivada total}
% =================================================================================================

Sea una función de las coordenadas y del tiempo $F=F(q_i,t)$ que sumamos al lagrangiano $\Lag$, de modo que
\[
\Lag'=\Lag + \dtot{F}{t} 
\]
y las ecuaciones de Euler-Lagrange para este nuevo lagrangiano son
\[
\frac{d}{dt}\left(\dpar{\Lag'}{\dot{q}_j}\right) - \dpar{\Lag'}{q_j} = 0
\]
\[
\dtot{}{t}\left(\dpar{\Lag}{\dot{q}_j} + \dpar{}{\dot{q}_j}\left(\dtot{F}{t}\right)\right) -
\dpar{\Lag}{q_j} - \dpar{}{q_j}\left( \dtot{F}{t}\right) = 0 
\]
\[
\dtot{}{t}\left(\dpar{\Lag}{\dot{q}_j}\right) - \dpar{\Lag}{q_j} + 
\dtot{}{t}\left(\dpar{}{\dot{q}_j}\left(\dtot{F}{t}\right)\right) 
- \dpar{}{q_j}\left( \dtot{F}{t}\right) = 0 
\]

Ahora es necesario escribir la derivada total de $F$,
\[
\dtot{F}{t} = 	\sum_j \dpar{F}{q_j}\dtot{q_j}{t} + \dpar{F}{t} =
			\sum_j \dpar{F}{q_j}\dot{q}_j + \dpar{F}{t}
\]
y ver que
\[
\dpar{}{\dot{q}_j}\left(\dtot{F}{t}\right) = \dpar{F}{q_j} \qquad\qquad
\dpar{}{q_j}\left(\dtot{F}{t}\right) = \dpar[2]{F}{q_j} \dot{q}_j + \dparcru{F}{t}{q_j} 
\]

Luego, usando esta información, resulta que los términos que surgen de la adición de la derivada total de $F$ resultan 
ser
\[
\dtot{}{t}\left( \dpar{}{\dot{q}_j}\left(\dtot{F}{t}\right)\right) - \dpar{}{q_j}\left( \dtot{F}{t}\right) = 
\dtot{}{t}\left( \dpar{F}{q_j} \right) - \dpar{}{q_j}\left( \dtot{F}{t}\right)
\]
\[
\dtot{}{t}\left( \dpar{F}{q_j} \right) - \dpar{}{q_j}\left( \dtot{F}{t}\right) =
\dpar[2]{F}{q_j} \dot{q}_j + \dparcru{F}{q_j}{t} - \dpar{}{q_j}\left( \dtot{F}{t}\right)
\]
y si aceptamos que $F$ es de clase C$^2$ se tiene
\[
\dpar[2]{F}{q_j} \dot{q}_j + \dparcru{F}{q_j}{t} - \dpar{}{q_j}\left( \dtot{F}{t}\right)=0
\]
de modo que las ecuaciones de Euler Lagrange no se modifican si añadimos una derivada total respecto del tiempo de una 
función de $q_j,t$.



% =================================================================================================
% =================================================================================================
% =================================================================================================
% =================================================================================================
% =================================================================================================

% =================================================================================================
% =================================================================================================
% =================================================================================================
% =================================================================================================
% =================================================================================================
% =================================================================================================
% =================================================================================================
% =================================================================================================
% =================================================================================================


% \section{Invariancia del lagrangiano ante adición de una derivada total}
% 
% Sea una función de las coordenadas y del tiempo $F=F(q_i,t)$ que sumamos al lagrangiano $\Lag$, de modo que
% \[\Lag'=\Lag + \dtot{F}{t} \]
% y las ecuaciones de Euler-Lagrange para este nuevo lagrangiano son
% \[ \frac{d}{dt}\left(\dpar{\Lag'}{\dot{q}_j}\right) - \dpar{\Lag'}{q_j} = 0 \]
% \[ \dtot{}{t}\left(\dpar{\Lag}{\dot{q}_j} + \dpar{}{\dot{q}_j}\left(\dtot{F}{t}\right)\right) - \dpar{\Lag}{q_j} - 
% \dpar{}{q_j}\left( \dtot{F}{t}\right) = 0 \]
% \[ \dtot{}{t}\left(\dpar{\Lag}{\dot{q}_j}\right) - \dpar{\Lag}{q_j} + 
% \dtot{}{t}\left(\dpar{}{\dot{q}_j}\left(\dtot{F}{t}\right)\right) - \dpar{}{q_j}\left( \dtot{F}{t}\right) = 0 \]
% 
% Ahora es necesario escribir la derivada total de $F$,
% \[  \dtot{F}{t} = \sum_j \dpar{F}{q_j}\dtot{q_j}{t} + \dpar{F}{t} = \sum_j \dpar{F}{q_j}\dot{q}_j + \dpar{F}{t} \]
% y ver que
% \[  \dpar{}{\dot{q}_j}\left(\dtot{F}{t}\right) = \dpar{F}{q_j} \qquad\qquad
% \dpar{}{q_j}\left(\dtot{F}{t}\right) = \dpar[2]{F}{q_j} \dot{q}_j + \dparcru{F}{t}{q_j} \]
% 
% Luego, usando esta información, resulta que los términos que surgen de la adición de la derivada total de $F$ resultan 
% ser
% \[  \dtot{}{t}\left( \dpar{}{\dot{q}_j}\left(\dtot{F}{t}\right)\right) - \dpar{}{q_j}\left( \dtot{F}{t}\right) = 
% \dtot{}{t}\left( \dpar{F}{q_j} \right) - \dpar{}{q_j}\left( \dtot{F}{t}\right)\]
% \[ \dtot{}{t}\left( \dpar{F}{q_j} \right) - \dpar{}{q_j}\left( \dtot{F}{t}\right) =
% \dpar[2]{F}{q_j} \dot{q}_j + \dparcru{F}{q_j}{t} - \dpar{}{q_j}\left( \dtot{F}{t}\right) \]
% y si aceptamos que $F$ es de clase C$^2$ se tiene
% \[\dpar[2]{F}{q_j} \dot{q}_j + \dparcru{F}{q_j}{t} - \dpar{}{q_j}\left( \dtot{F}{t}\right)=0\]
% de modo que las ecuaciones de Euler Lagrange no se modifican si añadimos una derivada total respecto del tiempo de una 
% función de $q_j,t$.




\bibliographystyle{CBFT-apa-good}	% (uses file "apa-good.bst")
\bibliography{CBFT.Referencias} % La base de datos bibliográfica

\end{document}
