	\documentclass[10pt,oneside]{CBFT_book}
	% Algunos paquetes
	\usepackage{amssymb}
	\usepackage{amsmath}
	\usepackage{graphicx}
	\usepackage{libertine}
	\usepackage[bold-style=TeX]{unicode-math}
	\usepackage{lipsum}

	\usepackage{natbib}
	\setcitestyle{square}

	\usepackage{polyglossia}
	\setdefaultlanguage{spanish}


	\usepackage{CBFT.estilo} % Cargo la hoja de estilo

	% Tipografías
	% \setromanfont[Mapping=tex-text]{Linux Libertine O}
	% \setsansfont[Mapping=tex-text]{DejaVu Sans}
	% \setmonofont[Mapping=tex-text]{DejaVu Sans Mono}

	%===================================================================
	%	DOCUMENTO PROPIAMENTE DICHO
	%===================================================================

\begin{document}

\chapter{Ecuaciones de Hamilton}

Se pasa de las variables $(q, \dot{q})$ hacia el par $(q,p)$ con 
\[
	p = \dpar{\Lag}{\dot{q}}
\]
Se parte del 
\[
	H(q_i, p_i, t) = \sum_{i}^{3N-k} p_i \dot{q}_i - \Lag(q_i, \dot{q}_i, t)
\]
y consideramos el diferencial
\[
	dH = \sum_i p_i d\dot{q}_i + \dot{q}_i dp_i - \dpar{\Lag}{q_i} dq_i - \dpar{\Lag}{\dot{q}_i}d\dot{q}_i - \dpar{\Lag}{t}dt
\]
\[
	dH = \sum_i \dot{q}_i dp_i - \frac{d}{dt}\left( \dpar{\Lag}{\dot{q}_i} \right) dq_i - \dpar{\Lag}{t}dt
\]
\[
	dH = \sum_i \dot{q}_i dp_i - \dot{p}_i dq_i - \dpar{\Lag}{t}dt
\]
se deducen entonces,
\[
	\dpar{H}{p_i} = \dot{q}_i \qquad \dpar{H}{q_i} = -\dot{p}_i \qquad \dpar{H}{t} = -\dpar{\Lag}{t}
\]
que son las ecuaciones de Hamilton. Donde $(p,q)$ son $2N$ grados de libertad del sistema llamados las variables canónicas.
Si $V\neq V(\dot{q})$ y los vínculos no dependen del tiempo entonces $T=T_2$ (la energía cinética es cuadrática en las 
velocidades) y $H=E$.

% =================================================================================================
\section{Transformación canónica del hamiltoniano}
% =================================================================================================

Es una transformación que verifica
\[
	H \longrightarrow K
\]
donde $K=K(Q_i,P_i,t)$ es un nuevo hamiltoniano proveniente de
\[
	\dpar{H}{p_i} = \dot{q}_i \longrightarrow \dot{Q}_i = \dpar{K}{P_i}
\]
\[
	-\dpar{H}{q_i} = \dot{p}_i \longrightarrow \dot{P}_i = -\dpar{K}{Q_i}
\]
y ahora usamos el Principio Variacional de Hamilton,
\[
	S = \int_{t_i}^{t_f} \Lag dt = \int_{t_i}^{t_f}  \left\{ \sum_i p_i \dot{q}_i - H(p_i,q_i,t) \right\} dt
\]
\[
	\delta S = \sum p_i \delta \dot{q}_i +  \dot{q}_i \delta p_i  - \dpar{H}{p_i}\delta p_i 
	-\dpar{H}{q_i}\delta q_i  - \dpar{H}{t}\delta t
\]
pero el último término es nulo porque la variación es a tiempo fijo.
Usando las ecuaciones de Euler-Lagrange en el primer término resulta que 
\[
	\delta S = \int_{t_i}^{t_f}  \left\{ \sum_i \left(-\dot{p}_i -\dpar{H}{q_i} \right) \delta q_i +
	\left( \dot{q}_i - \dpar{H}{p_i} \right) \delta p_i + \frac{d}{dt}\left( p_i \delta q_i \right) \right\} dt
\]
y llego pidiendo que sea extremo $S$ a las ecuaciones de Hamilton (dos primeros paréntesis) mientras que el 
último término resulta 
\[
	\int_{t_i}^{t_f}  \left\{ \frac{d}{dt}\left( p_i \delta q_i \right) \right\} dt =
	\left. p_i \delta q_i \right|_{t_i}^{t_f}.
\]

Entonces, usando la misma idea que el $\Lag$ se tiene 
\[
	\Lag' = \Lag + \frac{dF}{dt}
\]
siendo $F$ una función generatriz. Luego,
\[
	\sum p_i \dot{q}_i - H(p_i,q_i,t) = \sum P_i\dot{Q}_i - K(P_i,Q_i,t) + \dtot{F}{t}
\]









% \bibliographystyle{CBFT-apa-good}	% (uses file "apa-good.bst")
% \bibliography{CBFT.Referencias} % La base de datos bibliográfica

\end{document}
