	\documentclass[10pt,oneside]{CBFT_book}
	% Algunos paquetes
	\usepackage{amssymb}
	\usepackage{amsmath}
	\usepackage{graphicx}
	\usepackage{libertine}
	\usepackage[bold-style=TeX]{unicode-math}
	\usepackage{lipsum}

	\usepackage{natbib}
	\setcitestyle{square}

	\usepackage{polyglossia}
	\setdefaultlanguage{spanish}


	\usepackage{CBFT.estilo} % Cargo la hoja de estilo

	% Tipografías
	% \setromanfont[Mapping=tex-text]{Linux Libertine O}
	% \setsansfont[Mapping=tex-text]{DejaVu Sans}
	% \setmonofont[Mapping=tex-text]{DejaVu Sans Mono}

	%===================================================================
	%	DOCUMENTO PROPIAMENTE DICHO
	%===================================================================

\begin{document}

\chapter{Transformaciones canónicas}

% =================================================================================================
\section{Funciones generatrices}
% =================================================================================================

Consideraremos ahora varios casos diferentes de dependencia en la función generatriz,
\[
	F_1 = F_1(q_i,Q_i,t)
\]
\[
	\sum p_i \dot{q}_i - H + K - \sum P_i \dot{Q}_i - \dpar{F_1}{q_i} \dot{q}_i - \dpar{F_1}{Q_i} \dot{Q}_i -
	\dpar{F_1}{t} = 0 
\]
\[
	\sum \left( p_i - \dpar{F_1}{q_i} \right) \dot{q}_i  - \sum \left( P_i + \dpar{F_1}{Q_i} \right) \dot{Q}_i -
	\dpar{F_1}{t} - H + K = 0 
\]
y la transformación canónica queda definida por 
\[
	\dpar{F_1}{q_i} = p_i \qquad \qquad \dpar{F_1}{Q_i} = - P_i \qquad \qquad \dpar{F_1}{t} = K-H
\]

Todas las combinaciones posibles son 
\[
	F_1 = F_1(q_i,Q_i,t) \qquad 
	F_2 = F_2(q_i,P_i,t) \qquad 
	F_3 = F_3(p_i,Q_i,t) \qquad 
	F_4 = F_4(p_i,P_i,t)
\]
y para $F_2$, por ejemplo, se tiene 
\[
	F_2(q_i,P_i,t) = \sum_i^N q_i P_i
\]
la cual es una identidad (transformación). Y
\[
	\dpar{F_2}{q_i} = P_i = p_i \qquad \dpar{F_2}{Q_i} = q_i = Q_i
\]

% =================================================================================================
\section{Corchetes de Poisson}
% =================================================================================================

Sea $A=A(q_i,p_i,t)$ entonces
\[
	\frac{d}{dt}A = \sum_i \dpar{A}{q_i} \dpar{q_i}{t} + \dpar{A}{p_i} \dpar{p_i}{t} + \dpar{A}{t}
\]
\[
	\frac{d}{dt}A = \underbrace{\sum_i \dpar{A}{q_i} \dpar{q_i}{t} + \dpar{A}{p_i} \dpar{p_i}{t}}_{\equiv [A,H]} + 
\dpar{A}{t}
\]
entonces
\[
	\frac{d}{dt}A = [A,H] + \dpar{A}{t}.
\]

Las constantes de movimiento en un sistema cumplen que su corchete de Poisson con el hamiltoniano es nulo.
\[
	\dpar{H}{p_i} = \dot{q}_i = [q_i,H] \qquad  -\dpar{H}{q_i} = \dot{p}_i = [p_i,H]
\]
Una transformación canónica cumple 
\[
	[p_i,q_j] = \delta_{ij} \qquad [p_i,p_j] = 0 \qquad [q_i,q_j] = 0
\]
de modo que el corchete entre los momentos es nulo así también como el corchete entre las coordenadas.









% \bibliographystyle{CBFT-apa-good}	% (uses file "apa-good.bst")
% \bibliography{CBFT.Referencias} % La base de datos bibliográfica

\end{document}
