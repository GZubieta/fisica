	\documentclass[10pt,oneside]{CBFT_article}
	% Algunos paquetes
	\usepackage{amssymb}
	\usepackage{amsmath}
	\usepackage{graphicx}
	\usepackage{libertine}
	\usepackage[bold-style=TeX]{unicode-math}
	\usepackage{lipsum}

	\usepackage{natbib}
	\setcitestyle{square}

	\usepackage{polyglossia}
	\setdefaultlanguage{spanish}


	\usepackage{CBFT.estilo} % Cargo la hoja de estilo

	% Tipografías
	% \setromanfont[Mapping=tex-text]{Linux Libertine O}
	% \setsansfont[Mapping=tex-text]{DejaVu Sans}
	% \setmonofont[Mapping=tex-text]{DejaVu Sans Mono}

	%===================================================================
	%	DOCUMENTO PROPIAMENTE DICHO
	%===================================================================

\title{CBFT}
\author{Mecánica clásica}
\date{\today}

\begin{document}
\maketitle
\tableofcontents



\section{Invariancia del lagrangiano ante adición de una derivada total}

Sea una función de las coordenadas y del tiempo $F=F(q_i,t)$ que sumamos al lagrangiano $\Lag$, de modo que
\[\Lag'=\Lag + \dtot{F}{t} \]
y las ecuaciones de Euler-Lagrange para este nuevo lagrangiano son
\[ \frac{d}{dt}\left(\dpar{\Lag'}{\dot{q}_j}\right) - \dpar{\Lag'}{q_j} = 0 \]
\[ \dtot{}{t}\left(\dpar{\Lag}{\dot{q}_j} + \dpar{}{\dot{q}_j}\left(\dtot{F}{t}\right)\right) - \dpar{\Lag}{q_j} - 
\dpar{}{q_j}\left( \dtot{F}{t}\right) = 0 \]
\[ \dtot{}{t}\left(\dpar{\Lag}{\dot{q}_j}\right) - \dpar{\Lag}{q_j} + 
\dtot{}{t}\left(\dpar{}{\dot{q}_j}\left(\dtot{F}{t}\right)\right) - \dpar{}{q_j}\left( \dtot{F}{t}\right) = 0 \]

Ahora es necesario escribir la derivada total de $F$,
\[  \dtot{F}{t} = \sum_j \dpar{F}{q_j}\dtot{q_j}{t} + \dpar{F}{t} = \sum_j \dpar{F}{q_j}\dot{q}_j + \dpar{F}{t} \]
y ver que
\[  \dpar{}{\dot{q}_j}\left(\dtot{F}{t}\right) = \dpar{F}{q_j} \qquad\qquad
\dpar{}{q_j}\left(\dtot{F}{t}\right) = \dpar[2]{F}{q_j} \dot{q}_j + \dparcru{F}{t}{q_j} \]

Luego, usando esta información, resulta que los términos que surgen de la adición de la derivada total de $F$ resultan 
ser
\[  \dtot{}{t}\left( \dpar{}{\dot{q}_j}\left(\dtot{F}{t}\right)\right) - \dpar{}{q_j}\left( \dtot{F}{t}\right) = 
\dtot{}{t}\left( \dpar{F}{q_j} \right) - \dpar{}{q_j}\left( \dtot{F}{t}\right)\]
\[ \dtot{}{t}\left( \dpar{F}{q_j} \right) - \dpar{}{q_j}\left( \dtot{F}{t}\right) =
\dpar[2]{F}{q_j} \dot{q}_j + \dparcru{F}{q_j}{t} - \dpar{}{q_j}\left( \dtot{F}{t}\right) \]
y si aceptamos que $F$ es de clase C$^2$ se tiene
\[\dpar[2]{F}{q_j} \dot{q}_j + \dparcru{F}{q_j}{t} - \dpar{}{q_j}\left( \dtot{F}{t}\right)=0\]
de modo que las ecuaciones de Euler Lagrange no se modifican si añadimos una derivada total respecto del tiempo de una 
función de $q_j,t$.




\bibliographystyle{CBFT-apa-good}	% (uses file "apa-good.bst")
\bibliography{CBFT.Referencias} % La base de datos bibliográfica

\end{document}
