	\documentclass[10pt,oneside]{CBFT_book}
	% Algunos paquetes
	\usepackage{amssymb}
	\usepackage{amsmath}
	\usepackage{graphicx}
	\usepackage{libertine}
	\usepackage[bold-style=TeX]{unicode-math}
	\usepackage{lipsum}

	\usepackage{natbib}
	\setcitestyle{square}

	\usepackage{polyglossia}
	\setdefaultlanguage{spanish}
	



	\usepackage{CBFT.estilo} % Cargo la hoja de estilo

	% Tipografías
	% \setromanfont[Mapping=tex-text]{Linux Libertine O}
	% \setsansfont[Mapping=tex-text]{DejaVu Sans}
	% \setmonofont[Mapping=tex-text]{DejaVu Sans Mono}

	%===================================================================
	%	DOCUMENTO PROPIAMENTE DICHO
	%===================================================================

\begin{document}

% =================================================================================================
\chapter{Teorema de Wigner-Eckart}
% =================================================================================================

Es importante para el cálculo de transiciones evaluar elementos de matriz de operadores tensoriales.
Los elementos matriciales de operadores tensoriales respecto de autoestados de momento satisfacen 
\[
	\Braket{||} =
\]
un coeficiente que no depende de $q,m,m'$.

La regla de selección se construye 
\[
	\Braket{[]} =
\]
\[
	\neq
\]

Una idea de la demostración del teorema
\[
	sa
\]
\[
	ba
\]

Es la misma relación de recurrencia que la de los coeficientes de Clebsh-Gordan, si reemplazamos
\[
	m'=m \quad j=j_1 \quad m=m_1 \qquad j'=j \quad k=j_2 \quad q=m_2
\]

Como ambas relaciones son lineales, sus resultados serán proporcionales.
Se puede asociar 
\[
	\propto 
\]
\[
	\propto
\]
Logramos la igualdad metiendo una constante independiente de $m',q,m$.

\subsection{Reglas de selección}

Como se tiene a $\Braket{|T_q^{(k)}|}$ proporcional a los coeficientes de Clebsh-Gordan, serán válidas las 
mismas reglas de selección 
\[
	m' = m + q \qquad |j-k| \leq j' \leq j+k
\]

\section{Ejemplos de elementos matriciales de tensores}

Sea un escalar (tensor de rango cero)
\[
	\Braket{||} =
\]

No varían $j,m$ en los estados No conecta estados con $j,m$ diferentes un escalar.

Sea un vector (tensor de rango uno):
\[
	\Braket{||} \propto \Braket{j 1;m q|j 1;j' m'}
\]

Conecta estados que están separados por un $j$ y un $m$.

\subsection{Teorema de proyección}

Consideremos lo que sucede en el teorema de Wigner-Eckart si $j=j'$ y se lo aplicamos a un operador vectorial 
$T_q^{(k=1)} \equiv V_q$
\[
	=
\]

Como caso especial, si $\alpha' =\alpha$ estoy en un subespcaio donde coinciden los números cuánticos, se 
tiene
\[
	\vb{V} = \frac{\Braket{\cdot}}{} \vb{J}
\]

\subsection{Aplicación del teorema de proyección}

Sea un $H_0$ esféricamente simétrico 
\[
	cosas
\]

Si meto un campo $B$ en $\hat{z}$ tendré 
\[
	H \equiv H_0 + H_1 = H_0
\]
lo cual debería romper la degeneración.
\[
	L_z + 2S_z =
\]
pero no puedo poner este nuevo operador, que mete el campo B, en el CCOC directamente, entonces uso teorema 
de proyección.
\[
	escalares
\]
Entonces tengo todo expresado en función de $J_z$ que sí forma parte de mi CCOC.


% \bibliographystyle{CBFT-apa-good}	% (uses file "apa-good.bst")
% \bibliography{CBFT.Referencias} % La base de datos bibliográfica

\end{document}
