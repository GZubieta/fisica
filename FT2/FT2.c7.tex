	\documentclass[10pt,oneside]{CBFT_book}
	% Algunos paquetes
	\usepackage{amssymb}
	\usepackage{amsmath}
	\usepackage{graphicx}
	\usepackage{libertine}
	\usepackage[bold-style=TeX]{unicode-math}
	\usepackage{lipsum}

	\usepackage{natbib}
	\setcitestyle{square}

	\usepackage{polyglossia}
	\setdefaultlanguage{spanish}
	



	\usepackage{CBFT.estilo} % Cargo la hoja de estilo

	% Tipografías
	% \setromanfont[Mapping=tex-text]{Linux Libertine O}
	% \setsansfont[Mapping=tex-text]{DejaVu Sans}
	% \setmonofont[Mapping=tex-text]{DejaVu Sans Mono}

	%===================================================================
	%	DOCUMENTO PROPIAMENTE DICHO
	%===================================================================

\begin{document}

% =================================================================================================
\chapter{Armónicos esféricos como matrices de rotación}
% =================================================================================================
Se pueden hallar autoestados de dirección $\Ket{\hat{n}}$ rotando el $\Ket{\hat{z}}$,
\[
	\hat{n} = 
\]
Necesitamos aplicar 
\[
	D
\]
\[
	n
\]
\[
	l
\]
\[
	l
\]
\[
	Y
\]
pero como $\theta=0$ , $Y_\ell^m = 0$  con $m\neq 0$ se tiene 
\[
	lm
\]
\[
	Y^*,
\]
la matriz de rotación en este caso es un armónico esférico.

La $\Psi$ tiene la misma simetría que el potencial.

% =================================================================================================
\section{Suma de momentos angulares}
% =================================================================================================

\subsection{Dos momentos de spín $1/2$}

Sean dos estados de spín $1/2$
\[
	a
\]
en cada espacio valen las relaciones usuales de conmutación 
\[
	b, [S_{1i},S_{2j}] = 0
\]
donde el último indica que operadores de espacios diferentes conmutan.

Un estado general es 
\[
	a
\]
Hay cuatro estados
\[
	b
\]
que corresponden a los operadores $S_ 1^2, S_2^2, S_{1z}, S_{2z}$ que conmutan (son un CCOC).

Podemos elegir otras base de operadores que comutan que será: $S_ 1^2, S_2^2, S, S_{z}$, de modo que el estado 
general 
será
\[
	c
\]
Así tendremos
\[
	d
\]
\[
	S 2 =  \qquad 
\]

Dada la repetición de $S_1,D_2$ se suelen identificar a las bases solamente 
\[
	d
\]
Además la base $\{ \Braket{m_1,m_2}\}$ se puede poner como 
\[
	+ \equiv + 1/2 \qquad\qquad - \equiv - 1/2
\]

\subsection{Cambio entre bases}

Podemos hallar a ojo que 
\[
	\cdot \Ket{++} = \Ket{1,1} \qquad \cdot \Ket{--} = \Ket{1,-1} 
\]
de manera que la única forma de tener $m=1$ es con los dos spines up y la única forma de tener $m=-1$ es con 
los dos 
spines down.

Se hallan los otros con el operador de bajada
\[
	S_- \equiv S_{1-} + S_{2-}
\]
y si descompongo $S_-$ en $S_{1-}$ y $S_{2-}$ para operar en $\Braket{s,m}$ se tiene 
\[
	S
\]
y ahora si opero con $S_-$,
\[
	S
\]

Luego
\[
	\Braket{00}
\]
y puedo usar ortonormalidad 
\[
	= 0 = \qquad \text{con} \; |a|^2 + |b|^2 = 1
\]
\[
	\cdot 
\]


\section{Teoría formal de suma de momentos angulares}

Sea de sumar dos momentos angulares $J_1, J_2$. Las relaciones de conmutación son
\[
	[] = i\hbar \varepsilon_{ijk}J_{1k} \qquad [] = i\hbar \varepsilon_{ijk}J_{2k} \qquad
	[J_{1k},J_{2k}] = 0
\]
\[
	\vb{J} = \otimes + \otimes \equiv \vb{J}_1
\]
\[
	[]
\]

El momento total \vb{J} cumple que 
\[
	J^2 = 
\]
donde vemos que 
\[
	[] =
\]
pero 
\[
	[ J^2 , J_{1z}] \neq 0  \qquad \qquad [ J^2 , J_{2z}] \neq 0
\]

Esto deja dos opciones para elegir un CCOC

\begin{center}
\begin{tabular}{|c|c|}
\hline 
$J_1^2, J_2^2, J_{1z}, J_{2z}$ & $J_1^2, J_2^2, J^2, J_{z}$ \\
$\Ket{j_1,j_2;m_1,m_2}$ & $\Ket{j_1,j_2;j,m}$ \\
base desacoplada & base acoplada \\
\hline
\end{tabular}
\end{center}



Se puede pasar de uan base a otra con una identidad $\mathbb{1}$ apropiada
\[
	\Ket{j_1,j_2;j,m} =
\]
\[
	1.
\]
\[
	a
\]
\[
	2.
\]
\[
	a
\]

Donde los $C_{m_1 m_2}^j$ son los coeficientes de Clebsh-Gordan. En 2 la $\sum$ sería en $j\to\infty$, pero 
veamos la 
relacion que hace algunos $C_{m_1 m_2}^j=0$. Ante todo abreviaremos suprimiendo los índices $j_1,j_2$ con lo 
cual 
\[
	C
\]

\subsection{Restricciones para la no nulidad de los coeficientes}

\[
	a
\]
\[
	b
\]
\[
	\Braket{m_1,m_2| j,m} \neq 0 \Rightarrow m = m_1 + m_2
\]

A su vez, en la suma de $J_1$ y $J_2$ resultan los $j$ acotados por una desigualdad triangular 
\[
	|| \leq j \leq j_1 + j_2
\]

Asimismo los $C_{m_1 m_2}^j$ se toman reales, entonces 
\[
	C
\]
y juntando todo se tiene 
\[
	\Rightarrow
\]

Ambas bases tienen la misma dimensión
\[
	\sum = (2j_1 + 1)(2j_2 + 1)
\]

Recordemos que cada $j$ tiene $2j+1$ estados posibles (los $m$ correspondientes a cada $j$) ($|m|\leq j$). Si 
sumamos 
$j_1=1, j_2=3/2$ tendremos 
\[
	dim
\]
\[
	j = 1/2, 3/2, 5/2
\]

Podemos ver a ojo que 
\[
	\Ket{j}
\]
luego con el $J_=, J_-$ podemos construirnos los siguientes (utilizando ortonormalidad)
\[
	= 
\]
\[
	= \qquad \sum = 1
\]

\subsection{Relación de recurrencia}

\[
	J_\pm = \ ket{j,m} = 
\]
\[
	b
\]
y metiendo un bra $\Bra{m_1,m_2}$ se llega a la relación de recurrencia
\[
	\sqrt{()}
\]

\subsection{Suma de \vb{L} y \vb{S}}

Sea suma \vb{L} y \vb{S}, entonces 
\[
	a
\]
habrá sólo cuatro $C_{m_1 m_2}^j$ no nulos, que serán 
\[
	\Braket{|}
\]
donde vemos que los coeficientes linkean sólo los estados con $j=\ell-1/2$ y $j=\ell+1/2$ y podemos construir 
una 
matriz de $2\times 2$ para este caso.

Esto tórnase práctico para acoplamiento spin-órbita 
\[
	LS
\]
\[
	LS
\]
\[
	LS
\]

\section{Operadores vectoriales}

Queremos analizar como transforma un operador vectorial $\hat{v}$ bajo rotaciones en mecánica cuántica.
En mecánica clásica,
\[
	V_i = R_{ij} V_j \qquad \text{con} \; R \; \text{matriz diagonal}
\]
En mecánica cuántica tenemos que al rotar
\[
	=
\]
Pediremos entonces que $\Braket{V}$ transforme como un vector y eso lleva a que 
\[
	=
\]
\[
	\mathcal{D}(R)^+
\]
y calculando la expresión anterior 1 llegamos a que debe valer
\[
	[V_i,J_j] =  i\hbar \varepsilon_{ijR}V_R
\]
que es la manera de transformar de un operador vectorial. Podemos probar un caso simple de una rotación 
infinitesimal 
en $\hat{z}$ y ver que vale.


\section{Operadores tensoriales}

En mecánica clásica 
\[
	T_{ij} 
\]
que es un tensor de rango dos. Esto es un tensor cartesiano. Su problema es que no es irreducible, entonces 
puede 
descomponerse en objetos que transforman diferente ante rotaciones. Sea la díada $U_iV_j$, tensor de rango 
dos, que 
puede escribirse como 
\[
	UV =
\]
Hemos reducido el tensor cartesiano en tensores irreducibles. Podemos asociar esta descomposición con las 
multiplicidades de objetos con momento angular $\ell=0, \ell=1, \ell=2$
\[
	\text{escalar} \longrightarrow \ell=0 \; \text{singlete (un elemento independiente) }
\]
\[
	\text{vector} \longrightarrow \ell=1 \; \text{triplete (tres elementos independientes)}
\]
\[
	\text{tensor de traza nula} \longrightarrow \ell=2 \; \text{quintuplete (cinco elementos 
independientes)}
\]

Se define 
\[
	T^{(k)}_q \qquad \text{tensor esférico de rango $k$ y número magnético $q$}
\]
Un tensor esférico transforma como 
\[
	D T D = D T 
\]
Tendremos 
\[
	T^{(0)}_0 \quad \text{(escalar) tensor esférico de rango 0 ($\ell=0$)}
\]
\[
	(T^{(1)}_1,T^{(1)}_0,T^{(1)}_{-1}) \quad \text{(vector) tensor esférico de rango 1 ($\ell=1$)}
\]

En muchos casos se puede escribir un tensor esférico como armónico esférico 
\[
	Y_\ell^{m}(\theta,\varphi) = Y_\ell^{m}(\hat{n})  \overbrace{\hat{n} \longrightarrow 
	\vec{v}}^{\text{paso}} Y_\ell^m(\vec{v}) \equiv Y_k^q(\vec{v}) = T_q^{(k)}
\]
\[
	\hat{n} = ()
\]
\[
	Y \longrightarrow T
\]
\[
	Y
\]
Calculando en 2, cosa que podemos hacer para, por ejemplo, una rotación infinitesimal, llegamos a las 
relaciones de conmutación para tensores.
\[
	[]
\]



% \bibliographystyle{CBFT-apa-good}	% (uses file "apa-good.bst")
% \bibliography{CBFT.Referencias} % La base de datos bibliográfica

\end{document}
