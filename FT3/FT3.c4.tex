	\documentclass[10pt,oneside]{CBFT_book}
	% Algunos paquetes
	\usepackage{amssymb}
	\usepackage{amsmath}
	\usepackage{graphicx}
	\usepackage{libertine}
	\usepackage[bold-style=TeX]{unicode-math}
	\usepackage{lipsum}

	\usepackage{natbib}
	\setcitestyle{square}

	\usepackage{polyglossia}
	\setdefaultlanguage{spanish}
	



	\usepackage{CBFT.estilo} % Cargo la hoja de estilo

	% Tipografías
	% \setromanfont[Mapping=tex-text]{Linux Libertine O}
	% \setsansfont[Mapping=tex-text]{DejaVu Sans}
	% \setmonofont[Mapping=tex-text]{DejaVu Sans Mono}

	%===================================================================
	%	DOCUMENTO PROPIAMENTE DICHO
	%===================================================================

\begin{document}

% =================================================================================================
\chapter{Gases imperfectos}
% =================================================================================================


% =================================================================================================
\section{Cuánticos --reubicar}
% =================================================================================================

Ensamble de $ \mathcal{N} $ sistemas $(k=1,2,...,\mathcal{N})$. Cada uno tiene su estado descripto por 
\[
	\Psi^k(\vb{x},t), \qquad \qquad \hat{H} \Psi^k = i\hbar \dpar{\Psi^k}{t} \quad \forall k
\]

Si son estados puros entonces 
\notamargen{Todos son la {\it misma} combinación lineal de la base.}
\[
	\Psi^k = \sum_n a_n(t) \phi_n(\vb{x}) \qquad \{ \phi_n \} \text{ set ortonormal }
\]
Un estado puro es superposición coherente de una base 
\[
	i \hbar \dpar{}{t} a_m^k = \sum_n H_{mn}a_n^k
\]

El sistema k-ésimo puede describirse a partir de $ \Psi^k $ o bien a partir de los coeficientes $ \{ a_n \}$.

Definimos un operador de densidad,
\notamargen{Promedio en el ensamble de la interferencia cuántica entre $\phi_m$ y $\phi_n$. $p_k$ es la probabilidad
del estado $k$.}
\[
	\rho_{mn} \equiv \sum_{k=1}^\mathcal{N} p_k a_m^k (a_n^k)^*
\]
el cual proviene de 
\[
	\hat{\rho}_{mn} = \sum_{k=1}^\mathcal{N} p_k \Ket{\Psi^k}\Bra{\Psi^k}
\]
\notamargen{¿Y los índices $mn$ capo?}

Puede verse que se cumple
\[
	i \hbar \dot{\rho} = [ \hat{H}, \hat{\rho} ],  
\]
un teorema de Liouville cuántico.
 
Sea el valor medio de $ \hat{G} $
\[
	\braket{G}_{ENS} = \sum_{k=1}^\mathcal{N} p_k \braket{G}_k = 
	\sum_{k=1}^\mathcal{N} p_k \braket{\Psi^k|\hat{G}|\Psi^k}_k = 
	\sum_k p_k \int \sum_i a_i^{k*}\phi_i^* \hat{G}\sum_j a_j^k\phi_j dx
\]
\[
	\braket{G}_{ENS} = \sum_k p_k \sum_i \sum_j a_i^{k*}  a_j^k \int \phi_i^* G \phi_j dx =
	\sum_i \sum_j \left( \sum_k p_k a_i^{k*}  a_j^k \right) G_{ij}
\]
\[
	\braket{G}_{ENS} = \sum_i \sum_j \rho_{ij} G_{ij} = 
	\text{ Traza }(\hat{\rho}\hat{G}) = \sum_i [\rho G]_{ii}
\]

Ahora, si el conjunto $\{ \phi_n \}$ fuesen autoestados de $\hat{G}$ entonces 
\[
	\int dx \phi_i^* G \phi_j = \int dx \phi_i^* \phi_j g_j = \delta_{ij} g_j = g_i
\]
\[
	\braket{G}_{ENS} = \sum_k p_k \sum_i a_i^{k*}  a_i^k g_i = 
	\sum_k p_k \sum_i |a_i^k|^2 g_i
\]

La matriz densidad $\hat{\rho}$ se define de modo que sus elementos $\rho_{ij}$ resultan 
\[
	\braket{\phi_i|\hat{\rho}|\phi_j} = \sum_{k=1}^\mathcal{N} p_k \braket{\phi_i|\Psi^k} \braket{\Psi^k|\phi_j} =
	\sum_{k=1}^\mathcal{N} p_k \int dx \phi^*_i \sum_l a_l^k \phi_l \int dx' \phi_j \sum_m a_m^{k*} \phi_m^*
\]
\[
	\braket{\phi_i|\hat{\rho}|\phi_j} = 
	\sum_{k=1}^\mathcal{N} p_k \sum_l \sum_m a_l^k a_m^{k*} \int dx \phi^*_i \phi_l \int dx' \phi_j \phi_m^* =
	\sum_{k=1}^\mathcal{N} p_k \sum_l \sum_m a_l^k a_m^{k*} \delta_{il}\delta_{jm}
\]
\[
	\rho_{ij} = \sum_k p_k a_i^k a_j^{k*}
\]

El primer postulado de la QSM es asegurarse de que $\rho_{ij} \propto \delta_{ij} $, es decir que
EN PROMEDIO no hay correlación entre funciones $\{ \phi_i \}$ para diferentes miembros $k$ del ensamble.
El elemento $\rho_{ij}$ es el promedio en el ensamble de la interferencia entre $\phi_i$ y $\phi_j$.


En la práctica los ensambles serán mezcla, una superposición de estados puros pero incoherente, de modo
que 
\notamargen{Es muy difícil preparar un ensamble puro.}
\[
	\hat{\rho} = \sum_{k=1}^\mathcal{N} p_k \Ket{\Psi^k}\Bra{\Psi^k} \qquad p_k \geq 0 \quad \sum_k p_k = 1 
\]
donde $p_k$ serán las {\it abundancias relativas} de los estados puros $\Psi^k$.

Para un ensamble puro sería
\[
	\hat{\rho} = \Ket{\Psi}\Bra{\Psi}
\]
donde no hay supraíndice $k$ puesto que todos son el mismo estado.

Un estado puro puede escribirse 
\[
	\Psi^k = \sum_n a_n \phi_n, \quad \text{ o bien }\quad \Ket{\Psi^k} = \sum_n a_n \Ket{\phi_n}
\]
y sabemos que el valor de expectación será
\[
	\braket{A}_k = \braket{\Psi^k|\hat{A}|\Psi^k} = \int dx \Psi^{k*} A \Psi^k
\]

Un estado mezcla será en cambio 
\be
	\Ket{\xi} \cong \sum_n p_n\Ket{\phi_n}
	\label{estado_mezcla}	
\ee
donde $\sum_n p_n =1$ y $p_n \in \mathbb{R}>0$. Pero $\Ket{\xi} $ no es un estado de sistema como $\Psi^k$ pués
\be
	\Ket{\xi} \neq \sum_n c_n\Ket{\phi_n}
	\label{falso_mezcla}
\ee
no hay cambio de base que lleve \eqref{estado_mezcla} al miembro derecho de \eqref{falso_mezcla}.
Entonces
\[
	\braket{A}_\xi \neq \braket{\xi|\hat{A}|\xi}
\]

Pero como en la práctica lo que se tiene son estados mezcla, la matriz de densidad $\hat{\rho}$ permite trabajar
con ellos tranquilamente.

Sea que evaluamos el valor medio de $ \hat{G} = \hat{\Ham} $ que será la energía $\braket{E}$ en autoestados de 
$ \hat{\Ham} $.
\[
	\braket{\hat{\Ham}}_{ENS} = \braket{E} = \sum_k p_k \sum_i \sum_j a_i^{k*} a_j^k \int \phi_i^* \phi_j E_j =
	\sum_k p_k \sum_j a_j^{k*} a_j^k E_j
\]
\[
	\braket{E} = \sum_k p_k \sum_j a_j^{k*} a_j^k E_j = \sum_j \left( \sum_k p_k a_j^{k*} a_j^k \right) E_j =
	\sum_j \rho_{jj} E_j
\]

Se tiene que $ \hat{\rho} $ es diagonal para un operador $\hat{G}$ tal que utilizamos la base de autoestados.

Querremos que esto valga para cualquier base entonces necesitaremos que las fases sean números aleatorios:
\[
	\rho_{ij} = \sum_k^\mathcal{N} p_k a_i^{k*} a_j^k = 
	\sum_k^\mathcal{N} p_k | a_i^k | | a_j^k |\euler^{i( \theta_i^k - \theta_j^k )}
\]
y asi además son equiprobables (microcanónico) los estados base accesibles,
\[
	p_k = \frac{1}{\mathcal{N}} \qquad \text{ y } \qquad |a^k_i| = |a_i| \quad \forall k
\]
y asimismo pedimos que para cada miembro del ensamble la amplitud sea la misma, se tiene 
\[
	\rho_{ij} = | a_i | | a_j | \frac{1}{\mathcal{N}} \sum_k^\mathcal{N} \euler^{i( \theta_i^k - \theta_j^k )}
	= | a_i | | a_j | \delta_{ij}
\]
donde se han usado fases al azar, de modo que 
\[
	\rho_{ij} = | a_i |^2 \delta_{ij} = \rho_i \delta_{ij}
\]
\notamargen{Esto no está consistente: colapsas la delta o no, papi?}
y entonces 
\[
	\begin{cases}
	 \rho_i = \displaystyle{ \frac{1}{\Gamma} }\\
	 \rho_i = 0
	\end{cases}
\]

Entonces $ \rho_i $ será la probabilidad del estado de base $ \phi_i $. Se sigue que el operador densidad del
microcanónico puede escribirse 
\[
	\hat{\rho} = \sum_i | a_i |^2 \Ket{\phi_i}\Bra{\phi_i}
\]
de manera que es una superposición incoherente de estados de la base $\{ \phi_i \}$
\[
	\hat{\rho} = \sum_i \rho_i \Ket{\phi_i}\Bra{\phi_i}
\]
y al final del día
\[
	\rho_{kl} = \braket{ \phi_k | \hat{\rho} | \phi_l } = \sum_i \rho_i 
	\braket{ \phi_k | \phi_i }  \braket{ \phi_i | \phi_l } = \sum_i \rho_i \delta_{ki} \delta_{il} = 
	\rho_k \delta_{kl}
\]

\[
	\Omega = 1 \text{ ensamble puro } \qquad \qquad S = k\log \Omega = 0
\]
\[
	\rho_{mn} = \frac{1}{\mathcal{N}} \sum_k^\mathcal{N} a_m^{k*} a_m^k = a_m a_n^* 
\]
si es la misma $\Psi \forall k$ el sistema se halla en una combinación lineal de $\phi_n$, o bien
\[
	\rho_{mn} = |a_m|^2 \delta_{mn}
\]
el sistema se halla en un único autoestado $ \phi_n $

\[
	\Omega > 1 \text{ ensamble mezcla }
\]

\subsection{Resumen formalismo}

\[
	\rho_{ij} = \rho_i \delta_{ij}
\]
\[
	\rho_i = \frac{1}{\Omega} \qquad \text{ Microcanónico }
\]
\[
	\rho_i = \frac{\euler^{-\beta E_i}}{Q_N(V,T)}  \qquad  \text{ Canónico }
\]
\[
	\rho_i = \frac{\euler^{-\beta E_i + \beta \mu N_i }}{\Xi(z,V,T)}  \qquad  \text{ Gran canónico }
\]

\[
	\hat{\rho} = \sum_i \Ket{ \phi_i } \rho_i \Bra{ \phi_i } \qquad \qquad \text{ Traza }(\hat{\rho} ) =
	1 \text{ bien normalizado }
\]
\[
	\hat{\rho} = \frac{1}{\Omega} \sum_i^{\text{ACC}} \Ket{ \phi_i } \Bra{ \phi_i } = 
	\frac{1}{\Omega} \hat{\mathbb{1}}^{\text{ACC}}  \qquad \text{ Tr }(\hat{\rho} ) = 1 
\]
donde $ \hat{\mathbb{1}}^{\text{ACC}}  $ es una indentidad con 0 para los sitios de la diagonal donde no hay
estado accesible. Luego $ \text{ Traza }(\hat{\mathbb{1}}^{\text{ACC}})  = \Omega $. Para los otros dos casos,
\[
	\hat{\rho} = \frac{\euler^{-\beta E_i}}{Q_N(V,T)}  \sum_i^{\text{ACC}} \Ket{ \phi_i } \Bra{ \phi_i } = 
	\frac{\euler^{-\beta E_i}}{Q_N(V,T)} \hat{\mathbb{1}}^{\text{ACC}} 
	\qquad \text{ Tr }(\hat{\rho} ) = \frac{1}{Q_N} \text{ Tr }( \euler^{-\beta E_i} \hat{\mathbb{1}}^{\text{ACC}} )
\]
\[
	\hat{\rho} = \frac{\euler^{-\beta E_i + \beta \mu N_i }}{\Xi(z,V,T)} 
	\sum_i^{\text{ACC}} \Ket{ \phi_i } \Bra{ \phi_i } = 
	\frac{\euler^{-\beta E_i + \beta \mu N_i }}{\Xi(z,V,T)} \hat{\mathbb{1}}^{\text{ACC}} 
	\qquad \text{ Tr }(\hat{\rho} ) = 
	\frac{1}{\Xi} \text{ Tr }( \euler^{-\beta E_i + \beta \mu N_i } \hat{\mathbb{1}}^{\text{ACC}} ) 
\]

El conteo de estados se hace cuánticamente de modo que no hay paradoja de Gibbs. Los estados accesibles en el
microcanónico $ (\Omega) $ son tales que sus probabilidad es 
\[
	| a_i |^2 = \frac{1}{\Omega} \quad \forall i \text{ accesible }
\]
Serán aquellos de la base $ \{ \phi_i \} $ en cuestión tales que la energía resulte vale entre $E$ y $E+\Delta E$.

Los dos postulados
\begin{itemize}
 \item i) Equiprobabilidad
 \item ii) Fases al azar
\end{itemize}
aseguran que no hay correlación entre las funciones $ \{ \phi_i \} $ (en promedio).




% \bibliographystyle{CBFT-apa-good}	% (uses file "apa-good.bst")
% \bibliography{CBFT.Referencias} % La base de datos bibliográfica

\end{document} 
