	\documentclass[10pt,oneside]{CBFT_book}
	% Algunos paquetes
	\usepackage{amssymb}
	\usepackage{amsmath}
	\usepackage{graphicx}
	\usepackage{libertine}
	\usepackage[bold-style=TeX]{unicode-math}
	\usepackage{lipsum}

	\usepackage{natbib}
	\setcitestyle{square}

	\usepackage{polyglossia}
	\setdefaultlanguage{spanish}
	



	\usepackage{CBFT.estilo} % Cargo la hoja de estilo

	% Tipografías
	% \setromanfont[Mapping=tex-text]{Linux Libertine O}
	% \setsansfont[Mapping=tex-text]{DejaVu Sans}
	% \setmonofont[Mapping=tex-text]{DejaVu Sans Mono}

	%===================================================================
	%	DOCUMENTO PROPIAMENTE DICHO
	%===================================================================

\begin{document}

% =================================================================================================
\chapter{Gases imperfectos}
% =================================================================================================


% =================================================================================================
\section{Cuánticos --reubicar}
% =================================================================================================

Ensamble de $ \mathcal{N} $ sistemas $(k=1,2,...,\mathcal{N})$. Cada uno tiene su estado descripto por 
\[
	\Psi^k(\vb{x},t), \qquad \qquad \hat{H} \Psi^k = i\hbar \dpar{\Psi^k}{t} \quad \forall k
\]

Si son estados puros entonces 
\notamargen{Todos son la {\it misma} combinación lineal de la base.}
\[
	\Psi^k = \sum_n a_n(t) \phi_n(\vb{x}) \qquad \{ \phi_n \} \text{ set ortonormal }
\]
Un estado puro es superposición coherente de una base 
\[
	i \hbar \dpar{}{t} a_m^k = \sum_n H_{mn}a_n^k
\]

El sistema k-ésimo puede describirse a partir de $ \Psi^k $ o bien a partir de los coeficientes $ \{ a_n \}$.

Definimos un operador de densidad,
\notamargen{Promedio en el ensamble de la interferencia cuántica entre $\phi_m$ y $\phi_n$. $p_k$ es la probabilidad
del estado $k$.}
\[
	\rho_{mn} \equiv \sum_{k=1}^\mathcal{N} p_k a_m^k (a_n^k)^*
\]
el cual proviene de 
\[
	\hat{\rho}_{mn} = \sum_{k=1}^\mathcal{N} p_k \Ket{\Psi^k}\Bra{\Psi^k}
\]
\notamargen{¿Y los índices $mn$ capo?}

Puede verse que se cumple
\[
	i \hbar \dot{\rho} = [ \hat{H}, \hat{\rho} ],  
\]
un teorema de Liouville cuántico.
 
Sea el valor medio de $ \hat{G} $
\[
	\braket{G}_{ENS} = \sum_{k=1}^\mathcal{N} p_k \braket{G}_k = 
	\sum_{k=1}^\mathcal{N} p_k \braket{\Psi^k|\hat{G}|\Psi^k}_k = 
	\sum_k p_k \int \sum_i a_i^{k*}\phi_i^* \hat{G}\sum_j a_j^k\phi_j dx
\]
\[
	\braket{G}_{ENS} = \sum_k p_k \sum_i \sum_j a_i^{k*}  a_j^k \int \phi_i^* G \phi_j dx =
	\sum_i \sum_j \left( \sum_k p_k a_i^{k*}  a_j^k \right) G_{ij}
\]
\[
	\braket{G}_{ENS} = \sum_i \sum_j \rho_{ij} G_{ij} = 
	\text{ Traza }(\hat{\rho}\hat{G}) = \sum_i [\rho G]_{ii}
\]

Ahora, si el conjunto $\{ \phi_n \}$ fuesen autoestados de $\hat{G}$ entonces 
\[
	\int dx \phi_i^* G \phi_j = \int dx \phi_i^* \phi_j g_j = \delta_{ij} g_j = g_i
\]
\[
	\braket{G}_{ENS} = \sum_k p_k \sum_i a_i^{k*}  a_i^k g_i = 
	\sum_k p_k \sum_i |a_i^k|^2 g_i
\]

La matriz densidad $\hat{\rho}$ se define de modo que sus elementos $\rho_{ij}$ resultan 
\[
	\braket{\phi_i|\hat{\rho}|\phi_j} = \sum_{k=1}^\mathcal{N} p_k \braket{\phi_i|\Psi^k} \braket{\Psi^k|\phi_j} =
	\sum_{k=1}^\mathcal{N} p_k \int dx \phi^*_i \sum_l a_l^k \phi_l \int dx' \phi_j \sum_m a_m^{k*} \phi_m^*
\]
\[
	\braket{\phi_i|\hat{\rho}|\phi_j} = 
	\sum_{k=1}^\mathcal{N} p_k \sum_l \sum_m a_l^k a_m^{k*} \int dx \phi^*_i \phi_l \int dx' \phi_j \phi_m^* =
	\sum_{k=1}^\mathcal{N} p_k \sum_l \sum_m a_l^k a_m^{k*} \delta_{il}\delta_{jm}
\]
\[
	\rho_{ij} = \sum_k p_k a_i^k a_j^{k*}
\]

El primer postulado de la QSM es asegurarse de que $\rho_{ij} \propto \delta_{ij} $, es decir que
EN PROMEDIO no hay correlación entre funciones $\{ \phi_i \}$ para diferentes miembros $k$ del ensamble.
El elemento $\rho_{ij}$ es el promedio en el ensamble de la interferencia entre $\phi_i$ y $\phi_j$.


En la práctica los ensambles serán mezcla, una superposición de estados puros pero incoherente, de modo
que 
\notamargen{Es muy difícil preparar un ensamble puro.}
\[
	\hat{\rho} = \sum_{k=1}^\mathcal{N} p_k \Ket{\Psi^k}\Bra{\Psi^k} \qquad p_k \geq 0 \quad \sum_k p_k = 1 
\]
donde $p_k$ serán las {\it abundancias relativas} de los estados puros $\Psi^k$.

Para un ensamble puro sería
\[
	\hat{\rho} = \Ket{\Psi}\Bra{\Psi}
\]
donde no hay supraíndice $k$ puesto que todos son el mismo estado.

Un estado puro puede escribirse 
\[
	\Psi^k = \sum_n a_n \phi_n, \quad \text{ o bien }\quad \Ket{\Psi^k} = \sum_n a_n \Ket{\phi_n}
\]
y sabemos que el valor de expectación será
\[
	\braket{A}_k = \braket{\Psi^k|\hat{A}|\Psi^k} = \int dx \Psi^{k*} A \Psi^k
\]

Un estado mezcla será en cambio 
\be
	\Ket{\xi} \cong \sum_n p_n\Ket{\phi_n}
	\label{estado_mezcla}	
\ee
donde $\sum_n p_n =1$ y $p_n \in \mathbb{R}>0$. Pero $\Ket{\xi} $ no es un estado de sistema como $\Psi^k$ pués
\be
	\Ket{\xi} \neq \sum_n c_n\Ket{\phi_n}
	\label{falso_mezcla}
\ee
no hay cambio de base que lleve \eqref{estado_mezcla} al miembro derecho de \eqref{falso_mezcla}.
Entonces
\[
	\braket{A}_\xi \neq \braket{\xi|\hat{A}|\xi}
\]

Pero como en la práctica lo que se tiene son estados mezcla, la matriz de densidad $\hat{\rho}$ permite trabajar
con ellos tranquilamente.









% \bibliographystyle{CBFT-apa-good}	% (uses file "apa-good.bst")
% \bibliography{CBFT.Referencias} % La base de datos bibliográfica

\end{document}
