	\documentclass[10pt,oneside]{CBFT_book}
	% Algunos paquetes
	\usepackage{amssymb}
	\usepackage{amsmath}
	\usepackage{graphicx}
	\usepackage{libertine}
	\usepackage[bold-style=TeX]{unicode-math}
	\usepackage{lipsum}

	\usepackage{natbib}
	\setcitestyle{square}

	\usepackage{polyglossia}
	\setdefaultlanguage{spanish}
	



	\usepackage{CBFT.estilo} % Cargo la hoja de estilo

	% Tipografías
	% \setromanfont[Mapping=tex-text]{Linux Libertine O}
	% \setsansfont[Mapping=tex-text]{DejaVu Sans}
	% \setmonofont[Mapping=tex-text]{DejaVu Sans Mono}

	%===================================================================
	%	DOCUMENTO PROPIAMENTE DICHO
	%===================================================================

\begin{document}

% =================================================================================================
\chapter{Conjuntos estadísticos}
% =================================================================================================

La cantidad
\[
	\rho(\{ \vec{q}_i, \vec{p}_i\},t) d^{3N}qd^{3N}p
\]
es el número de microestados en el elemento $d^{3N}qd^{3N}p$ al tiempo $t$ centrado en $q,p$.
Si los microestados son equiprobables $\rho \equiv cte.$. El conjunto $\{ \vec{q}_i, \vec{p}_i\}$ son
$6N$ coordenadas.
\[
	\Omega = \int p d^{3N}qd^{3N}p
\]
\notamargen{La integral $\Omega$ es imposible porque es difícil determinar el volumen de integración.}

XXX Dibujos XXXX

el volumen en  $\mathbb{\Gamma}$ es proporcional al número de microestados compatibles con $E,N$,
el volumen $ \mathbb{\Gamma}$ del macroestado es $\Omega\{ n_i \}$

$n_i = f_i d^3q d^3p$ es el número de partículas en una celda $i$ (con su $\vec{p}$ en $\vec{p} + d\vec{p}$
y con su $\vec{q}$ en $\vec{q} + d\vec{q}$ )

Un microestados determina una distribución $f$ que da un conjunto $\{ n_i \}$. Pero una $f$ determina muchos
microestados porque la función de distribución no distingue entre partículas (importan los números de 
ocupación); entonces una $f$ determina un volumen en $\mathbb{\Gamma}$.
\notamargen{Cada microestado tiene su $f$.}

Suponemos que todos los microestados en $\mathbb{\Gamma}$ son igualmente probables.
La $f$ que determina el mayor volumen en  $\mathbb{\Gamma}$ es la más probable. Suponemos que en el 
equilibrio el sistema toma la $f$ más probable.
Si $f_i$ es el valor de $f$ en cada celda $i$
\[
	f_i = \frac{n_i}{d^3p d^3q} \quad \text{promediada en el ensamble} \quad \bar{f}_i =  \frac{<n_i>}{d^3p d^3q}
	\quad \text{en el equilibrio}
\]
\notamargen{$f_i$ es la distribución para un miembro en el ensamble.}

Esta $\bar{f}_i$ es la de equilibrio, pero la cuenta no es fácil. Asumiremos que la $f$ de equilibrio es la más
probable (la de mayor volumen en  $\mathbb{\Gamma}$); entonces maximizaremos dicho volumen para hallarla.

Un microestado determina una $f$; diferentes microestados pueden determinar otras $f$ pero muchos coincidirán en
una misma $f$.

La $f$ en el equilibrio es la que tiene mayor cantidad de microestados (la más probable) pero 
\[
	\bar{f}_i =  \frac{<n_i>}{d^3p d^3q}
\]
es el promedio en el ensamble y no será exactamente igual a la $f_i$ del mayor volumen, salvo que el volumen de $f$
sea mucho mayor al ocupado por $f',f''$, etc.

Dado el volumen $\Omega \{ n_i\}$ extremaremos el mismo sujeto a las condiciones
\[
	E = \sum_i^K n_i e_i \qquad \qquad N = \sum_i^K n_i
\]
y llegamos a la $f$ de equilibrio que es $f_{MB}$.
\notamargen{Necesito $\Omega = \Omega \{ n_i\}$ para obtener el $\{ \tilde{n}_i\}$.}

El volumen $\Omega$ se escribe en función de los números de ocupación
\[
	\Omega \left( \{ n_i \} \right) = 
	\frac{N!}{\prod_i^K n_i!} \prod_i^K g_i^{n_i} \qquad 
	(i=1,2,...,K \quad \text{identifica celdas en}\;\mu )
\]
\[
	\Omega \left( \{ n_i \} \right) = N! \prod_i^K \frac{g_i^{n_i}}{n_i!}
\]
donde $g_i$ son los subniveles en que podríamos dividir la celda $K$; es por matemática conveniencia y para abarcar 
más casos (luego será $g_i=1 \forall i$).

El conjunto $\{ \tilde{n}_i\}$ que extrema $\Omega \left( \{ n_i \} \right)$ es el más probable y consideraremos
\[
	\{ \tilde{n}_i\} = < n_i >
\]
Estaremos pensando que cuando $N \to \infty$ la mayor parte de los microestados van a una distribución $f_{MB}$


% =================================================================================================
\section{Microcanónico}
% =================================================================================================

\subsection{Solución de equilibrio}

La solución de equilibrio satisfacía
\[
	f(p_1) f(p_2) = f(p_1') f(p_2')
\]
\[
	\log f(p_1) + \log f(p_2) = \log f(p_1') + \log f(p_2')
\]
que luce como una ley de conservación y admite como solución
\[
	\log f(p) = A m + \vb{B}\cdot \vb{p} + C|\vb{p}|^2 
	\qquad (A,\vb{B},C \text{ctes. adimensionales})
\]
que lista los {\it invariantes colisionales}. Completando cuadrados
\[
	f \propto C_1 \euler^{-C_2(\vb{p} - \vb{p}_0)^2}
\]

La expresión completa se ajusta con 
\[
	n = \int f(\vb{p},t) d^3p
\]
donde el $\vb{p}$ de una partícula es
\[
	<\vb{p}> = \frac{\int f(\vb{p}) \vb{p} \; d^3p d^3q}{\int f(\vb{p}) \; d^3p d^3q } = 
	\frac{1}{n} \int f(\vb{p}) \; \vb{p} \; d^3p
\]
\notamargen{El cociente es $\vb{P}/N$.}
y la energía por partícula
\[
	<e> = \frac{\int f(\vb{p}) \; \vb{p}^2/ (2m) \; d^3p d^3q}{\int f(\vb{p}) d^3p d^3q } = 
	\frac{1}{n} \int f(\vb{p}) \frac{ \vb{p}^2 }{ 2m } \; d^3p
\]

Finalmente se llega a 
\[
	f(\vb{p}) = \frac{n}{(2\pi m k T)^{3/2}} \euler^{- \frac{( \vb{p} - \vb{p}_0 )^2}{2mkT} }
\]

que es la función de distribución de momentos de Maxwell-Boltzmann.
\notamargen{Solución de equilibrio de la ecuación de transporte}

\[
	\text{(presión ideal)} \qquad p = \frac{2}{3} \frac{U}{V} = \frac{2}{3} n \epsilon =
	\frac{2}{3} n \frac{3}{2} k T = nkT 
\]

\subsection{Método de la distribución más probable}

Con este método también llegamos a $f_{MB}$ pero extremandolo el volumen $\Omega(\{ n_i \})$ que ocupa en el espacio
$\mathbb{\Gamma}$ sujeto a los vínculos $E = \sum_i n_i e_i$ y $N = \sum_i n_i $.

Luego podemos estimar qué tan probable es la distribución de MB (la más probable) considerando
(ASUMIMOS)
\[
	\text{los \# de ocupación de MB} \quad \tilde{n}_i \cong <n_i> \quad \text{el promedio en el ensamble}
\]
pero esto sólo valdrá si las desviaciones son pequeñas; es decir si $f_{MB}$ es muy muy probable.

Calculamos la desviación cuadrática (varianza) se tiene 
\[
	<n_i^2> - <n_i>^2 = g_i \dpar{<n_i>}{g_i}
\]
donde se usó que 
\[
	<n_i> = \frac{\sum_{\{ n_j\}} n_i \Omega\{ n_j\} }{\sum_{\{ n_j\}} \Omega\{ n_j\}}
\]

Suponiendo que $ <n_i> \approx \tilde{n}_i$ entonces $ <n_i>  \propto f_{MB}$ con lo cual se tiene también 
\[
	<n_i^2> - <n_i>^2 \cong \tilde{n}_i
\]
\notamargen{como $ g_i \dpar{\tilde{n}_i}{g_i} = \tilde{n}_i $}
y las fluctuaciones relativas
\[
	\sqrt{<\left( \frac{m_i}{N}\right)^2 > - <\left( \frac{m_i}{N}\right) >^2 } \cong 
	\sqrt{ \frac{ \tilde{n}_i/N }{N} }\to_{N\to\infty} 0
\]

En el límite termodinámico MB es totalmente dominante.

\subsection{Hipótesis ergódica}

La trayectoria individual de casi cualquier punto en el $\Omega$ pasa, con el tiempo, a través de todos los
puntos permitidos del espacio $\mathbb{\Gamma}$. Si esperamos lo suficiente, todos los microestados posibles
son visitados.

\subsection{Observaciones sobre el microcanónico}

\[
	\Gamma(E) = \int_{E < \Ham < E + \Delta E} \rho d^{3n}p d^{3n}q \qquad 
	\Sigma(E) = \int_{\Ham < E} \rho d^{3n}p d^{3n}q
\]
entonces 
\[
	\Gamma(E) = \Sigma( E + \Delta E ) - \Sigma( E ) \cong \dpar{\Sigma( E )}{E}\Delta E  
	\qquad \text{si}\; \Delta E \ll E
\]
$\Delta E$ es el {\it paso} entre medidas de energía 
\[
	\Gamma(E) = \Gamma_1(E_1) \Gamma_2(E_2) \qquad \text{(1 y 2 son subsistemas)}
\]
\[
	E = E_1 + E_2 \Rightarrow \Gamma(E) = \sum_i^{E/\Delta E} \Gamma_1(E_i)\Gamma_2(E-E_i)
\]
siendo $E/\Delta E$ el número de términos tales que se cumple $ E = E_1 + E_2 $.
Si se da $ N_1 \to \infty $ y $ N_2 \to \infty $ será
\[
	\log \Gamma_1 \propto N_1 \quad \log \Gamma_2 \propto N_2 \quad E \propto N_1 + N_2
\]
luego $\log(E/\Delta E)$ es despreciable pues $\Delta E$ es constante y entonces
\notamargen{$\log(E/\Delta E) \propto \log(N)$ pues $E\propto N$ y $\Delta E$ cte.}
\[
	S(E,V) = S(\tilde{E}_1,V_1) + S(\tilde{E}_2,V_2) + \mathcal{O}(\log[N])
\]
con lo cual la mayoría de los microestados tienen los valores $\tilde{E}_1$ y $\tilde{E}_2$ de energía.

Asimismo
\[
	\delta( \Gamma_1(\bar{E}_1)  \Gamma_2(\bar{E}_2) ) = 0 \qquad \delta( \bar{E}_1 + \bar{E}_2 ) = 0
\]
\[
	\delta\Gamma_1 \Gamma_2 + \Gamma_1 \delta \Gamma_2 = 0 \quad \delta( \bar{E}_1 ) = -\delta ( \bar{E}_2 )
\]
\[
	\frac{\delta\Gamma_1}{\bar{E}_1}\Gamma_2 = \Gamma_1\frac{\delta\Gamma_2}{\bar{E}_2} \Rightarrow 
	\frac{1}{\Gamma_1} \dpar{\Gamma_1}{\bar{E}_1} = \frac{1}{\Gamma_2} \dpar{\Gamma_2}{\bar{E}_2} 
\]
\[
	\dpar{}{\bar{E}_1}\left( k\log \Gamma_1(\bar{E}_1) \right) = 
	\dpar{}{\bar{E}_2}\left( k\log \Gamma_1(\bar{E}_2) \right)
\]
\[
	\left. \dpar{}{{E}_1}S(E_1)\right|_{\bar{E}_1} = \left. \dpar{}{{E}_2}S(E_2)\right|_{\bar{E}_2}
	\equiv \frac{1}{T} \qquad \text{en equilibrio} \; T_1 = T_2
\]

La $T$ es el parámetro que gobierna el equilibrio entre partes del sistema.

La idea es que dado un sistema de $E = E_1 + E_2$, sistema compuesto de dos subsistemas, hay muchos valores
1,2 tales que $E = E_1 + E_2$ pero hay una combinación que maximiza $\Gamma(E)$ y es
\[
	\Gamma_{Max}(E) = \Gamma_1(\bar{E}_1)  \Gamma_2(\bar{E}_2) 
\]
\notamargen{El sistema es $E,N,V$ y yo lo pienso compuesto de dos partes $E_1,N_1,V_1$ y $E_2,N_2,V_2$.}

Luego, con $N_1, N_2 \to \infty$ se da que la mayoría de los sistemas tendrán $E_1=\bar{E}_1$ y $E_2=\bar{E}_2$.
Esa configuración, por supuesto, maximiza la entropía $S=k\log(\Gamma)$.

El hecho de que $\Delta S> 0$ para un sistema aislado lo vemos considerando que tal sistema sólo puede variar
$V$ (creciendo, como en la expansión libre de un gas), luego $V_F > V_I$ y entonces
\[
	\Sigma(E) = \int_{\Ham < E} \rho d^{3N}p d^{3N}q \underbrace{\longrightarrow}_{\text{Si aumento el volumen}}
	\Sigma(E)' = \int_{\Ham < E} \rho d^{3N}p d^{3N}q
\]
\notamargen{Será un número mayor porque el dominio de integración en $q$ es mayor.}
\[
	\Sigma(E)' > \Sigma(E) \qquad \Rightarrow \qquad \Delta S > 0
\]

Equipartición implica 
\[
	\left< x_i \dpar{\Ham}{x_j} \right> = \delta_{ij} k T
\]
y entonces
\[
	\left< p_i \dpar{\Ham}{p_i} \right> = \left< p_i \dot{q}_i \right> = kT 
\]
y
\[
	\left< q_i \dpar{\Ham}{q_i} \right> = \left< q_i \dot{p}_i \right> = kT 
\]
\[
	\left< \sum_i^{3N} q_i \dpar{\Ham}{q_i} \right> =  \sum_i^{3N} \left< q_i \dpar{\Ham}{q_i} \right> =
	\sum_i^{3N} k T = 3 N k T
\]
entonces llegamos al virial,
\[
	\sum_i^{3N} \left< q_i \dot{p}_i \right> = 3 N k T.
\]

Considerando un hamiltoniano armónico,
\[
	\left< \Ham \right> = E \qquad \text{con} \quad \Ham = \sum_i^{3N} a_i p_i^2 + b_i q_i^2
\]
\[
	p_k \dpar{\Ham}{p_k} = 2 a_k p_k^2 \qquad q_k \dpar{\Ham}{q_k} = 2 b_k q_k^2
\]
de modo que 
\[
	\Ham =  \sum_i^{3N} \frac{1}{2} p_k \dpar{\Ham}{p_k} + \frac{1}{2} q_k \dpar{\Ham}{q_k}
\]
\[
	\left< \Ham \right> =  \sum_i^{3N} \frac{1}{2} \left< p_k \dpar{\Ham}{p_k}\right> +
		\frac{1}{2} \left< q_k \dpar{\Ham}{q_k} \right>
\]
y si $f$ es el número de constantes $a_k,b_k$ no nulos
\[
	\left< \Ham \right> =  \frac{1}{2} f k T
\]

Si fuesen todas no nulas entonces
\[
	\left< \Ham \right>  = 3 N k T.
\]

\subsection{Gas ideal (microcanónico)}

\[
	\Ham =  \sum_i^{N} \frac{p^2_i}{2m}
\]
\[
	\Sigma(E) = \frac{1}{h^{3N}} \int_{\Ham < E} d^3p_1 ... d^3p_N d^3q_1 ... d^3q_N = 
	\left( \frac{V}{h^{3N}} \right)^N \int_{\Ham < E} d^3p_1 ... d^3p_N
\]
donde la integral en $\{ q_i\}$ es inmediata porque no están los mismos en los límites y donde el
límite de integración $\Ham < E$ implica la condición 
\[
	p_1^2 + p_2^2 + ... + p_N^2 < ( \sqrt{2mE} )^2
\]
\notamargen{Es una especie de radio $2mE$.}
\[
	\Sigma(E) = C_{3N} \left[ \frac{V}{h^3} (2mE)^{3/2}\right]^N
\]

Luego,
\[
	S = k \log \left\{ C\left( \frac{V}{h^3}(2mE)^{3/2}\right)^N \right\}
\]
\[
	S = k \log C + N k \log \left[ \frac{V}{h^3}(2mE)^{3/2}\right]
\]
\notamargen{$ k\log C \approx -3/2 Nk \log 3N/2 $}
\[
	\left. \dpar{S}{E} \right|_{V,N} = \frac{1}{T} \qquad \Rightarrow \qquad \frac{1}{T} = Nk\frac{3}{2}\frac{1}{E}
\]
y entonces
\[
	E = \frac{3}{2} NkT \qquad \text{gas ideal}
\]
\notamargen{Vemos que la termodinámica es bastante insensible a las aproximaciones.}

\subsection{Paradoja de Gibbs}

\[
	S \propto Nk\log(V) + Nk \log (E^{3/2})
\]
Supongamos dos gases idénticos con la misma $\rho$ y $T$
\notamargen{Quitar la pared es una operación mental si los gases son idénticos (o al menos eso podemos pensar).}

\[
	\Delta S = Nk \log V + Nk \log (E^{3/2}) - N_1k \log V_1 - N_2k \log (E_1^{3/2})
	- N_1k \log V_2 - N_2k \log (E_2^{3/2})
\]
\[
	\Delta S = N_1 k \log \left( \frac{V}{V_1} \right) + N_2 k \log \left( \frac{V}{V_2} \right) +
		N_1 k \log \left( \frac{E}{E_1} \right)^{3/2} + N_2 k \log \left( \frac{E}{E_2} \right)^{3/2}
\]
\notamargen{Si los gases son distintos está correcto $\Delta S > 0$ pero si son idénticos no porque un estado
como F podría provenir de infinitas compartimentacionales las cuales darían todas difrentes $\Delta S$ y entonces
la entropía $S$ no sería función de estado.}
\[
	\Delta S > 0 \quad \text{pues: } \; \frac{V}{V_1} = 1 + \frac{V_2}{V_1} > 1, \frac{V}{V_2} > 1, 
	\frac{E}{E_1} > 1, \frac{E}{E_2} > 1
\]

Podemos hacer algo menos cuentoso tomando
\[
	S \propto Nk\log \left( V\left[ \frac{4\pi m E}{3 h^2 N} \right]^{3/2} \right)
\]
donde la $N$ viene de $k\log C_{3N}$ con $N \to \infty$. Vemos que $E/N$ mantiene el cambio en $S$ respecto de $E$
igual, puesto que 
\[
	\frac{E}{N} = \frac{E_1 + E_2}{N_1 + N_2} = \frac{E_1}{N_1} = \frac{E_2}{N_2} = \epsilon
\]
pero $V$ no balance. Luego la inclusión de $1/N!$ hará que 
\[
	S = k \log (\frac{1}{N!}\Sigma(E,N,V)) = k \log (\Sigma) - k\log N!
\]
de forma que resultará
\[
	S \propto Nk\log \left( \frac{V}{N}\left[ \frac{4\pi m E}{3 h^2 N} \right]^{3/2} \right)
\]
y esta $S$ sí está libre de paradoja de Gibbs.

% =================================================================================================
\section{Canónico}
% =================================================================================================

Consideramos un microcanónico con 
\[
	E = E_1 + E_2, \qquad N = N_1 + N_2, \qquad V = V_1 + V_2 
\]
donde $N_i, V_i$ están fijos y $E_i$ varían de acuerdo a
\notamargen{Imagen del microcanónico...}
\[
	E = E_1 + E_2
\]

Consideramos un microcanónico 
\[
	\Gamma(E) = \Sigma_{E_1} \Gamma_1(E_1) \Gamma_2(E-E_1) \leq C \Gamma_1(\bar{E}_1) \Gamma_2(E-\bar{E}_1)
	\approx C \Gamma_2(\bar{E}_1)
\]
\[
	S(E-\bar{E}_1) \approx k \log \Gamma_2(E-\bar{E}_1)
\]
\[
	S(E) + \left.\dpar{S(E)}{E}\right|_E(-\bar{E}_1) \approx k \log \Gamma_2(E-\bar{E}_1)
\]
\[
	\euler^{\frac{S(E)}{k}} \euler^{-\frac{E_1}{kT}} \approx \Gamma_2(E-\bar{E}_1)
\]

Claramente como '1' siempre está metido dentro de '2' entre mayor sea el $\Gamma_2$ mayor también el tamaño de '1'
en $\mathbb{\Gamma}$, luego:
\[
	\# \text{de config en } \mathbb{\Gamma} \text{ del sistema '1+2'} = \# \text{de config de '1' en '2'} \times 
	\# \text{de config de '2' en} \mathbb{\Gamma}
\]
\[
	\# \text{ config '1' } = \frac{ \# \text{ config '1+2'} }{ \# \text{ config '2'} } \approx 
	\euler^{-\frac{E_1}{kT}} = C \int \euler^{-\Ham/kT} d^3p d^3q
\]
\[
	Q_N(V,T) = \frac{1}{h^{3N}N!}\int \euler^{-\Ham/kT} d^3p d^3q
\]
\notamargen{$1/N!$ es el factor de buen conteo.}

La función de partición es el volumen ocupado en $\mathbb{\Gamma}$.
El vínculo con la termodinámica viene de
\[
	Q_N(V,T) = \euler^{-\beta A}
\]
\[
	A = -kT\log [Q_N(V,T)]
\]
donde $A=A(T,V,N)$ es la energía libre de Helmholtz. Podemos ver que se deduce esto de 
\[
	< \Ham > = E = -\dpar{}{\beta} \log [ Q_N(V,T) ] = A + TS = A - T\left.\dpar{A}{T}\right|_{N,V}
\]
pero 
\[
	\dpar{}{\beta} = \dpar{}{T} \dpar{T}{\beta} = -k T^2 \dpar{}{T}, \qquad \text{pues } \dpar{\beta}{T} = - 
	\frac{1}{kT^2}
\]
\[
	\dpar{}{T}\left( \frac{A}{T} \right) = -\frac{A}{T^2} + \frac{1}{T}\dpar{A}{T}
\]
de modo que 
\[
	-T^2 \dpar{}{T}\left( \frac{A}{T} \right) = A - T \dpar{A}{T}
\]
\notamargen{$S=-\partial A / \partial T|_{N,V}$}
y entonces
\[
	E = -k T^2 \dpar{}{T}\log Q_N = -T^2 \dpar{}{T}\left( \frac{A}{T} \right) 
\]
de lo que se desprende
\[
	\log Q_N = -\frac{A}{kT}
\]

Podemos usar $E=A+TS$ y llegar a $Q_n=\exp(-\beta A)$ o bien $Q_N=exp(-\beta A)$ y llegar a $E=A+TS$.

\subsection{Equivalencia canónico y microcanónico}

Vemos cómo son las fluctuaciones de energía en el canónico. Desde 
\[
	U = <\Ham> = \frac{\int \euler^{-\beta\Ham} \Ham d^3p d^3q}{\int \euler^{-\beta\Ham} d^3p d^3q}
\]
\[
	\int \euler^{-\beta\Ham} U d^3p d^3q = \int \euler^{-\beta\Ham} \Ham d^3p d^3q
\]
\[
	\dpar{}{\beta}\left[ \int \euler^{-\beta\Ham} (U-\Ham) d^3p d^3q\right] = \dpar{}{\beta}\left[ 0 \right] = 0
\]
\[
	<\Ham^2> - <\Ham>^2 = kT^2C_V
\]

Las fluctuaciones van como el $C_V$, luego 
\[
	<\Ham^2/N^2> - <\Ham/N>^2 = kT^2c_V/N \qquad \text{donde } c_V = C_V/N
\]
\notamargen{$<\Ham> \propto N$ y $C_V \propto N$}
de modo que las fluctuaciones relativas van a 0 con $N\to\infty$.

Otro modo de verlo es considerando 
\[
	\frac{1}{h^{3N}N!}\int \euler^{-\beta\Ham} d^3p d^3q = \int_0^\infty dE \dpar{\Sigma(E)}{E} \euler^{-\beta E} =
	\int_0^\infty dE \euler^{-\beta E + \log (\partial \Sigma(E)/\partial E)}
\]
donde 
\[
	\dpar{\Sigma(E)}{E} dE = \frac{d^3p d^3q}{h^{3N}N!}
\]
y como $S/k = \beta TS$
\[
	Q_N = \int_0^\infty dE \euler^{-\beta E + \beta T S }
\]

Si suponemos que es $S$ máxima en $E=\bar{E}$ entonces $S_{MAX} = S(\bar{E})$ y será 
\[
	\left. \dpar{S}{E} \right|_{\bar{E}} = 0
\]
con lo cual
\[
	E + TS \cong \bar{E} + TS(\bar{E}) + \frac{1}{2}(E-\bar{E})^2 T \left. \dpar[2]{S}{E} \right|_{\bar{E}}
\]
\[
	E + TS \cong \bar{E} + TS(\bar{E}) - (E-\bar{E})^2 \frac{1}{2kTC_V}
\]
de modo que 
\[
	Q_N = \int_0^\infty dE \euler^{-\beta [\bar{E} + TS(\bar{E})] - \beta \frac{(E-\bar{E})^2}{2kTC_V}}
\]
\[
	Q_N = \euler^{-\beta [\bar{E} + TS(\bar{E})]} \int_0^\infty dE \euler^{- \beta \frac{(E-\bar{E})^2}{2kTC_V}}
\]
y vemos que la integral se va a una delta con $N\to \infty$ (pués $C_V \propto N$) en cuyo caso
\[
	Q_N = \euler^{-\beta [\bar{E} + TS(\bar{E})]} 
\]
y la mayor parte de los estados tienen energía $\bar{E}$, que es la de un sistema aislado a temperatura $T$.

La densidad de estados va entonces de acuerdo al producto de dos efectos contrarios:
\[
	g(E) = \dpar{\Sigma(E)}{E}\euler^{-\beta E}
\]

\subsection{Ejemplos sencillos}

\[
	\Ham = \sum_i^N \frac{p_i^2}{2m} + \frac{m}{2}\omega_i^2 q_i^2 \quad \qquad \text{oscilador clásico 1D}
\]
\[
	\Ham = \sum_i^N \left( n_i + \frac{1}{2} \right)\hbar\omega  \quad \qquad \text{oscilador Schrödinger 1D}
\]
\[
	\Ham = \sum_i^N  n_i \hbar\omega \quad \qquad \text{oscilador Planck 1D}
\]

\[
	U = NkT \rightarrow C_V = Nk  \qquad \text{Clásico}
\]
\[
	U \approx \frac{N\hbar\omega}{2} \quad U \approx 0 (T\ll 1) \qquad \rightarrow C_V = 0 
	\quad \text{Schrödinger-Planck}
\]
\[
	U \approx N kT \; (T \gg 1) \qquad \rightarrow C_V = Nk 
	\quad \text{Schrödinger-Planck}
\]

Los casos Schrödinger y Planck aproximan al $C_V$ clásico con $T$ altas.


\subsection{Una derivación más del canónico}

El tamaño del sistema '1' en $\mathbb{\Gamma}$ (su volumen $\Gamma_1(E_1)$) será proporcional al tamaño del sistema 
'2' en $\mathbb{\Gamma}$ (su volumen $\Gamma_2(E-E_1)$) de manera que 
\[
	\Gamma_1(E_1) \propto \Gamma_2(E-E_1)
\]
\[
	k\log \Gamma_1(E_1) \approx S(E) + \left.\dpar{S}{E}\right|_E (-E_1) = S(E) - \frac{E_1}{T} 
	\text{ (del sistema '2') }
\]
\[
	\Gamma_1(E_1) \approx \euler^{S(E)/k} \euler^{-E_1/kT} 
\]
\[
	\text{ \# conf '1' } = \text{ \# conf '2' } \times \text{ densidad del '1' en el '2' }
\]
y finalmente
\[
	Q_N (V,T) = \frac{1}{h^{3N}N!} \int d^{3N}p d^{3N}q \euler^{-\Ham(\{ p_i,q_i\})/kT}
\]

% =================================================================================================
\section{El gran canónico}
% =================================================================================================

\[
	Q_N (V,T) =  \frac{1}{h^{3N}N!} \int d^{3N_1}p_1 d^{3N_2}p_2  \sum_{N_1=0}^N \frac{N!}{N_1! N_2!}
	\int d^{3N_1}q_1 d^{3N_2}q_2 \euler^{-\beta [\Ham_1 + \Ham_2 ]}
\]
\[
	Q_N (V,T) =  \frac{1}{h^{3N_1} h^{3N_2} } \sum_{N_1=0}^N \frac{1}{N_1! N_2!}
	\int d^{3N_1}p_1 d^{3N_1}p_1 \euler^{-\beta\Ham_1} \int d^{3N_2}q_2 d^{3N_2}q_2 \euler^{-\beta\Ham_2}
\]
\[
	Q_N (V,T) =  \sum_{N_1=0}^N \int \frac{1}{h^{3N_1}N_1!} d^{3N_1}p_1 d^{3N_1}p_1 \euler^{-\beta \Ham_1 }
	\int \frac{1}{h^{3N_2}N_2!} d^{3N_2}q_2 d^{3N_2}q_2 \euler^{-\beta \Ham_2 }
\]
\[
	1 = 
	\sum_{N_1=0}^N \frac{1}{h^{3N_1}N_1!} \int d^{3N_1}q_1 d^{3N_1}p_1 \; 
	\euler^{-\beta\Ham_1} \frac{Q_{N_2}(V_2,T)}{Q_N(V,T)} 
\]
\[
	1 = 
	\sum_{N_1=0}^N \int d^{3N_1}q_1 d^{3N_1}p_1 \; \frac{\euler^{-\beta\Ham_1}}{h^{3N_1}N_1!} 
	\frac{Q_{N_2}(V_2,T)}{Q_N(V,T)} 
\]
siendo el último factor un $ \rho(\{ p_1,q_1\},N_1)$
\[
	\frac{Q_{N_2}(V_2,T)}{Q_N(V,T)} = \euler^{-\beta A (V-V_1,N-N_1,T)}\euler^{-\beta A (V,N,T)} =
	\euler^{-\beta [ \frac{\delta A}{\delta V} \delta V + \frac{\delta A}{\delta N} \delta N ] }
\]
donde las diferencias $\delta$ se toman discretas:
\[
	\frac{\delta A}{\delta V} \delta V + \frac{\delta A}{\delta N} \delta N =
	(-p )(-V_1) + \mu (-N)_1 = pV_1 - \mu N_1
\]
\[
	A = U - TS \qquad dA = dU - TdS - SdT = -pdV + \mu dN - SdT
\]
\[
	\frac{Q_{N_2}(V_2,T)}{Q_N(V,T)} = \euler^{-\beta PV_1 + \beta \mu N_1},
\]

De forma que la densidad del sistema '1' es
\[
	\frac{1}{h^{3N_1}N_1!} \euler^{-\beta\Ham_1}  \euler^{-\beta PV_1}  \euler^{\beta \mu N},
\]
y definiendo $z \equiv \euler^{\beta\mu}$
\[
	\rho(\{p,q\},N) = \frac{z^N}{h^{3N}N!} \euler^{-\beta\Ham}  \euler^{-\beta PV} 
\]

Nótese que $ \mu, P, V, T$  son los valores fijos del sistema mayor y hemos sacado subíndices.
\[
	1 = \sum_{N=0}^\infty \int d^{3N}q d^{3N}p \frac{z^N}{h^{3N}N!} \euler^{-\beta\Ham}  \euler^{-\beta PV} 
\]
\[
	\euler^{\beta PV} = \sum_{N=0}^\infty \frac{z^N}{h^{3N}N!} \int d^{3N}q d^{3N}p \euler^{-\beta\Ham}
	= \sum_{N=0}^\infty z^N Q_N(V,T)
\]
\be
	\beta PV = \log \left( \sum_{N=0}^\infty z^N Q_N(V,T) \right)
	\label{betaPV}
\ee
y tenemos 
\[
	\Xi(z,V,T) \equiv \sum_{N=0}^\infty z^N Q_N(V,T)
\]
que es la gran función de partición.
La termodinámica puede extraerse desde 
\[
	<N> = z\dpar{}{z} \log [ \:\Xi(z,V,T) \:]     \qquad 
	<E> = -\dpar{}{\beta} \log [ \:\Xi(z,V,T)\: ]
\]

La ecuación de estado se obtiene reemplazando $z$ en la expresión de \eqref{betaPV} y en $<N>$


\subsection{Fluctuaciones de densidad}

\[
	<N^2> - <N>^2 =  z\dpar{}{z}\left( z\dpar{}{z} \log \Xi \right)= kTV \dpar[2]{P}{\mu}
\]
\[
	<N^2> - <N>^2 = kTV \dpar{}{\mu}\frac{1}{v} = kTV \frac{1}{v^2}\kappa_T = kT\frac{N^2}{V}\kappa_T 
	= NkT \frac{\kappa_T}{v}
\]
\notamargen{Viene de $\dpar{}{\mu}\frac{1}{v} = -\frac{1}{v^2} \frac{1}{v} \dpar{v}{P} = \frac{1}{v^2}\kappa_T $}

Si $A=Na$ entonces $a=u-Ts$ y entonces 
\[
	\dpar{a}{v} = -p
\]
\[
	U = TS-pV+\mu N \quad \Rightarrow \quad u = Ts - pv
\]
\[
	\dpar{\mu}{v} = -P -v\dpar[2]{a}{v} + p = v \dpar{p}{v} \qquad \dpar{p}{\mu} \qquad 
			= \frac{\dpar{p}{v}}{\dpar{\mu}{v}}=\frac{1}{v}
\]
pues 
\[
	u - Ts = a = - pV + \mu \qquad \mu = a + pv
\]

Las fluctuaciones relativas tiende a cero cuando $N\to\infty$ provistos de que $\kappa_T < \infty$. Esto no vale 
en la transición de fase de primer oden pues 
\[
	\left. \dpar{p}{v} \right|_{\text{punto crítico}} = 0 \qquad \frac{1}{v} \dpar{v}{p} \to \infty
\]
Se calculan como 
\[
	\sqrt{\frac{<N^2> - <N>^2}{N^2}} = \sqrt{ kT\dpar{\kappa_T}{v}\frac{1}{N}} \to 0 \text{ si } N\to\infty
\]

\subsection{Fluctuaciones de energía}

\[
	<\Ham^2> - <\Ham>^2 = kT^2 \left( \dpar{U}{T} \right)_{z,V}
\]
y como 
\[
	\left( \dpar{U}{T} \right)_{z,V} = \left.\dpar{U}{T}\right|_{N,V} + \left.\dpar{U}{N}\right|_{T,V} 
	\left.\dpar{N}{T}\right|_{z,V}
\]
\[
	<\Ham^2> - <\Ham>^2 = kT^2C_V +  \left[\left. \dpar{U}{N} \right|_{T,V}\right]^2 <(\Delta N)^2>
\]
siendo $ kT^2C_V $ fluctuación del canónico y $(\Delta N)^2 = <N^2> - <N>^2 $


\subsection{Gas ideal}

\[
	Q_N = \frac{(Vf(T))^N}{N!} \Rightarrow \Xi = \sum_{N=0}^\infty \frac{(zVf(t))^N}{N!} = \euler^{zVf(T)}
\]
\[
	\beta pV = \log (\Xi) = zVf(T) \qquad <N> = z \dpar{}{z} \log (\Xi) = zVf(T)
\]
y luego 
\[
	\beta pV = <N> \qquad \rightarrow \quad pV = <N> k T
\]
y recuperamos la ecuación de estado del gas ideal.


\subsection{Equivalencia canónico-gran canónico}

Para ver que con $ N \to \infty $ son equivalentes consideramos 
\[
	\kappa_T = \frac{1}{v} \left( -\dpar{v}{p} \right) < \infty \qquad \dpar{p}{v} < 0
\]

Pero en la coexistencia de una transición de fase de 1er orden se da 
\[
	\dpar{p}{v} = 0  \rightarrow \kappa_T \to \infty \text{ (sistema homogéneo) }
\]

La idea es ver que 
\begin{itemize}
 \item Dado $z$ existe $N$ tal que $ \Xi = \sum_N z^N Q_N(V,T) $
 \item Dado $N$ existe $z$ tal que $ \Xi = \sum_N z^N Q_N(V,T) $
\end{itemize}

Esto se comprueba. Además, si:
\[
	W (N) = z^N Q_N (V,T) \propto \text{ Prob. de que el sistema tenga $N$ partículas }
\]

XXX dibujos XXXX

En la transición de fase, donde $ \dpar{p}{v} = 0  $ todos los $ N $ son igual de probables porque
fluctúa la densidad. La $p$ se mantiene constante pero se varían los $ N_i $ de cada fase 'i'.


\subsection{Otra derivación del gran canónico}

Podemos derivar el gran canónico desde 
\notamargen{Es la probabilidad de hallar al sistema '1' en un estado con $ E_1, N_1 $.}
\[
	\text{Prob } \propto \Gamma_2(E-E_1, N-N_1)
\]
\[
	\log  \Gamma_2(E-E_1, N-N_1) \cong  \log \Gamma_2( E, N ) + \frac{1}{k} \left. \dpar{S(E,N)}{E} \right|_E(-E_1)
	+\frac{1}{k} \left. \dpar{S(E,N)}{N} \right|_N(-N_1)
\]
\[
	\cong \log \Gamma_2( E, N ) - \frac{E_1}{kT} + \frac{N_1\mu}{kT}
\]
\[
	\text{Prob } \propto \euler^{-\beta E}  \euler^{\beta \mu N}  = \euler^{-\beta E} z^N
\]
donde $T$ y $\mu$ son las asociadas al baño.
\notamargen{$\partial S/\partial E = 1/T$ y $\partial S/\partial N = -\mu / T$.}

Pensamos en $\eta$ copias del sistema; $n_{E_1N_1} = \# $ de sistemas con energía $E_1$ y $N_1$ partículas,
luego 
\[
	\sum_{\{ E_1, N_1 \}} n_{E_1N_1} = \eta \qquad \sum_{\{ E_1, N_1 \}} n_{E_1N_1}E_1 = n\bar{E}_1 \cong 
	\text{ Energía Total }
\]
\[
	\sum_{\{ E_1, N_1 \}} n_{E_1N_1} N_1 = \eta \bar{N}_1 \cong \text{ \# Total de partículas (no físico) }
\]
donde $ \bar{N}_1 $ es el número de medio.
\[
	\Omega\{ n_{E_1N_1} \} = \frac{\eta !}{\prod (n_{E_1N_1})!} \qquad \text{ combinatorio }
\]

La conbinación de mayor volumen será 
\[
	\log \Omega - \alpha \sum n E_1 - \beta_L \sum n N_1 = 0
\]
\[
	-\sum \left[ n\log n - n - \alpha n E_1 - \beta_L n N_1 \right] = 0
\]
\[
	-\sum n \left[ \log n - 1 - \alpha E_1 - \beta_L N_1 \right] = 0 
	\rightarrow \log(\tilde{n}) = 1 + \alpha E_1 + \beta_L N_1
\]
\[
	\tilde{n} \propto \euler^{\alpha E_1 + \beta_L N_1}
\]
que es el conjunto $n_{E_1N_1}$ de mayor volumen en $ \Omega $.

Esperaremos qeu con $ \eta\to\infty $ sea $<n_{E_1N_1}> \cong \tilde{n}_{E_1N_1} $.
Para determinar $\alpha, \beta$ usaremos 
\[
	\tilde{N} \cong <N> = \dpar{}{\beta_L}\left( \log \sum_{\{ E_1, N_1 \}} 
	\euler^{\alpha E_1 + \beta_L N_1} \right)
\]
\[
	\tilde{E} \cong <\Ham> =  \dpar{}{\alpha}\left( \log \sum_{\{ E_1, N_1 \}}
	\euler^{\alpha E_1 + \beta_L N_1} \right)
\]

% =================================================================================================
\section{Entropía de Gibbs}
% =================================================================================================

Sea $X$ extensiva mecánica,
\[
	S = k \log \Gamma (E,X) \qquad dU = TdS + Y dX, \; \frac{dS}{k} = \beta dU + \xi dX
\]
\notamargen{Donde $\beta Y = \xi $}
Refiriéndo al estado $ \nu $
\[
	P_\nu = \frac{ \euler^{-\beta E_\nu - \xi X_\nu } }{ \sum_\nu \euler^{-\beta E_\nu - \xi X_\nu } } =
	\frac{ \euler^{-\beta E_\nu - \xi X_\nu }}{\Theta}
\]
\[
	<E> = -\dpar{}{\beta} \log \Theta  \qquad <X> = -\dpar{}{\xi} \log \Theta 
\]
\notamargen{Caso $X=N$ $z\dpar{}{z} \cong \dpar{}{\beta \mu }$ }
\[
	d( \log \Theta ) = -<E> d\beta - <X> d\xi 
\]

Sea 
\[
	\Lag \equiv -k \sum_\nu P_\nu \log P_\nu =
	-k \sum_\nu P_\nu \log \left[ \euler^{-\beta E_\nu - \xi X_\nu } \Theta^{-1} \right]
\]
\[
	\Lag = \sum_\nu P_\nu k \log \Theta + k P_\nu \beta E_\nu + k P_\nu \xi X_\nu
\]
\[
	\Lag = k\log \Theta + k\beta <E> + k\xi <X>
\]
\[
	d\Lag = k\beta d<E> + k\xi d<X>
\]

Es una transformada de Legendre que toma $\log \Theta$ y la lleva a una función de $ <E>, <X> $
\[
	d\Lag = k \beta dE + k \beta Y dX = dS = \frac{1}{T}dE + \frac{Y}{T} dX 
\]
entonces $\Lag$ es la entropía $S$.
\[
	\Lag = -k \sum_\nu P_\nu \log P_\nu 
\]
y $\nu$ son equiprobables
\[
	\Lag = -k \sum_\nu \frac{1}{\Gamma} \log \left( \frac{1}{\Gamma} \right) = 
	\sum_\nu \frac{k}{\Gamma} \log(\Gamma)
\]
y entonces
\[
	\Lag = k \log( \Gamma ) \equiv S.
\]


\subsection{Observación promedios}

\[
	<G> = \frac{\sum_N z^N G Q_N(V,T) }{\Xi} = \frac{\sum_N z^N \sum_\nu G(E_\nu, N, T) Q_N(V,T) }{\Xi}
\]
donde el último factor en la sumatoria es $<G>_{\text{CAN}} Q_N(V,T)$.

La parte crítica está en el pasaje de 
\[
	\sum_\nu \euler^{ -\beta E_\nu }
\]
a algún índice útil que permite realizar la sumatoria. En el caso de cuasipartículas, como osciladores, 
tenemos
\[
	\hat{H} = \sum_i^N \left( n_i + \frac{1}{2} \right) \hbar \omega_i 
\]
donde $ n_i $ es el número de fotones del oscilador i-ésimo. Los fonones cumplen el rol de partículas
\footnote{Porque podemos considerar que la $\sum$ se hace en niveles energéticos en lugar de entre osciladores
y tenemos un \# indeterminado de ``particulas'' (fonones) distribuidas en 'N' niveles energéticos.}
Un oscilador ddado puede tener en principio cualquier valor de energía (cualquier valor de $ n_i $) y esto 
independientemente de los otros $ N-1 $ osciladores. El número total de fonones del sistema
\[
	\sum_i^N n_i
\]
no es una constante del mismo con lo cual no hay vínculo. Entonces
\[
	\sum_\nu \qquad \rightarrow \qquad \sum_{n_1=0}^\infty \sum_{n_2=0}^\infty ... \sum_{n_\nu=0}^\infty
\]


\section{SUELTO: reubicar}

\[
	Z_N = \int d^{3N}q \prod_{i<j}^N (1+f_{ij}) \qquad \text{ integral configuracional }
\]
En realidad esta integral serán $ N(N-1)/2 $ integrales (N-grafos). Podemos factorizar los $ N(N-1)/2 $ grafos
en l-racimos teniendo en cuenta que se cumple
\[
	N = \sum_{l=1}^N ln_l,
\]
de forma que cada N-grafo dtermina un conjunto $ \{ m_l \} = (m_1,m_2, ..., m_N) $ de '$m_1$' 1-racimos, '$m_2$' 
2-racimos y '$m_N$' N-racimos. Por supuesto, un mismo conjunto $ \{ m_l \} $ determina muchos (en principio) N-grafos 
en función de la permutación de etiquetas.
\[
	\frac{N(N-1)}{2} \text{ N-grafos } \rightarrow M \text{ conjuntos } \{ m_l \}
\]
y la 
\[
	Z_N = \sum_1^{N(N-1)/2 } \text{ N-grafos } \quad \equiv \quad \sum_ {\{ m_l \}}' S(\{ m_l \})
\]
donde 
\[
	S(\{ m_l \}) = \prod_{l=1}^N \left( \sum \text{ l-racimos de l partículas }\right)^{m_l}
	\frac{N!}{ 1!^{m_1} 2!^{m_2} ..,N!^{m_N} m_1! m_2! ... m_N!}
\]
siendo la productoria entre todos los l-racimos posibles de l partículas y donde el combinatorio tiene en cuenta que 
habría que permutar entre las etiquetas de las $N$ partículas (pués la sumatoria contempla l-racimos de l partículas).

\[
	S(\{ m_l \}) = \frac{N!}{ 1!^{m_1} 2!^{m_2} ..,N!^{m_N} m_1! m_2! ... m_N!} \prod_{l=1}^N
	( l! \lambda^{3(l-1)}Vb_l )^{m_l} 
\]
\[
	S(\{ m_l \}) = N! \lambda^{3N} \prod_{l=1}^N \left( \frac{Vb_l}{\lambda^3}\right)^{m_l}\frac{1}{m_l!}
\]
\[
	Z_N = \sum_ {\{ m_l \}}' S(\{ m_l \})
\]
\[
	Q_N = \frac{1}{N! \lambda^{3N}} Z_N = \sum_ {\{ m_l \}}' \prod_{l=1}^N 
	\left( 	\frac{Vb_l}{\lambda^3}\right)^{m_l}\frac{1}{m_l!}
\]
\[
	\Xi = \sum_{N=0}^\infty z^N Q_N(V,T) = \sum_{N=0}^\infty z^N \sum_ {\{ m_l \}}' \prod_{l=1}^N 
	\left( 	\frac{Vb_l}{\lambda^3}\right)^{m_l}\frac{1}{m_l!}
\]
\[
	\Xi = \sum_ {m_1=0}^\infty ... \sum_ {m_N=0}^\infty   z^N \prod_{l=1}^N 
	\left( 	\frac{Vb_l}{\lambda^3}\right)^{m_l}\frac{1}{m_l!}
\]
donde hemos utilizado los resultados
\[
	z^N = z^{\sum_1^N l m_l } = \prod_1^N (z^l)^{m_l} \qquad 
	\prod_{l=1}^N \frac{(l!)^{m_l}}{1!^{m_1}...N!^{m_l}} = 1
\]
\[
	\prod_{l=1}^N  \lambda^{3lm_l} = \lambda^{3\sum_1^N lm_l} = \lambda^{3N}
\]

\[
	\Xi = \sum_ {m_1=0}^\infty ... \sum_ {m_N=0}^\infty   z^N \prod_{l=1}^N 
	\left( 	\frac{Vb_l}{\lambda^3}\right)^{m_l}\frac{1}{m_l!} = 
	\prod_{l=1}^N  \sum_ {m_1=0}^\infty \frac{1}{m_l!} \left( \frac{z^l Vb_l}{\lambda^3}\right)^{m_l} =
	\prod_{l=1}^N  \euler^{ \frac{z^l Vb_l}{\lambda^3} } 
\]
\[
	\beta pV = \log \Xi = \sum_l \frac{z^l V b_l}{\lambda^3} = \frac{V}{\lambda^3} \sum_l z^l b_l
\]
\[
	\begin{cases}
	\beta p = \frac{1}{\lambda^3} \sum_z^l b_l \\
	\frac{N}{V} = \frac{1}{\lambda^3} \sum_l z^l b_l
	\end{cases}
\]
que es la cluster-expansion.

\subsection{Integral configuracional y $Q_N(V,T)$}

Para un hamiltoniano usual
\[
	\Ham = \sum_i^N \frac{|\vec{p}_i|^2}{2m} + \sum_{i<j} V_{ij}(q_i) = K(\{ p_i \}) + V(\{ q_i \})
\]
\[
	Q_N(V,T) = \frac{1}{h^{3N}N!}\int d^{3N}p \int d^{3N}q \euler^{-\beta \Ham(\{ p_i,q_i \})} =
	\frac{1}{h^{3N}N!}\int d^{3N}p \euler^{-\beta K(\{ p_i \})}  \int d^{3N}q \euler^{-\beta V(\{ q_i \}) } 
\]
\[
	Q_N(V,T) = \frac{1}{\lambda^{3N}N!} \int d^{3N}q \euler^{-\beta V(\{ q_i \}) }  =
	\frac{1}{\lambda^{3N}N!} \: Z_N(V,T)
\]
donde $Z_N$ es la integral configuracional
\[
	\beta p = \frac{1}{\lambda^3} \sum_l z^l b_l  \qquad \frac{1}{v} = \frac{1}{\lambda^3} \sum_l lz^l b_l 
\]
\[
	\beta p v = \frac{ \sum_l z^l b_l }{ \sum_l l z^l b_l  }
\]
y el virial es 
\[
	\sum_{l=1} a_l(T) \left( \frac{\lambda^3}{v} \right)^{l-1} = \frac{ \sum_l z^l b_l }{ \sum_l l z^l b_l  }
\]
\[
	\sum_{l=1} a_l(T) \left( \sum_l l z^l b_l \right)^{l-1} \sum_l l z^l b_l = \sum_l z^l b_l 
\]
\[
	\sum_{k=1} a_k [ zb_1 + 2z^2b_2 ]^{k-1} (zb_1+2z^2b_2) \cong zb_1 + z^2b_2
\]
\[
	a_1(zb_1+2z^2b_2) + a_2(zb_1+2z^2b_2) (zb_1+2z^2b_2) \cong zb_1 + z^2b_2
\]
\[
	za_1b_1 + 2z^2a_1b_2 + a_2z^2b_1^2 + 4a_2z^3b_1b_2 + 4 a_2 z^4 b_2^2 \cong zb_1 + z^2b_2
\]
e igualando coeficientes de $ z $ tendremos 
\[
 	a_1b_1 = b_1 \quad \rightarrow \quad a_1 = 1
\]
\[
	2a_1b_2 + a_2b_1^2 = b_2  \quad \rightarrow \quad  a_2 = -\frac{b_2}{b_1^2} = -b_2
\]





% \bibliographystyle{CBFT-apa-good}	% (uses file "apa-good.bst")
% \bibliography{CBFT.Referencias} % La base de datos bibliográfica

\end{document}
