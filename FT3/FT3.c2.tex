	\documentclass[10pt,oneside]{CBFT_book}
	% Algunos paquetes
	\usepackage{amssymb}
	\usepackage{amsmath}
	\usepackage{graphicx}
	\usepackage{libertine}
	\usepackage[bold-style=TeX]{unicode-math}
	\usepackage{lipsum}

	\usepackage{natbib}
	\setcitestyle{square}

	\usepackage{polyglossia}
	\setdefaultlanguage{spanish}
	



	\usepackage{CBFT.estilo} % Cargo la hoja de estilo

	% Tipografías
	% \setromanfont[Mapping=tex-text]{Linux Libertine O}
	% \setsansfont[Mapping=tex-text]{DejaVu Sans}
	% \setmonofont[Mapping=tex-text]{DejaVu Sans Mono}

	%===================================================================
	%	DOCUMENTO PROPIAMENTE DICHO
	%===================================================================

\begin{document}

% =================================================================================================
\chapter{Conjuntos estadísticos}
% =================================================================================================

La cantidad
\[
	\rho(\{ \vec{q}_i, \vec{p}_i\},t) d^{3N}qd^{3N}p
\]
es el número de microestados en el elemento $d^{3N}qd^{3N}p$ al tiempo $t$ centrado en $q,p$.
Si los microestados son equiprobables $\rho \equiv cte.$. El conjunto $\{ \vec{q}_i, \vec{p}_i\}$ son
$6N$ coordenadas.
\[
	\Omega = \int p d^{3N}qd^{3N}p
\]
\notamargen{La integral $\Omega$ es imposible porque es difícil determinar el volumen de integración.}

XXX Dibujos XXXX

el volumen en  $\mathbb{\Gamma}$ es proporcional al número de microestados compatibles con $E,N$,
el volumen $ \mathbb{\Gamma}$ del macroestado es $\Omega\{ n_i \}$

$n_i = f_i d^3q d^3p$ es el número de partículas en una celda $i$ (con su $\vec{p}$ en $\vec{p} + d\vec{p}$
y con su $\vec{q}$ en $\vec{q} + d\vec{q}$ )

Un microestados determina una distribución $f$ que da un conjunto $\{ n_i \}$. Pero una $f$ determina muchos
microestados porque la función de distribución no distingue entre partículas (importan los números de 
ocupación); entonces una $f$ determina un volumen en $\mathbb{\Gamma}$.
\notamargen{Cada microestado tiene su $f$.}

Suponemos que todos los microestados en $\mathbb{\Gamma}$ son igualmente probables.
La $f$ que determina el mayor volumen en  $\mathbb{\Gamma}$ es la más probable. Suponemos que en el 
equilibrio el sistema toma la $f$ más probable.
Si $f_i$ es el valor de $f$ en cada celda $i$
\[
	f_i = \frac{n_i}{d^3p d^3q} \quad \text{promediada en el ensamble} \quad \bar{f}_i =  \frac{<n_i>}{d^3p d^3q}
	\quad \text{en el equilibrio}
\]
\notamargen{$f_i$ es la distribución para un miembro en el ensamble.}

Esta $\bar{f}_i$ es la de equilibrio, pero la cuenta no es fácil. Asumiremos que la $f$ de equilibrio es la más
probable (la de mayor volumen en  $\mathbb{\Gamma}$); entonces maximizaremos dicho volumen para hallarla.

Un microestado determina una $f$; diferentes microestados pueden determinar otras $f$ pero muchos coincidirán en
una misma $f$.

La $f$ en el equilibrio es la que tiene mayor cantidad de microestados (la más probable) pero 
\[
	\bar{f}_i =  \frac{<n_i>}{d^3p d^3q}
\]
es el promedio en el ensamble y no será exactamente igual a la $f_i$ del mayor volumen, salvo que el volumen de $f$
sea mucho mayor al ocupado por $f',f''$, etc.

Dado el volumen $\Omega \{ n_i\}$ extremaremos el mismo sujeto a las condiciones
\[
	E = \sum_i^K n_i e_i \qquad \qquad N = \sum_i^K n_i
\]
y llegamos a la $f$ de equilibrio que es $f_{MB}$.
\notamargen{Necesito $\Omega = \Omega \{ n_i\}$ para obtener el $\{ \tilde{n}_i\}$.}

El volumen $\Omega$ se escribe en función de los números de ocupación
\[
	\Omega \left( \{ n_i \} \right) = 
	\frac{N!}{\prod_i^K n_i!} \prod_i^K g_i^{n_i} \qquad 
	(i=1,2,...,K \quad \text{identifica celdas en}\;\mu )
\]
\[
	\Omega \left( \{ n_i \} \right) = N! \prod_i^K \frac{g_i^{n_i}}{n_i!}
\]
donde $g_i$ son los subniveles en que podríamos dividir la celda $K$; es por matemática conveniencia y para abarcar 
más casos (luego será $g_i=1 \forall i$).

El conjunto $\{ \tilde{n}_i\}$ que extrema $\Omega \left( \{ n_i \} \right)$ es el más probable y consideraremos
\[
	\{ \tilde{n}_i\} = < n_i >
\]
Estaremos pensando que cuando $N \to \infty$ la mayor parte de los microestados van a una distribución $f_{MB}$



% =================================================================================================
% \section{Energía y entropía}
% =================================================================================================





% \bibliographystyle{CBFT-apa-good}	% (uses file "apa-good.bst")
% \bibliography{CBFT.Referencias} % La base de datos bibliográfica

\end{document}
