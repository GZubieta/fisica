	\documentclass[10pt,oneside]{CBFT_book}
	% Algunos paquetes
	\usepackage{amssymb}
	\usepackage{amsmath}
	\usepackage{graphicx}
	\usepackage{libertine}
	\usepackage[bold-style=TeX]{unicode-math}
	\usepackage{lipsum}

	\usepackage{natbib}
	\setcitestyle{square}

	\usepackage{polyglossia}
	\setdefaultlanguage{spanish}
	



	\usepackage{CBFT.estilo} % Cargo la hoja de estilo

	% Tipografías
	% \setromanfont[Mapping=tex-text]{Linux Libertine O}
	% \setsansfont[Mapping=tex-text]{DejaVu Sans}
	% \setmonofont[Mapping=tex-text]{DejaVu Sans Mono}

	%===================================================================
	%	DOCUMENTO PROPIAMENTE DICHO
	%===================================================================

\begin{document}

% =================================================================================================
\chapter{Conjuntos estadísticos}
% =================================================================================================

La cantidad
\[
	\rho(\{ \vec{q}_i, \vec{p}_i\},t) d^{3N}qd^{3N}p
\]
es el número de microestados en el elemento $d^{3N}qd^{3N}p$ al tiempo $t$ centrado en $q,p$.
Si los microestados son equiprobables $\rho \equiv cte.$. El conjunto $\{ \vec{q}_i, \vec{p}_i\}$ son
$6N$ coordenadas.
\[
	\Omega = \int p d^{3N}qd^{3N}p
\]
\notamargen{La integral $\Omega$ es imposible porque es difícil determinar el volumen de integración.}

XXX Dibujos XXXX

el volumen en  $\mathbb{\Gamma}$ es proporcional al número de microestados compatibles con $E,N$,
el volumen $ \mathbb{\Gamma}$ del macroestado es $\Omega\{ n_i \}$

$n_i = f_i d^3q d^3p$ es el número de partículas en una celda $i$ (con su $\vec{p}$ en $\vec{p} + d\vec{p}$
y con su $\vec{q}$ en $\vec{q} + d\vec{q}$ )

Un microestados determina una distribución $f$ que da un conjunto $\{ n_i \}$. Pero una $f$ determina muchos
microestados porque la función de distribución no distingue entre partículas (importan los números de 
ocupación); entonces una $f$ determina un volumen en $\mathbb{\Gamma}$.
\notamargen{Cada microestado tiene su $f$.}

Suponemos que todos los microestados en $\mathbb{\Gamma}$ son igualmente probables.
La $f$ que determina el mayor volumen en  $\mathbb{\Gamma}$ es la más probable. Suponemos que en el 
equilibrio el sistema toma la $f$ más probable.
Si $f_i$ es el valor de $f$ en cada celda $i$
\[
	f_i = \frac{n_i}{d^3p d^3q} \quad \text{promediada en el ensamble} \quad \bar{f}_i =  \frac{<n_i>}{d^3p d^3q}
	\quad \text{en el equilibrio}
\]
\notamargen{$f_i$ es la distribución para un miembro en el ensamble.}

Esta $\bar{f}_i$ es la de equilibrio, pero la cuenta no es fácil. Asumiremos que la $f$ de equilibrio es la más
probable (la de mayor volumen en  $\mathbb{\Gamma}$); entonces maximizaremos dicho volumen para hallarla.

Un microestado determina una $f$; diferentes microestados pueden determinar otras $f$ pero muchos coincidirán en
una misma $f$.

La $f$ en el equilibrio es la que tiene mayor cantidad de microestados (la más probable) pero 
\[
	\bar{f}_i =  \frac{<n_i>}{d^3p d^3q}
\]
es el promedio en el ensamble y no será exactamente igual a la $f_i$ del mayor volumen, salvo que el volumen de $f$
sea mucho mayor al ocupado por $f',f''$, etc.

Dado el volumen $\Omega \{ n_i\}$ extremaremos el mismo sujeto a las condiciones
\[
	E = \sum_i^K n_i e_i \qquad \qquad N = \sum_i^K n_i
\]
y llegamos a la $f$ de equilibrio que es $f_{MB}$.
\notamargen{Necesito $\Omega = \Omega \{ n_i\}$ para obtener el $\{ \tilde{n}_i\}$.}

El volumen $\Omega$ se escribe en función de los números de ocupación
\[
	\Omega \left( \{ n_i \} \right) = 
	\frac{N!}{\prod_i^K n_i!} \prod_i^K g_i^{n_i} \qquad 
	(i=1,2,...,K \quad \text{identifica celdas en}\;\mu )
\]
\[
	\Omega \left( \{ n_i \} \right) = N! \prod_i^K \frac{g_i^{n_i}}{n_i!}
\]
donde $g_i$ son los subniveles en que podríamos dividir la celda $K$; es por matemática conveniencia y para abarcar 
más casos (luego será $g_i=1 \forall i$).

El conjunto $\{ \tilde{n}_i\}$ que extrema $\Omega \left( \{ n_i \} \right)$ es el más probable y consideraremos
\[
	\{ \tilde{n}_i\} = < n_i >
\]
Estaremos pensando que cuando $N \to \infty$ la mayor parte de los microestados van a una distribución $f_{MB}$


% =================================================================================================
\section{Microcanónico}
% =================================================================================================

\section{Solución de equilibrio}

La solución de equilibrio satisfacía
\[
	f(p_1) f(p_2) = f(p_1') f(p_2')
\]
\[
	\log f(p_1) + \log f(p_2) = \log f(p_1') + \log f(p_2')
\]
que luce como una ley de conservación y admite como solución
\[
	\log f(p) = A m + \vb{B}\cdot \vb{p} + C|\vb{p}|^2 
	\qquad (A,\vb{B},C \text{ctes. adimensionales})
\]
que lista los {\it invariantes colisionales}. Completando cuadrados
\[
	f \propto C_1 \euler^{-C_2(\vb{p} - \vb{p}_0)^2}
\]

La expresión completa se ajusta con 
\[
	n = \int f(\vb{p},t) d^3p
\]
donde el $\vb{p}$ de una partícula es
\[
	<\vb{p}> = \frac{\int f(\vb{p}) \vb{p} \; d^3p d^3q}{\int f(\vb{p}) \; d^3p d^3q } = 
	\frac{1}{n} \int f(\vb{p}) \; \vb{p} \; d^3p
\]
\notamargen{El cociente es $\vb{P}/N$.}
y la energía por partícula
\[
	<e> = \frac{\int f(\vb{p}) \; \vb{p}^2/ (2m) \; d^3p d^3q}{\int f(\vb{p}) d^3p d^3q } = 
	\frac{1}{n} \int f(\vb{p}) \frac{ \vb{p}^2 }{ 2m } \; d^3p
\]

Finalmente se llega a 
\[
	f(\vb{p}) = \frac{n}{(2\pi m k T)^{3/2}} \euler^{- \frac{( \vb{p} - \vb{p}_0 )^2}{2mkT} }
\]

que es la función de distribución de momentos de Maxwell-Boltzmann.
\notamargen{Solución de equilibrio de la ecuación de transporte}

\[
	\text{(presión ideal)} \qquad p = \frac{2}{3} \frac{U}{V} = \frac{2}{3} n \epsilon =
	\frac{2}{3} n \frac{3}{2} k T = nkT 
\]

\section{Método de la distribución más probable}

Con este método también llegamos a $f_{MB}$ pero extremandolo el volumen $\Omega(\{ n_i \})$ que ocupa en el espacio
$\mathbb{\Gamma}$ sujeto a los vínculos $E = \sum_i n_i e_i$ y $N = \sum_i n_i $.

Luego podemos estimar qué tan probable es la distribución de MB (la más probable) considerando
(ASUMIMOS)
\[
	\text{los \# de ocupación de MB} \quad \tilde{n}_i \cong <n_i> \quad \text{el promedio en el ensamble}
\]
pero esto sólo valdrá si las desviaciones son pequeñas; es decir si $f_{MB}$ es muy muy probable.

Calculamos la desviación cuadrática (varianza) se tiene 
\[
	<n_i^2> - <n_i>^2 = g_i \dpar{<n_i>}{g_i}
\]
donde se usó que 
\[
	<n_i> = \frac{\sum_{\{ n_j\}} n_i \Omega\{ n_j\} }{\sum_{\{ n_j\}} \Omega\{ n_j\}}
\]

Suponiendo que $ <n_i> \approx \tilde{n}_i$ entonces $ <n_i>  \propto f_{MB}$ con lo cual se tiene también 
\[
	<n_i^2> - <n_i>^2 \cong \tilde{n}_i
\]
\notamargen{como $ g_i \dpar{\tilde{n}_i}{g_i} = \tilde{n}_i $}
y las fluctuaciones relativas
\[
	\sqrt{<\left( \frac{m_i}{N}\right)^2 > - <\left( \frac{m_i}{N}\right) >^2 } \cong 
	\sqrt{ \frac{ \tilde{n}_i/N }{N} }\to_{N\to\infty} 0
\]

En el límite termodinámico MB es totalmente dominante.

\subsection{Hipótesis ergódica}

La trayectoria individual de casi cualquier punto en el $\Omega$ pasa, con el tiempo, a través de todos los
puntos permitidos del espacio $\mathbb{\Gamma}$. Si esperamos lo suficiente, todos los microestados posibles
son visitados.

\subsection{Observaciones sobre el microcanónico}

\[
	\Gamma(E) = \int_{E < \Ham < E + \Delta E} \rho d^{3n}p d^{3n}q \qquad 
	\Sigma(E) = \int_{\Ham < E} \rho d^{3n}p d^{3n}q
\]
entonces 
\[
	\Gamma(E) = \Sigma( E + \Delta E ) - \Sigma( E ) \cong \dpar{\Sigma( E )}{E}\Delta E  
	\qquad \text{si}\; \Delta E \ll E
\]
$\Delta E$ es el {\it paso} entre medidas de energía 
\[
	\Gamma(E) = \Gamma_1(E_1) \Gamma_2(E_2) \qquad \text{(1 y 2 son subsistemas)}
\]
\[
	E = E_1 + E_2 \Rightarrow \Gamma(E) = \sum_i^{E/\Delta E} \Gamma_1(E_i)\Gamma_2(E-E_i)
\]
siendo $E/\Delta E$ el número de términos tales que se cumple $ E = E_1 + E_2 $.
Si se da $ N_1 \to \infty $ y $ N_2 \to \infty $ será
\[
	\log \Gamma_1 \propto N_1 \quad \log \Gamma_2 \propto N_2 \quad E \propto N_1 + N_2
\]
luego $\log(E/\Delta E)$ es despreciable pues $\Delta E$ es constante y entonces
\notamargen{$\log(E/\Delta E) \propto \log(N)$ pues $E\propto N$ y $\Delta E$ cte.}
\[
	S(E,V) = S(\tilde{E}_1,V_1) + S(\tilde{E}_2,V_2) + \mathcal{O}(\log[N])
\]
con lo cual la mayoría de los microestados tienen los valores $\tilde{E}_1$ y $\tilde{E}_2$ de energía.

Asimismo
\[
	\delta( \Gamma_1(\bar{E}_1)  \Gamma_2(\bar{E}_2) ) = 0 \qquad \delta( \bar{E}_1 + \bar{E}_2 ) = 0
\]
\[
	\delta\Gamma_1 \Gamma_2 + \Gamma_1 \delta \Gamma_2 = 0 \quad \delta( \bar{E}_1 ) = -\delta ( \bar{E}_2 )
\]
\[
	\frac{\delta\Gamma_1}{\bar{E}_1}\Gamma_2 = \Gamma_1\frac{\delta\Gamma_2}{\bar{E}_2} \Rightarrow 
	\frac{1}{\Gamma_1} \dpar{\Gamma_1}{\bar{E}_1} = \frac{1}{\Gamma_2} \dpar{\Gamma_2}{\bar{E}_2} 
\]
\[
	\dpar{}{\bar{E}_1}\left( k\log \Gamma_1(\bar{E}_1) \right) = 
	\dpar{}{\bar{E}_2}\left( k\log \Gamma_1(\bar{E}_2) \right)
\]
\[
	\left. \dpar{}{{E}_1}S(E_1)\right|_{\bar{E}_1} = \left. \dpar{}{{E}_2}S(E_2)\right|_{\bar{E}_2}
	\equiv \frac{1}{T} \qquad \text{en equilibrio} \; T_1 = T_2
\]

La $T$ es el parámetro que gobierna el equilibrio entre partes del sistema.

La idea es que dado un sistema de $E = E_1 + E_2$, sistema compuesto de dos subsistemas, hay muchos valores
1,2 tales que $E = E_1 + E_2$ pero hay una combinación que maximiza $\Gamma(E)$ y es
\[
	\Gamma_{Max}(E) = \Gamma_1(\bar{E}_1)  \Gamma_2(\bar{E}_2) 
\]
\notamargen{El sistema es $E,N,V$ y yo lo pienso compuesto de dos partes $E_1,N_1,V_1$ y $E_2,N_2,V_2$.}

Luego, con $N_1, N_2 \to \infty$ se da que la mayoría de los sistemas tendrán $E_1=\bar{E}_1$ y $E_2=\bar{E}_2$.
Esa configuración, por supuesto, maximiza la entropía $S=k\log(\Gamma)$.

El hecho de que $\Delta S> 0$ para un sistema aislado lo vemos considerando que tal sistema sólo puede variar
$V$ (creciendo, como en la expansión libre de un gas), luego $V_F > V_I$ y entonces

% \bibliographystyle{CBFT-apa-good}	% (uses file "apa-good.bst")
% \bibliography{CBFT.Referencias} % La base de datos bibliográfica

\end{document}
