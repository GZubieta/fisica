	\documentclass[10pt,oneside]{CBFT_book}
	% Algunos paquetes
	\usepackage{amssymb}
	\usepackage{amsmath}
	\usepackage{graphicx}
	\usepackage{libertine}
	\usepackage[bold-style=TeX]{unicode-math}
	\usepackage{lipsum}

	\usepackage{natbib}
	\setcitestyle{square}

	\usepackage{polyglossia}
	\setdefaultlanguage{spanish}
	



	\usepackage{CBFT.estilo} % Cargo la hoja de estilo

	% Tipografías
	% \setromanfont[Mapping=tex-text]{Linux Libertine O}
	% \setsansfont[Mapping=tex-text]{DejaVu Sans}
	% \setmonofont[Mapping=tex-text]{DejaVu Sans Mono}

	%===================================================================
	%	DOCUMENTO PROPIAMENTE DICHO
	%===================================================================

\begin{document}

% =================================================================================================
\chapter{Evolución temporal de sistemas macroscópicos}
% =================================================================================================


% =================================================================================================
 \section{Teorema de Liouville}
% =================================================================================================

Un sistema de $N$ partículas en el espacio físico $3D$ descripto por 
\[
	\Ham = \Ham(\{ p_i, q_i\},t) \qquad 1 \leq i \leq 3N
\]
evolucionará de acuerdo a
\[
	\dot{p}_i = -\dpar{\Ham}{q_i} \qquad \dot{q}_i = \dpar{\Ham}{p_i}
\]

Entonces se tendrá que 
\[
	\dpar{\rho}{t} = 0 \qquad \dpar{\rho}{t} + 
	\sum_i^{3N} \left[ \dpar{\rho}{p}\dpar{p}{t} + \dpar{\rho}{q}\dpar{q}{t} \right]
\]
\notamargen{$\rho = \rho(\{ p_i, q_i\},t)$ describe un ensamble}

Pero el número de estados se conserva. Sea $\omega$ un volumen arbitrario, el número de estados en
$\omega$ es
\[
	\Omega_\omega = \int \rho d^{3N}q d^{3N}p \equiv \int_\omega \rho d\omega
\]
\notamargen{Los estados que se fugan van a parar a otros $\omega$ dentro del ensamble}
y entonces si hay una variación es porque se fugan estados de $\omega$ y 
\[
	-\dpar{}{t} \left( \Omega_\omega \right) = \int_{S=\partial \omega} \rho \vb{v} \cdot d\vb{S}
\]
siendo el rhs el flujo saliente de estados del volumen $\omega$ huyendo por la superficie $S$ y
siendo $\vb{v} \equiv ( \dot{q}_1, \dot{q}_2, ..., \dot{q}_{3N}, \dot{p}_1, \dot{p}_2, ..., \dot{p}_{3N} )$.
Aplicando teorema de la divergencia,
\[
	-\dpar{}{t} \int_\omega \rho d\omega = \int_\omega \mathrm{div}(\rho\vb{v}) d\omega
\]
\[
	\int_\omega \left[ \dpar{\rho}{t} + \mathrm{div}(\rho\vb{v}) \right] d\omega = 0
\]
\[
	\dpar{\rho}{t} +  \sum_i^{3N} \dpar{}{q_i}(\rho\dot{q}_i) + \dpar{}{p_i}(\rho\dot{p}_i) = 0
\]
\[
	\dpar{\rho}{t} +  \sum_i^{3N} \dpar{\rho}{q_i} \dot{q}_i + \rho\dpar{\dot{q}_i}{q_i}  +
	\dpar{\rho}{p_i} \dot{p}_i + \rho\dpar{\dot{p}_i}{p_i} = 0
\]
y vemos que se tiene un cero en
\[
	\rho \left( \dparcru{\Ham}{p_i}{q_i} - \dparcru{\Ham}{q_i}{p_i} \right) = 0 
\]
\[
	\dpar{\rho}{t} +  \sum_i^{3N} \dpar{\rho}{q_i} \dot{q}_i + \dpar{\rho}{p_i} \dot{p}_i = 0
\]

El ensamble evoluciona como un fluido incompresible, pues el volumen se conserva.

% =================================================================================================
 \section{Jerarquía BBGKY}
% =================================================================================================

Podemos definir funciones de correlación $f_s$. Las ecuaciones de movimiento para calcularlas resultan
acopladas de modo que relacionan $f_1$ con $f_2$, $f_2$ con $f_3$, etc.

Este sistema es la jerarquía BBGKY. Truncándola se puede llegar a Boltzmann
\[
	z_i \equiv ( \vec{p}_i ,\vec{q}_i ) \quad \text{con} \; i = 1,2,...,N
\]
\notamargen{$f_s:$ probabilidad de hallar $s$ partículas con ciertos $\{ p_i, q_i\} \; (i=1,...,s)$}
\[
	1 = \int \rho(z_1, z_2, ... ,z_N) dz_1 ... dz_N \quad \text{normalizada}
\]
\[
	f_s = \int dz_{s+1} ... dz_N \rho(z_1, z_2, ... ,z_N) \Rightarrow f_s = f_s(z_1, z_2, ... ,z_s)
\]
\notamargen{Es una matnera de pasar de $\mathbb{\Gamma}$ a $\mu$}

Dadas $(N-s)$ partículas con cualesquiera $\vec{p},\vec{q}$ consideramos la probabilidad de tener $s$
partículas con ciertos  $\vec{p},\vec{q}$
\[
	f_1 = f_1(z_1) \quad \text{es la función de distribución}
\]
Se reescribe Liouville $\partial{\rho}/\partial {t} = 0$ con $\rho = \rho( \{ p_i, q_i\},t )$
\[
	\dpar{\rho}{t} +  \sum_{i=1}^{3N} \dpar{\rho}{q_i} \dot{q}_i + \dpar{\rho}{p_i} \dot{p}_i = 0
\]
\[
	\dpar{\rho}{q_i} \dpar{\Ham}{p_i} - \dpar{\rho}{p_i} \dpar{\Ham}{q_i} = 0
\]
\[
	\dpar{\rho}{t} + \sum_{i=1}^{3N} \left[ \nabla_{\vec{q}_i}\rho \cdot \nabla_{\vec{p}_i}\Ham -
	\nabla_{\vec{p}_i}\rho \cdot \nabla_{\vec{q}_i}\Ham \right] = 0 \qquad \text{con un $\Ham$ generico}
\]
\[
	\Ham = \sum_i^N \frac{|\vec{p}_i|^2}{2m} + \sum_i^N U_i(q_i)+ \sum_{i<j}^N V_{ij}(q_i)
\]
y tomándole el gradiente
\[
	\nabla_{\vec{p}_k}\Ham = \frac{|\vec{p}_k|\hat{k}}{m} = \frac{\vec{p}_k}{m},\qquad
	\nabla_{\vec{q}_k}\Ham = \nabla_{\vec{q}_k}U_k + \sum_{i<j}^N \nabla_{\vec{q}_k} V_{kj}
% 	\sum_i^N \frac{|\vec{p}_i|^2}{2m} + \sum_i^N U_i(q_i)+ \sum_{i<j}^N V_{ij}(q_i)
\]
\[
				, \nabla_{\vec{q}_k}\Ham = -\vec{F}_k - \sum_{i<j}^N \vec{K}_{kj}
\]
\[
	\dpar{\rho}{t} + \frac{\vec{p}_i}{2m} \cdot \nabla_{\vec{q}_i}\rho + \vec{F}_i \cdot \nabla_{\vec{p}_i}\rho + 
	\sum_{i<j}^N \vec{K}_{kj}\cdot \nabla_{\vec{p}_i} \rho = 0
\]
\[
	\left[ \dpar{}{t} + \sum_i^N \frac{\vec{p}_i}{2m} \cdot \nabla_{\vec{q}_i} + 
	\vec{F}_i \cdot \nabla_{\vec{p}_i} + \sum_{i\neq j}^N \frac{1}{2}\vec{K}_{kj}\cdot \left( \nabla_{\vec{p}_i} -
	\nabla_{\vec{p}_j} \right) \right] \rho = 0
\]
\[
	\left[ \dpar{}{t} + \underbrace{\sum_i^N S_i + \frac{1}{2} \sum_i^N\sum_j^N {i\neq j} P_{ij}}_{\equiv
	h_N(1,2,...,N)} \right] \rho = 0
\]
\[
	\left[ \dpar{}{t} + h_N(1,2,...,N) \right] \rho = 0
\]
\[
	\left[ 
	\dpar{}{t} + \sum_i^S S_i + \sum_{i=S+1}^N S_i + \frac{1}{2} \sum_i^S\sum_j^S {i\neq j} P_{ij} 
	+ \frac{1}{2} \sum_{i=S+1}^N\sum_{j=S+1}^N {i\neq j} P_{ij}
	\right] \rho = 0
\]
\[
	\left[ 
	\dpar{}{t} + h_S(1,2,...,S)  + h_{N-S}(S+1,...,N)  + \frac{1}{2} \sum_{i=1}^S\sum_{j=S+1}^N {i\neq j} P_{ij}
	\right] \rho = 0
\]
Ahora
\[
	f_s(1,2,...,S) = \frac{N!}{(N-S)!} \int dz_{S+1} ... dz_N
\]

% \bibliographystyle{CBFT-apa-good}	% (uses file "apa-good.bst")
% \bibliography{CBFT.Referencias} % La base de datos bibliográfica

\end{document}
