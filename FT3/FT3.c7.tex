	\documentclass[10pt,oneside]{CBFT_book}
	% Algunos paquetes
	\usepackage{amssymb}
	\usepackage{amsmath}
	\usepackage{graphicx}
	\usepackage{libertine}
	\usepackage[bold-style=TeX]{unicode-math}
	\usepackage{lipsum}

	\usepackage{natbib}
	\setcitestyle{square}

	\usepackage{polyglossia}
	\setdefaultlanguage{spanish}
	



	\usepackage{CBFT.estilo} % Cargo la hoja de estilo

	% Tipografías
	% \setromanfont[Mapping=tex-text]{Linux Libertine O}
	% \setsansfont[Mapping=tex-text]{DejaVu Sans}
	% \setmonofont[Mapping=tex-text]{DejaVu Sans Mono}

	%===================================================================
	%	DOCUMENTO PROPIAMENTE DICHO
	%===================================================================

\begin{document}

% =================================================================================================
\chapter{Elementos de la teoría de fenómenos críticos}
% =================================================================================================


% =================================================================================================
\section{Ising 1}
% =================================================================================================

\begin{itemize}
 \item Modelo sencillo de sistema interactuante
 \item Magnetización espontánea 1D y 2D:
 \subitem En 1D no hay magnetización espontánea
 \subitem En 2D hay magnetización espontánea
\end{itemize}

Fase es una porción de materia física y químicamente homogénea (asociada a la densidad atómica o molecular
uniforme) que no puede separarse por medios mecánicos.

Una fase puede ser una única sustancia o una mezcla.

El concepto de fase está también relacionado con el pasaj de la materia de una a otra fase.

\notamargen{corregir}

Estados de agregación (en función de la proximidad de sus componentes). Agua y aceite (líquido) es un sistema de dos 
fases.

La materia puede encontrarse en gran variedad de fases; las más conocidas están relacionadas con los estados de 
agregación. Pero dentro del estado sólido tenemos fases dependiendo de cómo sea la estructura interna.

Tenemos también sistemas que manifiestan fases ordenadas y desordenadas; aleaciones de sólidos, superconductividad.

Transición de fase: cuando una propiedad del sistema cambia discontinuamente frente a la variación de un parámetro 
intensivo ($T, p$ campo magnético).

\[
	\text{ interacciones entre partículas } \qquad \rightarrow \qquad \text{ CORRELACIÓN A GRAN ESCALA }  
\]

Las transiciones de fase emergen de la interacción. Uno de los modelos más sencillos fuera del gas ideal es el modelo de
Ising (red con interacción entre primeros vecinos)
\[
	E_\nu =
	\label{energia_ising}
\]
\notamargen{Ising es energía dada por \eqref{energia_ising} e interacción a primeros vecinos.}

dibujo 

donde $\nu$ es una dada configuración de la red (valores $S_i$ con $i=1,2,...,N$)
\[
	=
\]
donde $\mu>0$, $J$ es constante de acoplamiento y $\sum_{<i,j>}$ se extiende sobre los pares de vecinos (primeros).

Con $J>0$ es favorable que todos los spines se hallen alineados. Entonces esto llevará a la magnetización espontánea: 
fenómeno de cooperación; la mayoría de los spines se orienta en una dirección y dan un valor de magnetización 
$\braket{M} \neq 0$
\[
	\text{ (Magnetización) }
\]
\notamargen{$J>0$ ferromagnetismo y $J<0$ paramagnetismo}

Si los spines están orientados al azar, entonces habrá igual cantidad de $+1$ que de $-1$ y entonces
\[
	M \approx 0
\]

Si $H=0$ entonces $M$ es la magnetización espontánea.
\notamargen{$M$ se define como un momento dipolar magnético por unidad de volumen.}

Habrá magnetización con $T$ baja (o $J$ alto) y hasta una $T_\text{curie}$
\[
	Q_N(H,T) = \sum_{s_1=-1}^{+1} 
\]
donde las sumatorias toman para cada $i$ los valores $S_i = +1, -1$
\[
	A = -kT\log Q \qquad \braket{E} = -\dpar{}{\beta}\log Q = kT^2 \dpar{}{T} \log Q
\]

Como es
\[
	E_\nu =
\]
\[
	=
\]
\[
	\braket{M} = 
\]
\[
	\braket{M} = 
\]

\subsection{No hay magnetización espontánea en 1D}

DIBUJO 

\[
	=
\]

Varían los términos asociados a la pared
\[
	=
\]

La variación de $S$ está asociada con el número de formas de ubicar la pared
\[
	S = l \log (N-1)
\]
y es la $S$ del estado con una pared, el desordenado.
\[
	\Delta S = k \log(N-1)	\qquad (S_0 \equiv 0)
\]
que define al estado sin pared como de entropía $S_0=0$
\[
	A = U - TS \quad \rightarrow \quad \delta A = \delta U - T \delta S
\]
\[
	\delta J - kT \log(N-1)
\]
\notamargen{Para $p$ paredes es 
$\Delta A = 2Jp - kT \log [(N-1)(N-2)...(N-p)]$}

Con $T > 0$ tenemos que si desordeno (agrego paredes) sube $U$ y sube $S$.
En general, como 
\[
	\frac{\delta A}{kT} = \frac{2J}{kT} - \log(N-1)
\]
vemos que para $N\to\infty$ $\delta A < 0$ a menos de que $J/kT$ sea muy grande.

\notamargen{$S$ domina la minimización de $A$.}

En un sistema macroscópico 1D el desorden baja la $A$, entonces el equilibrio tiende
al desorden (no al orden).

Es decir, un sistema 1D de spines a $ T \neq 0 $ espontáneamente irá hacia $ A $
mínimas (mayor aleatoriedad), no se tiende a alcanzar estados ordenados.

\subsection{Magnetización espontánea en 2D}

La magnetización media por spín es
\[
	\mathcal{M} = \frac{1}{2} \Frac{N_+ - N_-}{N}
\]
Con $N\to\infty$ claramente será 0 a no ser que exista una preferencia por cierta dirección $+$ o $-$.

Queremos calcular todas las configuraciones posibles de un arreglo 2D de spines.
Para ello sistematizamos una dada construcción en dominios $\Box$ que engloban spines ($-$) y están
limitados por paredes.

DIBUJO ising


Los spines $+$ son una condición de contorno que con $N\to\infty$ es una perturbación que rompe la
simetría. También sirven para cerrar los dominios.

Cada dominio tiene una longitud $b$ medido en paredes $|$ y una dirección de recorrido de forma que 
los spines $-$ están siempre a la izquierda de la pared.
El tamaño de la red es $\sqrt{N} \times \sqrt{N} = N$. El área se mide en términos del dominio 
mínimo ``$\Box$''
\[
	\text{ dominio } = (b,i)
\]
donde $b$ es el número de paredes e $i$ una etiqueta.

A un mismo número de paredes según forma y localización tendrá varios dominios.

Una dada configuración del sistema tendrá ciertos dominios $(b,i)$

\begin{center}
\begin{tabular}{llll}
 & $b$ (paredes) & Areas (spines) & $b^2/16$ \\
\hline
 & 4 & 1 & 1\\
 & 6 & 2 & 2.25\\
 & 8 & 3,4 & 4
\end{tabular}
\end{center}

Si cada spin ocupa un área de 1, en términos de paredes el área que engloba un dominio de $b$ paredes
es 
\[
	\text{ Área dominio } \neq \frac{b^2}{16} \qquad \rightarrow \qquad S([b,i]) = \text{ Área dominio}
\]
\notamargen{Tengo una figura de longitud $b$ y si la quiero llevar a un cuadrado con suerte el lado
será $b/4$ de modo que su área es $b^2/16$}

Definimos ahora 
\[
	\chi([b,i]) = \begin{cases}
	              1 \qquad \text{ Si (b,i) ocurre en una dada configuración } \\
	              0 \qquad \text{ En caso contrario}
	             \end{cases}
\]
y $m(b)$ número de dominios de $b$ paredes.

Luego;
\[
	\boxed{ N_- = \sum_b \sum_i^{m(b)} \chi([b,i]) S([b,i]) } \quad [1]
\]
en el caso dibujado sería
\[
	N_- = 1 \cdot S(6,i) + 1 \cdot S(8,i') + 1 \cdot S(26,i'') \qquad 
	N_- = 1 \cdot  2 + 1 \cdot  4 + 1 \cdot 12 = 18
\]

Por la [1] se puede acotar, empezando por $m(b)$. Para ver el número de dominios de longitud $b$ piénsese que 
para la primera pared tengo $N$ posibilidades; para las siguientes $b-1$ tengo tres opciones pues no puedo volver,
y entonces 
\[
	m(b) \leq N 3^{b-1}
\]

Nótese que estamos considerando paredes abiertas y cerradas.









% =================================================================================================
\section{Ising 2}
% =================================================================================================



% \bibliographystyle{CBFT-apa-good}	% (uses file "apa-good.bst")
% \bibliography{CBFT.Referencias} % La base de datos bibliográfica

\end{document}
