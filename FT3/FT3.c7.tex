	\documentclass[10pt,oneside]{CBFT_book}
	% Algunos paquetes
	\usepackage{amssymb}
	\usepackage{amsmath}
	\usepackage{graphicx}
	\usepackage{libertine}
	\usepackage[bold-style=TeX]{unicode-math}
	\usepackage{lipsum}

	\usepackage{natbib}
	\setcitestyle{square}

	\usepackage{polyglossia}
	\setdefaultlanguage{spanish}
	



	\usepackage{CBFT.estilo} % Cargo la hoja de estilo

	% Tipografías
	% \setromanfont[Mapping=tex-text]{Linux Libertine O}
	% \setsansfont[Mapping=tex-text]{DejaVu Sans}
	% \setmonofont[Mapping=tex-text]{DejaVu Sans Mono}

	%===================================================================
	%	DOCUMENTO PROPIAMENTE DICHO
	%===================================================================

\begin{document}

% =================================================================================================
\chapter{Elementos de la teoría de fenómenos críticos}
% =================================================================================================


% =================================================================================================
\section{Ising 1}
% =================================================================================================

\begin{itemize}
 \item Modelo sencillo de sistema interactuante
 \item Magnetización espontánea 1D y 2D:
 \subitem En 1D no hay magnetización espontánea
 \subitem En 2D hay magnetización espontánea
\end{itemize}

Fase es una porción de materia física y químicamente homogénea (asociada a la densidad atómica o molecular
uniforme) que no puede separarse por medios mecánicos.

Una fase puede ser una única sustancia o una mezcla.

El concepto de fase está también relacionado con el pasaje de la materia de una a otra fase.

\notamargen{corregir}

Estados de agregación (en función de la proximidad de sus componentes). Agua y aceite (líquido) es un sistema de dos 
fases.

La materia puede encontrarse en gran variedad de fases; las más conocidas están relacionadas con los estados de 
agregación. Pero dentro del estado sólido tenemos fases dependiendo de cómo sea la estructura interna.

Tenemos también sistemas que manifiestan fases ordenadas y desordenadas; aleaciones de sólidos, superconductividad.

Transición de fase: cuando una propiedad del sistema cambia discontinuamente frente a la variación de un parámetro 
intensivo ($T, p$ campo magnético).

\[
	\text{ interacciones entre partículas } \qquad \rightarrow \qquad \text{ CORRELACIÓN A GRAN ESCALA }  
\]

Las transiciones de fase emergen de la interacción. Uno de los modelos más sencillos fuera del gas ideal es el modelo de
Ising (red con interacción entre primeros vecinos)
\[
	\boxed{ E_\nu = -H \sum_{i=1}^N (\mu S_i) - J \sum_{\braket{i,j}} S_i \cdot S_j }
	\label{energia_ising}
\]
\notamargen{Ising es energía dada por \eqref{energia_ising} e interacción a primeros vecinos.}

dibujo 


donde $\nu$ es una dada configuración de la red (valores $S_i$ con $i=1,2,...,N$)
\[
	S_i = \pm 1 \qquad \rightarrow \qquad \pm \mu S_i \text{ Momento magnético del spin i-ésimo }
\]
donde $\mu>0$, $J$ es constante de acoplamiento y $\sum_{<i,j>}$ se extiende sobre los pares de vecinos (primeros).

Con $J>0$ es favorable que todos los spines se hallen alineados. Entonces esto llevará a la magnetización espontánea: 
fenómeno de cooperación; la mayoría de los spines se orienta en una dirección y dan un valor de magnetización 
$\braket{M} \neq 0$
\[
	M_\nu = \sum_{i=1}^N \mu \cdot S_i^\nu
	\text{ (Magnetización) }
\]
\notamargen{$J>0$ ferromagnetismo y $J<0$ paramagnetismo}

Si los spines están orientados al azar, entonces habrá igual cantidad de $+1$ que de $-1$ y entonces
\[
	M \approx 0
\]

Si $H=0$ entonces $M$ es la magnetización espontánea.
\notamargen{$M$ se define como un momento dipolar magnético por unidad de volumen.}

Habrá magnetización con $T$ baja (o $J$ alto) y hasta una $T_\text{curie}$
\[
	Q_N(H,T) = \sum_{s_1=-1}^{+1} \; \sum_{s_2=-1}^{+1} \; ... \sum_{s_N=-1}^{+1} \;
	\euler^{ +\beta H \mu \sum_i^N S_i + \beta J \sum_{\braket{i,j}} S_i \cdot S_j }
\]
donde las sumatorias toman para cada $i$ los valores $S_i = +1, -1$
\[
	A = -kT\log Q \qquad \braket{E} = -\dpar{}{\beta}\log Q = kT^2 \dpar{}{T} \log Q
\]

\begin{ejemplillo}{(Ejercicio 5.1 Chandler)}
\[
	E_0 = -J \sum_{\braket{i,j}} S_i \cdot S_j = -J  \sum_i^N \sum_j^\gamma \frac{S_i S_j}{2}
\]
para cada $i$ sumo en sus $\gamma$ vecinos el $j$ (sobre 2 para no contar dos veces).
\[
	E_0 = -J \sum_i^N \frac{S_i\gamma}{2} = -J N\frac{\gamma}{2} = -JND
\]
donde $D$ es la dimensionalidad.
\end{ejemplillo}


Como es
\[
	E_\nu =  -H \sum_{i=1}^N \mu \cdot S_i - J \sum_{\braket{i,j}} S_i \cdot S_j \qquad
	\text{ y } \qquad \braket{M} = \braket{ \sum_i^N \mu \cdot S_i }
\]
entonces
\[
	\braket{M} = \braket{ \sum_i^N \mu \cdot S_i } = 
	\frac{ \sum_{s_1} \; \sum_{s_2} \; ... \sum_{s_N} \;
	\euler^{ \beta H \mu \sum_i^N S_i + \beta J \sum_{\braket{i,j}} S_i \cdot S_j } }{Q_N}
\]
\[
	\braket{M} = \frac{ \sum_{s_1} \; \sum_{s_2} \; ... \sum_{s_N} \;
	\dpar{}{\beta H} \left[ \euler^{ \beta H \mu \sum_i^N S_i + \beta J \sum_{\braket{i,j}} S_i \cdot S_j } 
	\right]}{Q_N} = \frac{\dpar{}{\beta H}(Q_N)}{Q_N}
\]
\[
	\braket{M} = \dpar{}{\beta H}\left( \log Q_N \right) = \dpar{}{\beta H}\left( -\beta A \right) =
	\dpar{}{H}\left( A \right)
\]

\subsection{No hay magnetización espontánea en 1D}

\notamargen{Error en Huang (14.6); es $-\dpar{}{H}(A_I)$}

DIBUJO 

Con $H=0$ invierto spines detrás de una pared.
\[
	E_0 = -J(N-1) \qquad E_f = -J[(N-1)-2p]
\]
\notamargen{$p$ es el número de paredes}

Varían los términos asociados a la pared
\[
	\Delta E = E_f - E_0 = 2Jp > 0
\]
con $p=1$ es $\Delta E = 2J$ y con $p=2$ es $\Delta E = 4J$ (es 2 por pared puesto que desaparece un $+$ y aparece
en su lugar un $-$).

La variación de $S$ está asociada con el número de formas de ubicar la pared
\[
	S = l \log (N-1)
\]
y es la $S$ del estado con una pared, el desordenado.
\[
	\Delta S = k \log(N-1)	\qquad (S_0 \equiv 0)
\]
que define al estado sin pared como de entropía $S_0=0$
\[
	A = U - TS \quad \rightarrow \quad \delta A = \delta U - T \delta S
\]
\[
	\delta J - kT \log(N-1)
\]
\notamargen{Para $p$ paredes es 
$\Delta A = 2Jp - kT \log [(N-1)(N-2)...(N-p)]$}

Con $T > 0$ tenemos que si desordeno (agrego paredes) sube $U$ y sube $S$.
En general, como 
\[
	\frac{\delta A}{kT} = \frac{2J}{kT} - \log(N-1)
\]
vemos que para $N\to\infty$ $\delta A < 0$ a menos de que $J/kT$ sea muy grande.

\notamargen{$S$ domina la minimización de $A$.}

En un sistema macroscópico 1D el desorden baja la $A$, entonces el equilibrio tiende
al desorden (no al orden).

Es decir, un sistema 1D de spines a $ T \neq 0 $ espontáneamente irá hacia $ A $
mínimas (mayor aleatoriedad), no se tiende a alcanzar estados ordenados.

\subsection{Magnetización espontánea en 2D}

La magnetización media por spín es
\[
	\mathcal{M} = \frac{1}{2} \Frac{N_+ - N_-}{N}
\]
Con $N\to\infty$ claramente será 0 a no ser que exista una preferencia por cierta dirección $+$ o $-$.

Queremos calcular todas las configuraciones posibles de un arreglo 2D de spines.
Para ello sistematizamos una dada construcción en dominios $\Box$ que engloban spines ($-$) y están
limitados por paredes.

DIBUJO ising


Los spines $+$ son una condición de contorno que con $N\to\infty$ es una perturbación que rompe la
simetría. También sirven para cerrar los dominios.

Cada dominio tiene una longitud $b$ medido en paredes $|$ y una dirección de recorrido de forma que 
los spines $-$ están siempre a la izquierda de la pared.
El tamaño de la red es $\sqrt{N} \times \sqrt{N} = N$. El área se mide en términos del dominio 
mínimo ``$\Box$''
\[
	\text{ dominio } = (b,i)
\]
donde $b$ es el número de paredes e $i$ una etiqueta.

A un mismo número de paredes según forma y localización tendrá varios dominios.

Una dada configuración del sistema tendrá ciertos dominios $(b,i)$

\begin{center}
\begin{tabular}{llll}
 & $b$ (paredes) & Areas (spines) & $b^2/16$ \\
\hline
 & 4 & 1 & 1\\
 & 6 & 2 & 2.25\\
 & 8 & 3,4 & 4
\end{tabular}
\end{center}

Si cada spin ocupa un área de 1, en términos de paredes el área que engloba un dominio de $b$ paredes
es 
\[
	\text{ Área dominio } \neq \frac{b^2}{16} \qquad \rightarrow \qquad S([b,i]) = \text{ Área dominio}
\]
\notamargen{Tengo una figura de longitud $b$ y si la quiero llevar a un cuadrado con suerte el lado
será $b/4$ de modo que su área es $b^2/16$}

Definimos ahora 
\[
	\chi([b,i]) = \begin{cases}
	              1 \qquad \text{ Si (b,i) ocurre en una dada configuración } \\
	              0 \qquad \text{ En caso contrario}
	             \end{cases}
\]
y $m(b)$ número de dominios de $b$ paredes.

Luego;
\[
	\boxed{ N_- = \sum_b \sum_i^{m(b)} \chi([b,i]) S([b,i]) } \quad [1]
\]
en el caso dibujado sería
\[
	N_- = 1 \cdot S(6,i) + 1 \cdot S(8,i') + 1 \cdot S(26,i'') \qquad 
	N_- = 1 \cdot  2 + 1 \cdot  4 + 1 \cdot 12 = 18
\]

Por la [1] se puede acotar, empezando por $m(b)$. Para ver el número de dominios de longitud $b$ piénsese que 
para la primera pared tengo $N$ posibilidades; para las siguientes $b-1$ tengo tres opciones pues no puedo volver,
y entonces 
\[
	m(b) \leq N 3^{b-1}
\]

Nótese que estamos considerando paredes abiertas y cerradas.

Luego,
\[
	\braket{N_-} \leq \sum_b \sum_i^{N3^{b-1}} \chi([b,i]) \underbrace{S([b,i])}_{\leq b^2/16}
\]
\[
	N_- \leq \sum_b \frac{b^2}{16} \sum_i^{N3^{b-1}} \chi([b,i])
\]
\[
	\braket{N_-} \leq \sum_b \frac{b^2}{16} \sum_i^{N3^{b-1}} \braket{\chi([b,i])}
\]

Pero
\[
	\braket{\chi([b,i])} = \frac{ \sum_{\{Si\}}' \euler^{-\beta E_{\{Si\}}} }
	{ \sum_{\{Si\}} \euler^{-\beta E_{\{Si\}}} }
\]
donde la sumatoria es en aquellas configuraciones que contienen al dominio $(b,i)$.

\notamargen{num: de todas las configuraciones posibles aquellas en las cuales se da el dominio $(b,i)$.
den: todas las configuraciones posibles.}

Removemos términos del denominador para acotar: pensamos que si en una dada configuración $C$ con $\{b,i\}$ revertimos 
en el dominio $\{b,i\}$ los spines llegamos a una configuración $\tilde{C}$
\[
	E_C - E_{\tilde{C}} = 2 \varepsilon b
\]

Al revertir los spines de un dominio pasamos a una configuración más ordenadas y por ende de menor energía
 
 DIBUJO
 
\[
	\frac{ \sum_{\{Si\}}' \euler^{-\beta E_{\{Si\}}} }{ \sum_{\{Si\}} \euler^{-\beta E_{\{Si\}}} }
	\leq 
	\frac{ \sum_{\{C\}} \euler^{-\beta E_C} }{ \sum_{\{C'\}} \euler^{-\beta E_{\tilde{C}}} } =
	\frac{ \sum_{\{C\}} \euler^{-\beta E_C} }{ \sum_{\{C\}} \euler^{-\beta E_{\tilde{C}}} 
	\euler^{2\beta\varepsilon b} } = \euler^{-2\beta \varepsilon b}
\] 
\[
	\braket{N_-} \leq \sum_b \frac{b^2}{16} \euler^{-2\beta \varepsilon b} N 3^{b-1} =
	\frac{N}{48} \sum_b b^2 [ 3 \euler^{-2\beta \varepsilon } ]^b
\]
\[
	\braket{N_-} \leq \frac{N}{48} \sum_{b=4,6,8,...} b^2 x^2,
\]
con $ x \equiv 3 \euler^{-2\beta \varepsilon } $

\[
	\braket{N_-} \leq \frac{N}{48} (16x^4 + 36x^6 + 64x^8 + ...) 
\]
Sea $b=2n$, entonces
\[
	\braket{N_-} \leq \frac{N}{48} \sum_{n=2,3,4,...} 4 n^2 (x^2)^n,
\]
con $x^2 = 9 \euler^{-4\beta\varepsilon} $
\[
	\braket{N_-} \leq \frac{N}{12} \sum_{n=2}^\infty n^2 r^n,
\]
con $r=9 \euler^{-4\beta\varepsilon}$
\[
	\braket{N_-} \leq \frac{N}{3} \frac{r^2}{(1-r)^3} \left[ 1 - \frac{3}{4}r + \frac{1}{4}r^2 \right]
\]
y esta cantidad para algún $\beta$ grande pero finito es menor a $N/2$.


% =================================================================================================
\section{Ising 2}
% =================================================================================================

La energía se podía escribir como
\[
	E_\nu = - H \sum_i^N (\mu s_i) - J \sum_{\braket{i,j}} s_i \cdot s_j 
\]

El grado de un nodo es $\gamma$ que depende de la red y de la dimensión,
\begin{center}
\begin{tabular}{lll}
2D & cuadrada & $\gamma=4$ \\
3D & SC & $\gamma=6$ \\
3D & BCC & $\gamma=8$
\end{tabular}
\end{center}

\notamargen{$\gamma$ es el número de vecinos. De cada nodo salen $\gamma$ líneas.}

\[ 
	\sum_{\braket{i,j}} = \frac{\gamma N}{2} = \text{ nro total de líneas } 
\]

dibujos

Tomando un nodo y trazando líneas a sus $\gamma$ vecinos tengo $ \gamma N/2 $ líneas dibujadas
(se divide en 2 por el doble conteo).

Tomando cada $\oplus$ trazo líneas a sus vecinos y defino
\[ 
	N_+ = \text{ nro de spines $\uparrow$} \qquad N_- = \text{ nro de spines $\downarrow$}
\]
\[ 
	N_{++} = \text{ nro de pares $\uparrow\uparrow$} \qquad 
	N_{--} = \text{ nro de pares $\downarrow\downarrow$}
\]
\[ 
	N_{+-} = \text{ nro de pares $\uparrow\downarrow$ o ($\downarrow\uparrow$)}
\]
\begin{itemize}
 \item 1) $ \gamma N_+ = 2N_{++} + N_{+-} $
 \item 2) $ \gamma N_- = 2N_{--} + N_{+-} $
 \item 3) $ N_+ + N_- = N $
\end{itemize}

\[
	\gamma N = 2N_{++} + 2N_{--} + 2N_{+-}
\]
\[
	\gamma N_+ + \gamma N_- = (2N_{++} + N_{+-}) + (2N_{--} + N_{+-})
\]
\[
	\frac{\gamma N}{2} = N_{++} + N_{--} + N_{+-} 
\]

Podemos poner todo en términos de $N_{++}, N_+, N$ y entonces
\[
	N_{+-} = \gamma N_+ - 2N_{++} \qquad N_- = N - N_+
\]
\[
	N_{--} = \frac{\gamma}{2}(N-N_+) - \frac{1}{2}(\gamma N_+ - 2N_{++}) =
	\frac{\gamma}{2} N - \gamma N_+ + N_{++}
\]
\begin{multline*}
	\sum_{\braket{i,j}} s_i \cdot s_j = N_{++} + N_{--} - N_{+-} =
	N_{++} + \frac{\gamma}{2}N - \gamma N_+ + N_{++} - \gamma N_+ + 2N_{++}= \\
	4N_{++} - 2\gamma N_+ + \frac{\gamma}{2}N 
\end{multline*}
\[
	s_i = N_+ - N_- = N_+ - (N-N_+) = 2N_+ - N
\]

La energía se puede escribir en función de estas variables
\[
	E_I = - H\mu (2N_+ - N) - J(4N_{++} - 2\gamma N_+ + \frac{\gamma}{2}N)
\]
\[
	E_I = -4JN_{++} - 2(H\mu - \gamma J)N_+ - \left( \frac{\gamma J}{2}- H\mu \right)N
\]

\notamargen{La energía depende de las cantidades $N,N_+,N_{++}$ y no del detalle de la distribución
de los mismos.}

La función canónica será
\[
	Q_I = \sum_{E_I} \euler^{-\beta E_I} = \euler^{\beta(J\gamma/2 - H\mu)N } \sum_{N_+=0}^N
	\left( \euler^{2\beta( \mu H - J\gamma )N_+ } \sum_{N_{++}}' \euler^{4\beta J N_{++}}
	\: g(N_+,N_{++})\right)
\]
donde $g(N_+,N_{++})$ es el número de configuraciones de $N_{++}$ y $N_{+}$ y la sumatoria primada se hace 
sobre los valores de $N_{++}$ consistentes con que hay $N_{+}$ spines up.

Esta expresión no ha sido resuelta salvo en 2D.

\subsection{Aproximación de Bragg-Williams}

\[
	\frac{N_+}{N} = \text{ (promedio) $\leftarrow$ correlaciones de largo rango } 
\]
\[
	\frac{N_{++}}{\gamma/2 N} = \text{  $\leftarrow$ correlaciones de corto rango } 
\]
y entonces $N_{+}/N$ está asociado a una visión global del sistema (un cuerpo), 
mientras que $N_{++}/(\gamma/2 N)$ lo está a una visión local del sistema (dos cuerpos).

Si un dado spin es $\oplus$ entonces tiene en promedio $N_{++}/(\gamma/2 N)$ vecinos del tipo $\oplus$.

Definimos unos parámetros de orden $L$ y $\sigma$
\[
	\frac{N_+}{N} = \frac{1}{2}(L+1) \qquad (\text{ todo } \downarrow) -1 \leq L \leq 1 (\text{ todo } \uparrow)
\]
\[
	\frac{N_{++}}{\gamma/2 N} = \frac{1}{2}(\sigma + 1) \qquad -1 \leq \sigma \leq 1
\]
\notamargen{Estamos viendo todo del lado de los spines $\oplus$.}
pero 
\[
	\sum_i s_i = 2N_+ - N = (L+1)N - N = NL,
\]
\[
	\braket{M} = \braket{ \sum_i^N \mu s_i } = \mu \braket{ \sum_i^N s_i } = \mu N \braket{L} 
\]
\[
	\frac{\braket{M}}{\mu N} \equiv \mathcal{M} = \braket{L}
\]
que es la magnetización por partícula (adimensional).
\[
	\sum_{\braket{i,j}} = \frac{1}{2} N \gamma (2\sigma - 2L + 1) 
\]

La energía es 
\[
	E = - H \mu \sum_i^N s_i - J \sum_{\braket{i,j}} =
	-H\mu NL - \frac{J}{2} N \gamma (2\sigma - 2L + 1) 
\]
y por partícula,
\[
	\epsilon \equiv \frac{E}{N} = -H\mu L - \frac{J}{2} \gamma (2\sigma - 2L + 1) 
\]

Hasta aquí el planteo es exacto; Bragg-Williams hace la aproximación
\notamargen{Significa que no hay correlaciones de orden corto salvo las que surgen
del orden largo. Me quedo sólo con el parámetro $L$.}
\[
	\frac{N_{++}}{\gamma/2 N}= \Frac{N_+}{N}^2
\]
\[
	\frac{1}{2}(\sigma + 1) \approx \frac{1}{2^2} (L+1)^2 \quad \rightarrow \quad 
	\sigma \approx \frac{L^2 + 2L + 1}{2}
\]
\[ 
	\boxed{ E = -\mu H N L - \frac{JN\gamma }{2}L^2 },
\]
que es la $E$ en Bragg-Williams.
\[
	Q(H,T) = \sum_{\{ s_i\}} \euler^{-\beta N(- \H\mu L - JL^2 \gamma / 2) }
\]
donde $\{ s_i\}$ es la configuración de los N spines.

La suma se extiende sobre todos los conjuntos $\{ s_i\}$, pero el sumando sólo depende de $L$.
Queremos saber cuántos conjuntos $\{ s_i\}$ tienen el mismo $L$,
\[
	\frac{N!}{N_+!(N-N_+)!}
\]
que es el número de maneras de tomar $N_+$ de $N$ indistinguibilizando dentro de $N_+$ y de $N_-\equiv
N-N_+$
\[
	Q(H,T) = \sum_{L=-1}^{L=1} \frac{N!}{N_+!(N-N_+)!}
	 \euler^{\beta N( \H\mu L + JL^2 \gamma / 2) }
\]

La suma es ahora en todos los $L$ posibles. Con $N\to\infty$ el logaritmo de $Q$ es dominado por el término
(con $\bar{L}$) que maximiza el sumando.
\notamargen{La clave es el término que maximiza el sumando en valor absoluto. Será máximo o mínimo. }
\[
	\log (Q(H,T)) = \log \left( \sum_L \frac{N!}{N_+!(N-N_+)!} \euler^{\beta N f(L)} \right)
\]
si pensamos que la suma está dominada por un término,
\[
	\log (Q(H,T)) \approx \log \left( \frac{N!}{ N/2(\bar{L}+1)!N/2(1-\bar{L})!} 
	\euler^{\beta N f(\bar{L})} \right)
\]
\[
	\log (Q(H,T)) \approx \log N! - \log N/2(\bar{L}+1)!N/2(1-\bar{L})! + 
	\beta N ( H\mu \bar{L} + J\bar{L}^2 \gamma/2 ) 
\]
y usando Stirling,
\[
	\log Q = \beta N H \mu \bar{L} + \beta \frac{NJ}{2} \gamma \bar{L}^2 -
	\frac{N}{2} \log \Frac{1-\bar{L}^2}{4} - \frac{N\bar{L}}{2} \log \Frac{1+\bar{L}}{1-\bar{L}}
	\qquad [1]
\]
pero no sabemos quién es $\bar{L}$. Y si hacemos
\[
	\dpar{}{\bar{L}}( \log Q[H,T] ) = 0
\]
llegamos a que debe valer [2]
\[
	\log \frac{1+\bar{L}}{1-\bar{L}} = 2\beta H\mu + 2\beta \gamma \bar{L} J
\]
y por ello el valor de $\bar{L}$ sale de
\[
	\bar{L} = \tanh( \beta H \mu + \beta \gamma J \bar{L} )
\]

Con $H=0$ es 
\[
	\boxed{ \bar{L} = \tanh( \beta \gamma J \bar{L} ) } \qquad \text{ condición para $\bar{L}$ }
\]

DIBUJO 

busco igualar $f=\tanh(\beta\gamma J\bar{L})$ con $f=\bar{L}$.

Entonces, si
\[
	T_c \equiv \frac{\gamma J}{k} > T \qquad \rightarrow \quad \bar{L} = 0, L_0, -L_0 \; 
	\text{ son soluciones } 
\]
\[
	T_c \equiv \frac{\gamma J}{k} \geq T \qquad \rightarrow \quad \bar{L} = 0 \; 
	\text{ es solución } 
\]
siendo $T_c$ la temperatura de Curie.
\notamargen{El $L$ máximo, el $\bar{L}$, es el que domina en $\log Q$.
Asimismo, como $A = -kT\log Q$, el valor que maximiza $\log Q$ también minimiza $A$.}
Usando (2) en (1) podemos escribir 
\[
	- \beta A = \log Q(H,T) \approx - \beta \frac{\gamma J N}{2} \bar{L}^2 
	- \frac{N}{2} \log \Frac{1-\bar{L}^2}{4}
\]
\[
	\log Q(H,T) \approx - \Frac{T_c}{T}\frac{N}{2}\bar{L}^2 - \frac{N}{2} \log \Frac{1-\bar{L}^2}{4}
	\; [3]
\]
pero (3) vale para el $\bar{L}$ que maximiza $\log Q$. Vemos que es independiente de $H$.
Es más, (3) graficado en función de $\bar{L}$ no me dice nada. Lo que es valioso es (1).
Desde allí,
\[
	A \approx -NH\mu \bar{L} - kT_C \frac{N}{2}\bar{L}^2  + kT\frac{N}{2} \log \Frac{1-\bar{L}^2}{4}
	+ kT \frac{N}{2} L \log \Frac{1+\bar{L}}{1-\bar{L}}
\]

Considerando $H=0$ resulta
\[
	\frac{\beta A}{N/2} \approx -\frac{T_c}{T} \bar{L}^2 + \log \Frac{1-\bar{L}^2}{4}
	+ \bar{L} \log \Frac{1+\bar{L}}{1-\bar{L}}
\]
y si 
\[
	-\frac{2H\mu}{kT}\bar{L},
\]
siendo chico el factor,

DIBUJOS

\[
	A \approx k T_c \frac{N}{2} \bar{L}^2 + k T N \log \Frac{1-\bar{L}^2}{4}
\]

El efecto del $H\neq 0$ es entonces romper la degeneración. Por otro lado $\bar{L}$ es el valor de
magnetización por partícula. Entonces podemos graficar $A(\mu)$

DIBUJO

Las otras funciones termodinámicas resultan (con $H=0$)
\[
	\frac{M}{\mu N} = \begin{cases}
	                   0 \qquad T > T_c \\
	                   L_0 \qquad T< T_c
	                  \end{cases}
\]
\[
	\frac{A}{N} = \begin{cases}
	                   0 \qquad T > T_c \\
	                   \frac{\gamma J}{2}\bar{L_0}^2 + \frac{kT}{2} \log \Frac{1-\bar{L_0}^2}{4}
	                   \qquad T< T_c
	                  \end{cases}
\]
\[
	\frac{U}{N} = \begin{cases}
	                   0 \qquad T > T_c \\
	                   -\frac{\gamma J}{2}\bar{L_0}^2 \qquad T< T_c
	                  \end{cases}
\]
\[
	\frac{C}{N} = \dpar{}{T}\Frac{U}{N} = \dpar{}{T}\left(-\frac{\gamma J}{2}\bar{L_0}^2 \right)
	= - \gamma J L_0 \dtot{L_0}{T}
\]
donde $L_0$ debe computarse numéricamente pero podemos aproximar en dos límites $T\approx 0$ y $T\approx T_c$

\[
	L_0 = \tanh\Frac{T_c L_0}{T} = \frac{(1-\euler^{-2x})}{1+\euler^{-2x}}
	\approx (1-\euler^{-2x})^2
\]
siendo $x\equiv T_cL_0/T$ y amasando tenemos
\[
	\begin{cases}
	L_0 \approx 1 - 2\euler^{-T_cL_0/T} \quad \text{ si } T_c/T \gg 1 \\
	L_0 \approx 3^{1/2}\left( 1 - \frac{T}{T_c} \right)^{1/2} \quad \text{ si } T_c \approx T 
	\end{cases}
\]	


DIBUJOS

\subsection{Aproximación de Bette-Peierls}

Tiene en cuenta correlaciones de corto orden. Se piensa en un elemento fundamental de la red de spines y el
efecto de toda la red sobre el mismo.
\[
	z \equiv \text{ parámetro que mide el efecto de la red sobre el elemento }
\]

dibujete

$P(s,n)$ es la probabilidad de que el spin central tenga valor 's' y halla 'n' vecinos $\oplus$
de manera que 
\[
	P(+,n) \quad \rightarrow \qquad n \text{ pares } ++, \quad \gamma_- n \text{ pares } +- 
\]
\[
	P(-,n) \quad \rightarrow \qquad n \text{ pares } +-, \quad \gamma_- n \text{ pares } -- 
\]

Para un dado $n$ hay $(\gamma n)$ [combinatorio] posibles ordenamientos.
Se propone:
\[
	P(+,n) = \frac{1}{q}\binom{\gamma}{n} \euler^{\beta J (2n-\gamma)} z^n
\]
\[
	P(-,n) = \frac{1}{q}\binom{\gamma}{n} \euler^{\beta J (\gamma-2n)} z^n
\]
con $q$ una normalización.
\[
	\sum_{n=0}^\gamma [ P(+,n) + P(-,n) ] = 1
\]
\[
	q = \sum_{n=0}^\gamma \binom{\gamma}{n} z^n 
	\left[ \euler^{2\beta J n} \cdot \euler^{-\beta J \gamma} + 
	\euler^{-2\beta J n} \cdot \euler^{\beta J \gamma} \right]
\]
y armando binomios dentro del paréntesis puede arribarse a
\notamargen{Estamos usando teorema del binomio, ponerlo en apéndice de cuentas.}
\[
	q = \left[ z\euler^{\beta J} + \euler^{-\beta J} \right]^\gamma +
	\left[ \euler^{\beta J} + z\euler^{-\beta J} \right]^\gamma
\]

Ahora se tendrá
\[
	\frac{N_+}{N} = \frac{1}{2}(L+1) = \sum_{n=0}^\gamma P(+,n) =
	\frac{1}{q} \left[ \euler^{\beta J} + z\euler^{-\beta J} \right]^\gamma
\]
\[
	\frac{N_{++}}{N\gamma/2} = \frac{1}{2}(\sigma+1) = \frac{1}{\gamma} \sum_{n=0}^\gamma n P(+,n) =
	\frac{z}{q} \euler^{\beta J} \left[ \euler^{-\beta J} + z\euler^{\beta J} \right]^{\gamma-1}
\]
y suponemos que estas dos ecuaciones se cumplen en toda la red.
Entonces tenemos $L,\sigma$ en función de $z$ y $T$.
Dado que los centros son indistinguibles de un vecino,
\[
	\sum_{n=0}^\gamma P(+,n) = \frac{1}{\gamma} \sum_{n=0}^\gamma n \left[ P(+,n) + P(-,n) \right]
\]
pero
\[
	\dpar{}{z}P(+,n) = P(+,n)\frac{n}{Z}
\]
de manera que 
\be
	z = \Frac{1+z\euler^{2\beta J}}{z+\euler^{2\beta J}}^{\gamma -1}
	\label{z_equation}
\ee
y podemos calcular
\[
	L = \frac{z^x - 1}{z^x + 1} \qquad \qquad \sigma = \frac{2z^2}{(1+z\euler^{-2\beta J})(1+z^x)} -1
\]
considerando $x\equiv \frac{\gamma}{\gamma -1}$

Pero \eqref{z_equation} debe hacerse gráficamente
\begin{itemize}
 \item $z=1$ es solución siempre 
 \item Si $z_0$ es solución, entonces $1/z_0$ también lo es
 \item $z=1$ hace $L=0$ y $z\to\infty$ hace $L=1$
\end{itemize}

DIBUJO

Hay que ver la pendiente $C$ de la curva azul en $z=1$,
\[=\]

\[=\]

\[=\]

La $T_c$ se impone desde
\[=\]
\[=\]
\[ 
	\boxed{ kT_c = \frac{2J}{\log \Frac{\gamma}{\gamma -2 } }}
\]

\[=\]
\[=\]
y en este último caso hay magnetización espontánea.

DIBUJO

El $c_V$ no se va a cero para $T>T_c$.
La solución exacta, Onsager, tiene allí una divergencia logarítmica.

\subsection{Cosas sin título}

dibujillos tipo tablero de ajedrez

\begin{center}
\begin{tabular}{l|l|l|l}
dibujo & dibujo & dibujo & dibujo \\
$ \displaystyle \frac{N_+}{N} =\frac{1}{2} \to L=0 $ & $ \displaystyle \frac{N_+}{N} =\frac{1}{2} \to L=0 $ & 
$ \displaystyle \frac{N_+}{N} =\frac{1}{2} \to L=0 $ & $ \displaystyle \frac{N_+}{N} =1 \to L=1 $ \\
 & & & \\
$ \displaystyle \frac{N_{++}}{\gamma/2 N} =\frac{1}{2} \to \sigma=0 $ & 
$ \displaystyle \frac{N_{++}}{\gamma/2 N} = 0 \to \sigma=-1 $ & 
$ \displaystyle \frac{N_{++}}{\gamma/2 N} =\frac{1}{4} \to \sigma=-\frac{1}{2} $ & 
$\displaystyle \frac{N_{++}}{\gamma/2 N} = 1 \to \sigma=1$ \\
 & & & \\
$\mathcal{M}=0$ & $\mathcal{M}=0$ & $\mathcal{M}=0$ & $\mathcal{M}=1$
\end{tabular}
\end{center}

Ahora las energías son
\[=\]
\begin{center}
\begin{tabular}{l|l|l|l}
dibujo & dibujo & dibujo & dibujo \\
 $E=-\frac{J\gamma N}{2}$ & & & 
\end{tabular}
\end{center}

Notemos que $N_{++}/(\gamma/2)N$ significa que todas las líneas $\oplus - \oplus$ dividido sobre todas
las líneas posibles $(\oplus - \oplus, \oplus - \ominus, \ominus - \ominus)$.

\subsection{Orden corto y orden largo}

dibujo bola spines

$N_+ / N$ : no me dice algo preciso en A. Creciendo hacia B va adquiriendo cada vez más sentido, entonces
es un parámetro global.

$N_{++}/(\gamma/2)N$ : tiene sentido en A. Creciendo hacia B ya en general no lo conservará, entonces es
un parámetro local.

\notamargen{En $N_{++}2/(\gamma N)$} note que nos paramos en un $\oplus$ para que tenga el sentido de
vecinos con $\oplus$.
\[
	A = U - TS
\]

Con $ST$ chicos la minimización de $A$ la domina $U$ (min.) pero con $U$ chicas la minimización de $A$ la domina
$TS$ (ma)

\subsection{Comentario magnetización}

DIBUJETE

Con $H=0$ es claro que deberíamos tener $ \braket{M}=0 $ por simetría entre $\oplus$ y $\ominus$.

Los dos ramos son equivalentes, pero el sistema cae en una u otra por una ``rotura espontánea de simetría''
llevada a cabo por un $ H \to 0 $ o por impurezas.

Acá hay una cuenta que no paso que tiene que ver con el $ \log Q $.

El mínimo de $A$ será 
\[ = \]

dibujo

dibujo

dibujo

La presencia de $ H \neq 0 $ añade un término 
\[
	A \approx - N H \mu I + ...
\]
que hará menor a $ A $ en $ + L_0 $ y mayor en $ - L_0 $. Rompe degeneración.

\[=\]

\subsection{Metropolis Monte-Carlo}

Repaso de teoría de procesos de Markov.

$Y$ es una variable estocástica en un espacio muestral $(y_1, y_2,...)$
Sea por simplicidad todo equiprobable, entonces
\[
	P(y_1) = \frac{1}{4}
\]
\[=\]
\notamargen{$Y$ puede tomar cualquier valor de su espacio muestral.}
\[=\]
Asumo que estoy con probabilidad 1 (certeza) en $y_1$ tengo tres flechas
\[ = \]

Las normalizaciones vienen de integrar
\begin{itemize}
 \item $ \int P(y_1) dy_1 = 1 $
 \item $ \int P(y_1|y_2) dy_2 = 1 $
\end{itemize}

Para procesos de Markov sólo interesará un único paso anterior. El sistema no tiene mucha memoria que 
digamos.
\[
	P(y_1|y_2) \equiv \text{ Probabilidad de transición }
\]
El proceso de Markov lo definimos por 
\[=\]
\[=\]
\[=\]
\[=\]

Sea ahora un espacio muestral discreto $(y_1,y_2,...,y_L)$ con el tiempo discretizado
\[=\]
\notamargen{Cadenas de Markov son en espacios discretos.}
\[=\]

La información de las transiciones se introduce en 
\[=\]

Para todo el sistema podemos definir un vector de dimensión $L$
\[=\]
\[=\]
\[=\]

Es regular la matriz estocástica $ Q $ si existe $ k : [ Q^k ]_{ij} > 0 \forall i,j $
Si es regular, entonces existe $ \$ : Q^\$ =  Q^{\$+1} $ y entonces $T = QT$ con $T\equiv Q^\$$.

Llegado un momento, $\$$ pasos, el sistema ya no cambia. A partir del paso $\$$ la matriz Q ya no cambia
la distribución en $\vec{P}$.

\notamargen{Hay una hoja con algunas preguntas pegada acá.}

\[=\]
\[=\]
\[
	\vec{P} = \vec{P} Q,
\]
lo cual define el equilibrio. Este punto fijo define
\[ =\]

% =================================================================================================
\section{Método de Metropolis Monte Carlo}
% =================================================================================================

Tenemos el espacio $ \mathbb{\Gamma} $ 3DN dimensional para un sistema de N partículas.
Un punto es el estado del sistema.

DIBUJETE

Queremos generar una cadena de Markov con probabilidades constantes de transición.

Subdividimos el volumen $ \mathbb{\Gamma} $  en S celdas; el estado del sistema a un paso $n$ es su
ubicación en la celda $S_k$, $k=1,2,...,S$. La energía en ese estado es $U_k$.

DIBUJO 

El sistema visita estados en un paso de tiempo (no es el tiempo físico)

\[=\]

que es la probabilidad de ir desde 1 a 2 (transición)

Los estados se ponderan de acuerdo a su energía $U_k$ para que aparezcan en la cadena con frecuencia
$ \propto \exp(-\beta U_k ) $

De esta forma la cuenta converge al canónico.
\[=\]
\[=\]
\[=\]
\[=\]

El tiempo medio de recurrencia es
\[=\]
con la condición 
\[=\]

Estamos viendo la suma entre todos los caminos para llegar desde $i$ a $j$; debe dar uno.
Siempre hay un camino entre dos estados (la red es conexa).

\[=\]

Un estado $i$ es recurrente si 
\[=\]
\[=\]
\[=\]
\[=\]


Si los estados pertenecen a la misma clase cumplen UNA de estas condiciones 
\begin{itemize}
 \item Son todos no recurrentes
 \item Son todos positivos
 \item Son todos nulos
\end{itemize}

Todos los estados están conectados en $ \mathbb{\Gamma} $  (su probabilidad es no nula) y entonces
si uno solo de ellos es nulo y son de la misma clase entonces deben ser todos nulos.
Lo mismo si uno es positivo, todos deben serlo.

DIBUJO 

Cadena ergódica: cadena irreducible finita, con todos sus estados aperiódicos. 
Se dará que:
\[=\]

Con infinitos loops la probabilidad de llegar desde cualquier 'j' no depende del punto inicial 'j'.
La probabilidad del estado 'k' tiene un valor asintótico estable
\[
	\pi_k = \frac{1}{n_k} \qquad \text{ (razonable) } 
\]
\[=\]
\[=\]

dibujin

\subsection{Metropolis}

\[= \text{ Promedio en el ensamble } \]

La probabilidad del estado asintótico del sistema será:
\[=\]
pero no conozco todos los posibles 's' y no puedo evaluar $\sum_s$.
Pido reversibilidad macroscópica y entonces
\[=\]
luego,
\[=\]

Entonces, si se da 
REVERSIBILIDAD + ERGODICIDAD + NORMALIZACIÓN
se tiene que la cadena converge.

El método:
Se propone una $P^*$ cadena de Markov con 
\[=\]
\[=\]
\[=\]
\[=\]

Tenemos tres situaciones $ \pi_i < \pi_j , \pi_i = \pi_j, \pi_i > \pi_j$
\[=\]
\[=\]

\subsection{Aplicación a Ising}

\[=\]
\[=\]
\[=\]

Las probabilidades de transición serán:
\[=\]
de modo que para un sistema con single spin flip se tendrá:
\begin{center}
\begin{tabular}{llll}
1 &  & $ \Delta E = 8J $  & El menos conveniente \\
4 &  & $ \Delta E = 4J $ &  \\
6 &  & $ \Delta E = 0 $ & Caso neutro \\
4 &  & $ \Delta E = -4J $ &  \\
1 &  & $ \Delta E = -8J $ & Es más conveniente
\end{tabular}
\end{center}

Con estas consideraciones tengo todas las probabilidades de MMC

tres dibujetes seguidos

% =================================================================================================
\section{Fenómenos críticos}
% =================================================================================================

Existe analogía entre sistemas magnéticos $(H,M)$ y gases $(p,V)$
\[=\]
\[=\]
\[=\]
\[=\]
\[=\]
\[=\]

Derivando ambos lados:

dibujos

$L$ revienta en $T=T_c$

y las derivadas segundas de $G$ discontinuas en $T_c$

Landau propone una teoría unificada de comportamiento de un sistema cerca del punto crítico.
Introduce el parámetro de orden $m_0$ que vale 0 si $T>T_c$ y $\neq 0$ si $T<T_c$.

La idea es expandir $ A = A(m_0) $.
Con $H=0$ el sistema es simétrico y entonces términos pares
\[=\]
\[=\]
\[=\]
\[=\]
\[=\]

Elegimos
\[=\]
\[=\]
\[=\]
\[=\]

dibujo

\[=\]
\[=\]

Si hay campo externo :
\[=\]
\[=\]

Con $f \to 0$ recordamos
\[=\]
\[=\]

Ambos revienta en $T_c$

DIBUJO

% =================================================================================================
\section{Exponentes críticos}
% =================================================================================================

\begin{itemize}
 \item Al cruzar el punto crítico el parámetro de orden crece.
 \item En la vecindad del punto crítico el sistema sobrelleva procesos de ajuste, entonces
 hay grandes fluctuaciones.
 \item Algunas fluctuaciones termodinámicas tienen diferentes comportamientos.
\end{itemize}

Los exponentes críticos describen la naturaleza de las singularidades en el punto crítico.
Son seis: $ \alpha, \beta, \gamma, \delta, \eta, \nu $
\[
	t = \frac{T-T_c}{T_c} \qquad \text{ parámetro } 
\]

Cerca del punto crítico las funciones termodinámicas pueden escribirse como:
\[=\]
\[=\]

\begin{center}
\begin{tabular}{lll}
 & expresión & rango \\
\hline
$C_M, C_v$ & $(-t)^{-\alpha}$ & $t < 0$  \\
× & × & ×\\
× & × & ×\\
× & × & ×\\
× & × & ×\\
× & × & ×\\
× & × & ×
\end{tabular}
\end{center}

Se cumplen ciertas desigualdades entre exponentes:
\[=\]
\[=\]

\subsection{Exponentes críticos Van Der Waals} 

\[=\]
\[=\]
\[=\]

Para la teoría de Landau es:
\[=\]
\[=\]
\[=\]

\subsection{Sobre trabajo y relación $p,V$ [mover]}

\[ 
	dU = dQ - p dV
\]

DIBUJO 

El sistema entrega trabajo al entorno si $ pdV > 0 $ y entonces $ dV > 0 $ (expansión)
pues $ p > p_{ext} $

El sistema absorbe energía en forma de trabajo si $ pdV < 0 $ y entonces $ dV < 0 $ (compresión)
pues $ p < p_{ext} $
\notamargen{$p$ y $V$ no varían separadamente; si $p$ sube entonces $V$ baja.}
\[=\]
\[=\]
\[=\]

Para el sistema magnético si $ H > 0 $ entonces $ dM > 0 $
\[=\]

Si el sistema se ordena $dM>0$ y baja su energía y por ende hace trabajo.
\notamargen{Este sistema hace trabajo no mecánico.}

\subsection{Comentarios varios [mover]}

Los efectos de la estadística cuántica surgen de la indistinguibilidad de las partículas.
\[=\]
Cuando $\lambda^3/v \approx 1$ surgen efectos cuánticos y tiene importancia la estadística de las
partículas (BE o FD).

Para un sistema de cuasipartículas es $\mu=0$. En general,
\[
	\mu = \mu(T)
\]
es un valor único para el sistema y depende de la temperatura.
\[=\]
\[=\]
\[=\]

RESTA UNA HOJA DE EXPRESIONES Y CUENTAS QUE DEJAMOS PARA EL futuro APÉNDICE.

% \bibliographystyle{CBFT-apa-good}	% (uses file "apa-good.bst")
% \bibliography{CBFT.Referencias} % La base de datos bibliográfica

\end{document}
