	\documentclass[10pt,oneside]{CBFT_book}
	% Algunos paquetes
	\usepackage{amssymb}
	\usepackage{amsmath}
	\usepackage{graphicx}
	\usepackage{libertine}
	\usepackage[bold-style=TeX]{unicode-math}
	\usepackage{lipsum}

	\usepackage{natbib}
	\setcitestyle{square}

	\usepackage{polyglossia}
	\setdefaultlanguage{spanish}
	



	\usepackage{CBFT.estilo} % Cargo la hoja de estilo

	% Tipografías
	% \setromanfont[Mapping=tex-text]{Linux Libertine O}
	% \setsansfont[Mapping=tex-text]{DejaVu Sans}
	% \setmonofont[Mapping=tex-text]{DejaVu Sans Mono}

	%===================================================================
	%	DOCUMENTO PROPIAMENTE DICHO
	%===================================================================

\begin{document}

% =================================================================================================
\chapter{Gas de Fermi}
% =================================================================================================

DIBUJOS
\[
	\braket{n_e} = \frac{1}{z^{-1} \euler^{\beta e} + 1 } = \frac{1}{ \euler^{\beta ( \mu - e) } + 1 }
\]
Si $ \mu < 0 $ como $ e > 0 $ siempre, ni aún en el estado de más baja energía se llega a ocupar el
nivel (restan muchos niveles vacíos).

Sea que $ T \to \infty $ entonces $ \beta \to \infty $ y se sigue que 
\[
	\euler^{\beta(e-\mu)} \to \infty e> \mu
\]
\[
	\euler^{\beta(e-\mu)} \to 0 e< \mu
\]
\[
	\euler^{\beta(e-\mu)} \to 1 e = \mu
\]
Luego, con $ T = 0 $ es Fermi un escalón. El valor de $ \mu $ que determina el último
estado ocupado se llama $ e_F$ 

DIBUJO

\[
	f_{3/2}(z) = \frac{\lambda^3}{v} = \int_0^{\xi = \beta\mu } \frac{x^{1/2}}{\Gamma(3/2)3/2}dx =
	\frac{4}{3} \frac{1}{\pi^{1/2}} ( \beta\mu )^{3/2} = 
	\frac{4}{3} \frac{1}{\pi^{1/2}} ( \beta e_F )^{3/2}
\]

% =================================================================================================
\section{Análisis del gas ideal de Fermi}
% =================================================================================================

La primera aproximación consiste en 
\begin{itemize}
 \item Caso no degenerado : $\frac{\lambda^3}{v} \ll 1 $  que lleva a $ T $ alta y $ v $ alto
 por ende $ N/V $ chico.
 \[	
	z \ll 1 \qquad f_\nu(z) \approx z \qquad \frac{\lambda^3}{v} \approx z
 \]
 Si vale la condición entonces 
 \[
	\frac{\lambda^3}{v} = \sum_{l=1}^\infty \frac{(-1)^{l+1} z^l }{l^{3/2}} \ll 1 \qquad z \ll 1
 \]
 \[
	\beta p V \approx 1 + \frac{\lambda^3}{v 2^{5/2}} \qquad \qquad U = \frac{3}{2} \frac{N}{\beta}
	\left( 1 + \frac{\lambda^3}{v 2^{5/2}} \right)
 \]
 \item $\frac{\lambda^3}{v} < 1 $ entonces $ z < 1 $ y hay que expandir el virial,
 \[
	\beta p V = \sum_{l=1}^\infty (-1)^{l-1} a_l \left(\frac{\lambda^3}{v} \right)^{l-1}
 \]
 que igualando coeficientes se hace (¿?)
 \notamargen{ $\lambda^3 / v $ a orden 1 hay efectos cuánticos }
 \[
	f_{5/2}(z) = f_{3/2}(z) \cdot \sum_{l=1}^\infty (-1)^{l-1} a_l \left(\frac{\lambda^3}{v} \right)^{l-1}
 \]
 \item $\frac{\lambda^3}{v} \approx 1 $ Cálculo numérico
 \item Caso altamente degenerado : $\frac{\lambda^3}{v} \gg 1 $ se tiene $ z \gg 1 $ 
 Se puede expandir $ f_\nu(z) $ en función de $ (\log )^{-1} $ mediante lema de Sommerfeld
 \notamargen{ $ z \ggg 1 $ entonces $ \log z \gg 1 $ $ ( \log z )^{-1} \ll 1 $ $ \log z = \beta \mu $ }
 \[
	f_{5/2}(z) = \frac{8}{15\pi^{1/2}} (\log z)^{5/2} \left[ 1 + \frac{5\pi^2}{8}(\log z)^{-2} + ... \right]
 \]
 \[
	f_{3/2}(z) = \frac{4}{3\pi^{1/2}} (\log z)^{3/2} \left[ 1 + \frac{\pi^2}{8}(\log z)^{-2} + ... \right]
 \]
 y entonces
 \[
	\frac{\lambda^3}{v} = \frac{4}{3\pi^{1/2}} (\log z)^{3/2}  \quad \text{ a orden 0 }
 \]
 \[
	\frac{h^3}{ (2\pi mkT)^{3/2} } \frac{N}{V} \frac{3\pi^{1/2}}{4} (kT)^{3/2} = \mu^{ 3/2 }
 \]
 \[
	\frac{ h^3 }{ \pi } \frac{ N }{ V } \frac{ 3 }{ ( 2m )^{ 3/2 } 4 } = \mu^{ 3/2 } = e_F^{3/2}
 \]
 \[
	\frac{\lambda^3}{v}\frac{3\pi^{1/2}}{4} (kT)^{3/2} = 
	\mu^{3/2}\left[ 1 + \frac{\pi^2}{8}(\log z)^{-2} + ... \right]
 \]
 \[
	\frac{ h^3 }{ \pi } \frac{ N }{ V } \frac{ 3 }{ ( 2m )^{ 3/2 } 4 } = e_F^{3/2} \approx
	\mu^{3/2} \left[ 1 + \frac{\pi^2}{8}(\log z)^{-2} \right]
 \]
 \[
	e_F \approx \mu \left[ 1 + \frac{\pi^2}{8}( \frac{ \mu }{ kT } )^{-2} \right]^{ 2/3 } \approx 
	\mu \left[ 1 + \frac{\pi^2}{12} \Frac{ kT }{ \mu }^{2} \right]
 \]
 \notamargen{Anoté {\it investigar este pasaje}. }
 \[
	e_F \approx \mu \left[ 1 - \frac{\pi^2}{12}( \frac{ kT }{ e_F } )^{2} \right]
 \]
 y consideramos
 \[
	\frac{1}{\mu^2} \approx \frac{1}{e_F^2}
 \]
 pués $ \mu $ es muy grande.
 \[
	\beta p v = \frac{ f_{5/2}(z) }{ f_{3/2}(z) } \approx \frac{ 2 \beta \mu }{ 5 } 
	\left[ 1 + \frac{ 5\pi^2 }{ 8 } \left( \frac{kT}{\mu} \right)^2 \right]
	\left[ 1 - \frac{ \pi^2 }{ 8 } \left( \frac{kT}{\mu} \right)^2 \right]
 \]
 Hasta orden dos en $ T $ resulta 
 \[
	pv \approx \frac{ 2 \mu }{ 5 } \left[ 1 + \frac{ \pi^2 }{ 2 } \left( \frac{kT}{\mu} \right)^2 \right] =
	\frac{ 2 e_F }{ 5 }\left[ 1 - \frac{ \pi }{ 12 } \left( \frac{kT}{e_F} \right)^2 \right] 
	\left[ 1 + \frac{ \pi^2 }{ 2 } \left( \frac{kT}{e_F} \right)^2 \right] 
 \]
 \[
	pv \approx \frac{ 2 e_F }{ 5 } \left[ 1 + \frac{ 5 \pi^2 }{ 12 } \left( \frac{kT}{e_F} \right)^2 \right] 
 \]
 \[
	U = \frac{3}{2} p v \approx \frac{3}{5} N e_F 
	\left[ 1 + \frac{ 5 \pi^2 }{ 12 } \left( \frac{kT}{e_F} \right)^2 \right] 
 \]
 \[
	C_V = \dpar{U}{T} \approx \frac{ N \pi^2 k^2 T }{ 2e_F } \qquad C_V \propto T
 \]
 \[
% 	C_V \approx \frac{\pi^2}{2} Nk \left( \frac{T}{T_F} \right)
	C_V \approx \frac{\pi^2}{2} Nk \Frac{T}{T_F}
 \]
 DIBUJO 
 $T_F$ siempre estará ene general en la zona clásica donde no vale la aproximación degenerada.
 
 Calor específico Fermi (¿?)
 \item Caso totalmente degenerado : $\frac{\lambda^3}{v} \to \infty \qquad (T \to 0) \qquad z \to \infty $
 
 La distribución de estados es escalón,
 \[
	\braket{N} = \frac{ 4 \pi V }{ h^3 } \int_0^{p_F} p^2 \Frac{ 1 }{ z^{-1} \euler^{\beta p^2 / 2m } + 1} dp
 \]
 \notamargen{$ z = \euler^{\beta\mu} $ y $z(T\to 0) = \euler^{\beta e_F } \to \infty $}
 \[
	\braket{N} = \frac{ 4 \pi V }{ h^3 } \int_0^{p_F} p^2 dp
 \]
 
 Notemos que 
 \notamargen{Teniendo el límite sale la cuenta}
 \[
	pV = \frac{ 4 \pi V }{ h^3 } \int_0^{p_F} p^2 kT \log (1 + \euler^{ -1/kT( p^2/2m - \mu_0 )} ) dp
 \]
 tiene un comportamiento no trivial con $ T \to 0 $. Si $ kT \to 0 $ entonces si $e > \mu_0$ el $\log \to 0$
 y si $e < \mu_0$ el $\log \to \infty $.
 Parecería que con $ T \to 0 $ es
 \[
 	pV = \frac{ 4 \pi V }{ h^3 } \int_0^{p_F} p^2 \left( \frac{ p^2 }{ 2m } - \mu_0 \right) dp
 \]
 y haciendo el cambio de variables de acuerdo a $ p^2 / 2m = e $, que lleva a $ pdp = m de $, se tiene 
 \[
 	pV = \frac{ 4 \pi V }{ h^3 } \int_0^{e_F} \sqrt{2e} m^{3/2} ( e -\mu_0 ) de
 \]
 \[
	pV = \frac{ 4 \pi V }{ h^3 } 2^{ 1/2 } m^{ 3/2 } 
	\left( \frac{e_F^{ 5/2 }}{5/2} - \mu_0 \frac{e_F^{ 5/2 }}{3/2} \right) =
	\frac{ 4 \pi V }{ h^3 }2^{ 1/2 } m^{ 3/2 } e_F^{ 5/2 } \frac{ 4 }{ 15 }
 \]
 \[
	U = \frac{3}{2} p V = \frac{ 4 \pi V }{ h^3 }2^{ 1/2 } m^{ 3/2 } e_F^{ 5/2 } \frac{ 2 }{ 5 }
 \]
 \[
	p = \frac{2}{5} e_F \frac{\braket{N}}{V} \qquad U = \frac{3}{5} e_F \braket{N} 
 \]
 A $ T = 0 $ tenemos presión y energía no nulas; las partículas no se acomodan todas en un único nivel energético
 (exclusión de Pauli).
 Para $ T \approx 0 $ ( $T$ bajas) el escalón en estados apenas se desdibuja
 
 DIBUJO.
 
\end{itemize}

% =================================================================================================
\section{Cuánticos III --reubicar--}
% =================================================================================================
 
 \subsection{Los números de ocupación}
 
 DIBUJO
 
 Se ve que para Bose $ \mu < 0 $ siempre pero $ \braket{n} \to \infty $ si $ \mu \to 0^+ $.
 El gráfico es para $T$ alta. Con $ T $ bajas todo tiende a suceder más pegado al eje $ \beta(e-\mu) = 0 $
 
 \subsection{Comportamiento de $ f_{3/2}(z) $}
 
 \[
	f_{3/2}(z) = \sum_{j=1}^\infty (-1)^{j+1}\frac{z^j}{j^{3/2}} \approx z - \frac{z^2}{2^{3/2}} \qquad 
	\text{ $z$ chico }
 \]
 \[
 	f_{3/2}(z) = \frac{1}{\Gamma(3/2)} \int_0^\infty \frac{x^{1/2}}{z^{-1}\euler^{x} + 1} dx \approx  
 	\frac{1}{\Gamma(3/2)} \int_0^{\log z = \beta\mu} x^{1/2} dx
 \]
 
Notemos que con $ \beta \mu $ grande el integrando es 1 o 0 (DIBUJO); en realidad es un escalón en el límite
en que $ \xi \equiv \beta\mu \to \infty$
\notamargen{Definimos $ \log z \equiv \xi $ para no especular con temperaturas. }

\[
	f_{3/2}(z) = \frac{4}{ 3 \sqrt{\pi} } (\log z )^{3/2} \text{ $z$ muy alto }
\]
\[
	f_{3/2}(z) = \frac{4}{ 3 \sqrt{\pi} } \left[ (\log z )^{3/2} + \frac{\pi^2}{8}(\log z )^{-1/2} + ... \right]
\]

El valor $ \lambda^3 / v $ determina relación entre $ T,V,N $ que son los parámetros macroscópicos que uno fija.

\subsection{Casos}

\begin{itemize}
 \item Comportamiento clásico: $\frac{\lambda^3}{v} \ll 1$ Altas $T$ y bajas $n\equiv \frac{N}{V}$ 
 \[
	\frac{\lambda^3}{v} = f_{3/2}(z) \approx z - \frac{z^2}{2^{3/2}}
 \]
 y por inversión de la serie 
 \[
	z = \frac{\lambda^3}{v} + \Frac{\lambda^3}{v}^2 2^{-3/2}
 \]
 \notamargen{Sabemos que en Boltzmann es $\frac{\lambda^3}{v} = z$ }
 y entonces si $\frac{\lambda^3}{v} \ll 1$ se tiene que $ z \ll 1 $
 \[
	\frac{pv}{kT} = \frac{v}{\lambda^3} f_{5/2}(z) \qquad \qquad \frac{\lambda^3}{v} = f_{3/2}(z)
 \]
 \[
	\frac{pv}{kT} = \frac{f_{5/2}(z)}{f_{3/2}(z)} \approx \frac{z - z^2/2^{5/2}}{z - z^2/2^{3/2}}
	\approx 1 + \frac{1}{2^{3/2}} \Frac{\lambda^3}{v}
 \]
 siendo el último término una corrección cuántica.
 
  \item Comportamiento cuántico : $\frac{\lambda^3}{v} \gg 1$ Bajas $T$ y altas $n\equiv \frac{N}{V}$ 
  
  A $ T = 0 $ determinamos la $ e_F $ como (con el límite de $T\to 0$)
  \[
	\frac{\lambda^3}{v} = \frac{1}{\Gamma(3/2)} \int_0^{\log z = \beta\mu} x^{1/2} dx = 
	 \frac{4}{ 3 \sqrt{\pi} } (\log z )^{3/2}
  \]
  \[
	\Frac{ 3  \lambda^3  \sqrt{\pi} }{4v}^{2/3} = \Frac{ 3  h^3  \sqrt{\pi} }{4(2\pi m k T)^{3/2}v}^{2/3} 
	= \log z =  \beta e_F
  \]
  \[
	\frac{h^2}{2m}\Frac{3}{4\pi v}^{2/3} = e_F = \frac{\hbar}{2m}\Frac{6\pi^2}{v}^{2/3}
  \]
  A $ T = 0 $ la ocupación por nivel es un escalón ($e_F = \mu(T=0) $) 
  \[
	\braket{n_e} = 	\begin{cases}
			1 \qquad e < e_F \\
			0 \qquad e > e_F
			\end{cases}
  \]
\end{itemize}

\subsection{Funciones termodinámicas con $T$ baja y $n$ alta}

Usamos Sommerfeld
\[
	\frac{\lambda^3}{v} = f_{3/2}(z) \qquad  \qquad \mu = e_F
\]
orden 1
\[
	\frac{\lambda^3}{v} =  
	\frac{4}{3\sqrt{\pi} } (\log z )^{3/2} \left[ 1 + \frac{\pi^2}{8}(\log z )^{-2} \right] 
\]
\[
	\frac{\lambda^3}{v} \frac{3\sqrt{\pi} }{4} \left[ 1 + \frac{\pi^2}{8}(\log z )^{-2} \right]^{-1} 
	\approx (\log z)^{3/2} 
\]
\[
	e_F \left( 1- \frac{\pi^2}{12}\Frac{T}{T_F}^2 \right) \approx \mu( T ) \text{ cumple $\mu(T=0)=e_F$}
\]
Puede verse que con $ \frac{\lambda^3}{v} \ggg 1 $ ($T$ baja y $n$ alta) es
\[
	C_V \approx \frac{N \pi^2 k^2 T}{2 e_F}
\]
DIBUJO

Aún a $T=0$ hay presión no nula pero $S \to 0$ con $T \to 0$ respetando la tercera ley.
Existe una relación de recurrencia
\[
	z\dpar{}{z} f_\nu(z) = z\dpar{}{z} \sum_{l=1}^\infty (-1)^{l+1} \frac{z^l}{l^\nu} =
	\sum_{l=1}^\infty (-1)^{l+1} \frac{l z^{l-1} z}{l^\nu} = 
	\sum_{l=1}^\infty (-1)^{l+1} \frac{z^l}{l^{\nu-1}} = f_{\nu-1}(z)
\]
\[
	f_\nu(z) = \int \frac{1}{z} f_{\nu-1}(z) dz
\]
\[
	f_{3/2}(z) \approx \frac{4}{3\sqrt{\pi}} (\log z)^{5/2}
\]
entonces
\[
	f_{5/2}(z) = \int dz \frac{4}{3\sqrt{\pi}} \frac{(\log z)^{3/2}}{z} =
	 \frac{4}{3\sqrt{\pi}} \int dz \frac{ 2 }{ 5 } \dpar{}{z} (\log z)^{5/2} =
	\frac{8}{15\sqrt{\pi}}(\log z)^{5/2}
\]
\notamargen{Usamos $d(\log z)^n = n (\log z)^{n-1}/z $}

\notamargen{¿Y estas cuentas sueltas?}
\[
	n_x^2 + n_y^2 + n_z^2 = r^2 \qquad V=\frac{4}{3}\pi r^3 \qquad dV = 4\pi r^2 dr
\]
\[
	E = \frac{(\hbar \pi)^2}{2ma^2}r^2 \qquad \qquad dE = \frac{(\hbar \pi)^2}{ma^2}r dr
\]
\[
	N(r) dr = \frac{\pi}{2} r^2 dr 
\]
será lo mismo que el incremento en niveles energéticos
\[
	N(e)de = \frac{m^2 a^3}{\pi^2 \hbar^3} \Frac{E}{2}^{1/2} dE
\]
Pensamos un conjunto de nucleones como un gas de Fermi. 
\notamargen{Recordemos que a $T=0$ era $pV=2/5 N e_F$ y $U=3/5 N e_F$}
Claramente
\[
	N = 2 \int_0^{e_F} N(e) \; de
\]
porque tenemos la ocupación en función de la energía
\[
	e_F \propto \Frac{N}{V}^{2/3} \text{ según la definición de $e_F$ }
\]

Para un núcleo se tienen N=A-Z nucleones, siendo Z protones y A nucleones.
\[
	E = \frac{3}{5}N_Te_F (\text{ a } T=0)
\]
\[
	E = \frac{3}{5}\left[ Z\Frac{Z}{V}^{2/3} + (A-Z)\Frac{A-Z}{V}^{2/3} \right] =
	\frac{3}{5}\Frac{ Z^{5/3} + (A-Z)^{5/3} }{A^{2/3}}
\]
\notamargen{$V \propto A$}
donde hemos supuesto ambos pozos iguales. Si los pozos no fueran iguales cambia la
$e_F$.

Se minimiza $E$ con $Z=A/2$ (simetría)
\[
	f_4 \propto E-E_0 = \frac{3}{5A^{2/3}} \left[ Z^{5/3} + (A-Z)^{5/3} - 2(A/2)^{5/3} \right]
\]
que se puede reescribir como 
\[
	f_4 \propto \frac{(A/2 - Z)^2}{A} \text{ término de simetría }
\]



% \bibliographystyle{CBFT-apa-good}	% (uses file "apa-good.bst")
% \bibliography{CBFT.Referencias} % La base de datos bibliográfica

\end{document}
